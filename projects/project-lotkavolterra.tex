\begin{emphbox}[]
%	\begin{quote}
		Snow collects on the brim of your fur coat and musket 
		as you stalk your prey through the white woods.  A flash 
		of orange, a rustling of branches, and then it's
		gone.
		You mutter a curse under your breath.  Foxes are scarce this year, 
		and you'll have to explain to the Dutch East India Trading Company 
		why you've come up short.  Worse yet, your rival, a
		trapper who only hunts 
		rabbits, is having a terrific year.  You shouldn't have teased him so 
		much when rabbits were down and foxes were up just a few seasons ago.

		If only you could somehow predict which game would be plentiful, you could always bid on the easier contract!  But how?

		Back at camp, amid the crackling of your lonely fire, the answer comes to you.  Just two months ago you attended a talk by Dr.~\!Lotka on autocatalytic chemical reactions.  It was quite a spectacle when, after Dr.~\!Lotka had finished talking, a Dr.~\!Volterra stood up and proclaimed 	that he had applied the same model to predator-prey ecology. At the time you were rushed and didn't think much about the proclamation, but now the 	basic assumptions were making more and more sense:

		\begin{enumerate}[label=\emph{(\alph*)}]
			\item In the absence of foxes, the rabbit population grows
				at a rate proportional to the number of rabbits.
			\item In the absence of rabbits, the fox population declines
				at a rate proportional to the number of foxes.
			\item The population of rabbits declines at a rate proportional
				to the product of the rabbit and fox populations.
			\item The population of foxes grows at a rate proportional 
				to the product of the rabbit and fox populations.
		\end{enumerate}

		Pop! A hot coal explodes, snapping you out of your pondering state
		and into one of action.  Grabbing a piece of paper from your limited
		supplies, you begin to grapple with the consequences of the
		Lotka--Volterra model.
%	\end{quote}
\end{emphbox}

\emph{Task.} 
	Let $R$ and $F$ stand for the rabbit and fox populations, respectively, and
	let $\alpha$, $\beta$, $\gamma$, and $\delta$ be the constants of proportionality for
	parts \emph{(a)}--\emph{(d)}.

	Work through the following before you begin your report.

	\begin{enumerate}[label=\emph{\arabic*.}]
		\item Write down the Lotka-Volterra system of differential equations.  For
			each of (a)--(d), explain whether or not the assumption is reasonable.
		\item When is the fox population increasing or decreasing?  Given $R$ and $F$,
			could you predict which one is on the rise on which one is on the decline?
		\item Is there a steady state for the fox population?  Could the fox population
			remain steady while the rabbit population is changing?
		\item \label{params}
			Sketch an $RF$-phase portrait for the Lotka-Volterra system of differential
			equations with the following constants:
			\begin{align*}
				\alpha &= 0.2\text{ rabbits per month per rabbit}\\
				\beta &= 0.1\text{ foxes per month per fox}\\
				\gamma &= 0.002\text{ rabbits per month per rabbit-fox}\\
				\delta &= 0.001\text{ foxes per month per rabbit-fox}
			\end{align*}

			Hint: you will need to consider rabbit and fox populations of well over
			100 to see interesting behaviour in your phase portrait.
		\item Does your phase portrait have any singular points?  What do they mean?
		\item Use technology to graph $R(t)$ and $F(t)$ for some initial conditions.  
			Do the initial conditions affect the period of the population increase
			or decrease?  Does this seem reasonable when looking at your phase portrait?
	\end{enumerate}


	Your writeup should include the following:
	\begin{itemize}
		\item An explanation of the Lotka-Volterra model along with a discussion of
			whether or not each assumption is reasonable.
		\item A description of what behaviour you expect from which initial conditions.  You may use
			the parameters specified in question \ref{params}.  Include a phase portrait in
			your description as well as how to interpret the phase portrait, and make sure to point
			out any critical points.
		\item Suppose you wanted to legislate limits on the hunting of rabbits and foxes to ensure the population of either
			never dipped below a certain level.  Based on the Lotka-Volterra model, propose
			legislation.  Be specific and comment on whether a flat-out hunting ban
			would achieve the desired effect.
	\end{itemize}


	Be careful with your simulations.  Euler's method loses accuracy quickly on Lotka-Volterra--based systems.




\vfill

\hfill ``Hunting Inspiration'' is a collaboration with Max Brugger.




\begin{noexercises}
\end{noexercises}