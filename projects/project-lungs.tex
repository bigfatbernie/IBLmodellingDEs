
Lungs are somewhat important to human beings! 
They are the source of oxygen to our bodies, so it is important to maximize the amount of oxygen that can be absorbed within the available volume.

\emph{Task 1. }Let us find out the volume and surface area of the lungs.
\begin{enumerate}[label=\emph{(\alph*)}]
	\item The lungs are composed of a branched structure as in the figure below.
	\begin{center}
		\begin{tikzpicture}[scale=0.5]
			\draw[ultra thick] (0,0) ellipse (2 and 1);
			\draw[ultra thick] (-2,0)--(-2,-4);
			\draw[ultra thick] (2,0)--(2,-4);
			\draw[ultra thick] ($(0, -4) + (180:2cm and 1cm)$(P) arc  (180:360:2cm and 1cm);
			\draw[thick] (-2,-4)--(-3.075,-7.1);
			\draw[thick] (0,-5)--(-0.96,-7.45);
			\draw[thick] ($(-2, -7.25) + (165:1.1cm and 0.55cm)$(P) arc  (165:350:1.1cm and 0.55cm);
			\draw[thick] (2,-4)--(3.075,-7.1);
			\draw[thick] (0,-5)--(0.96,-7.45);
			\draw[thick] ($(2, -7.25) + (190:1.1cm and 0.55cm)$(P) arc  (190:375:1.1cm and 0.55cm);
			\draw (-3.075,-7.1)--(-4.2,-8.5);
			\draw (-2.125,-7.8)--(-2.93,-8.9);
			\draw ($(-3.62, -8.6) + (160:0.6cm and 0.4cm)$(P) arc  (160:330:0.7cm and 0.5cm);
			\draw (-2.125,-7.8)--(-2.125,-9);
			\draw (-0.96,-7.45)--(-0.96,-9);
			\draw ($(-1.5425, -9) + (180:0.5825cm and 0.3cm)$(P) arc  (180:360:0.5825cm and 0.3cm);
			
			\draw (3.075,-7.1)--(4.2,-8.5);
			\draw (2.125,-7.8)--(2.93,-8.9);
			\draw ($(3.45, -8.7) + (210:0.6cm and 0.4cm)$(P) arc  (210:380:0.7cm and 0.5cm);
			\draw (2.125,-7.8)--(2.125,-9);
			\draw (0.96,-7.45)--(0.96,-9);
			\draw ($(1.5425, -9) + (180:0.5825cm and 0.3cm)$(P) arc  (180:360:0.5825cm and 0.3cm);
		\end{tikzpicture}
	\end{center}
		The first segment is very large, then it bifurcates into smaller segments in a geometrical pattern.
		
		Here is some data about human lungs:
		\begin{itemize}
			\item radius of first segment: $r_0 = 0.5$cm
			\item length of first segment: $\ell_0 = 5.6$cm
			\item ratio of daughter to parent length: $\alpha = 0.9$
			\item ratio of daughter to parent radius: $\beta = 0.86$
			\item number of branch generations: $M=30$
			\item average number of daughters per parent: $b = 1.7$
		\end{itemize}
		
		In the figure, there are 2 daughters per parent, in 	real lungs, it isn't perfectly regular, so we have an average that is not a whole number.
		
		
		
	\item Calculate the volume inside the segments. What is the limit as the number of segments gets larger and larger?
	\item Calculate the surface area inside the segments. What is the limit as the number of segments gets larger and larger?

	\item If we had $b=2$ instead and the same number of generations, would that be possible? If not, how many generations would be possible?
\end{enumerate}


%\vfill
\newpage
\emph{Task 2. } Let us model the gas exchange that happens inside the lungs.
\begin{enumerate}[label=\emph{(\alph*)}]
\item Suppose that a lung has a volume of 3L when full. With each breath, 0.6L of the air is exhaled and replaced by 0.6L of outside air.

	After exhaling the volume is 2.4L and it returns to 3L after inhaling.
	
	Suppose further that the lung contains a chemical with a concentration of 2 milimoles per litre before exhaling (a mole is a chemical unit for $6.02 \times 10^{23}$ molecules). The ambient air has a concentration of 5mmol/L of the same chemical.

	What is the concentration after one breath? What is the concentration after $n$ breaths?

\item Update your model to match the oxygen exchange inside a real human lung.
	
\item The model above ignored the fact that the body absorbs some of the oxygen. Assume now that the lungs absorb 30\% of the oxygen with each breath. Update your model.

\item What is the equilibrium concentration of oxygen in the lungs? Find the graph of the equilibrium concentration as a function of the fraction of oxygen absorbed with each breath.

\end{enumerate}

\begin{graybox}
You may choose to model the gas exchange in the lungs using either a discrete time \emph{difference equation} or a continuous time \emph{differential equation}.
	\begin{enumerate}[label=\emph{(\alph*)}]
	\item If you choose a difference equation, you might assume that every breath the concentration of the chemical changes. If $c(n)$ or $c_n$ represented the concentration of chemical after $n$ breaths, your difference equation might look like
	\begin{itemize}
		\item $\Delta c(n) = c(n) - c(n-1) =$ some function at time $n$, where $n$ is only allowed to take whole numbers.
	\end{itemize}

	\item If you choose to use differential equations, the analogous equation would look like
	\begin{itemize}
		\item $c'(t)=$ some function at $t$, where $t$ can take any positive real value.  
	\end{itemize}
\end{enumerate}
\end{graybox}

\vfill

\emph{Task 3. } Assume that the lungs only absorb a fraction of the air in contact with its surface. Combine the two previous tasks.

\vfill

\emph{Further Investigation:}
\begin{enumerate}[label=\emph{\arabic*.}]
%	\item Assume that the lungs only absorb a fraction of the air in contact with its surface. Combine the two tasks.
	\item Investigate how the what is known about human lungs. Compare how the branching of real human lungs differs from this model. Refine the estimate.
	\item Investigate how the what is known about human lungs. Compare how the gas exchange of real human lungs differs from this model. Refine the model.
\end{enumerate}

\vspace{1cm}


\begin{graybox}
\hfill ``\lungstitle'' is a collaboration with Kseniya Garaschuk and Miroslav Lovric.
\end{graybox}
\begin{noexercises}
\end{noexercises}
