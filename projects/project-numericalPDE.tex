\emph{Task. } We want to approximate solutions of a PDE. \\




\begin{minipage}{11cm}

The heat equation is
$$
\frac{\partial^2 u}{\partial x^2} + \frac{\partial^2 u}{\partial y^2} = 0,
$$
where $u(x,y)$ is the equilibrium temperature at the position $(x,y)$ given some boundary conditions. \\

This is a Partial Differential Equation (PDE), which we don't know how to solve. We can however obtain an approximation of the solution.

In this example, the domain is shown on the right $\Omega = [0,2] \times [0,2]$ and the initial conditions are the following
$$
u(x,0)=90 
\quad , \quad u(x,2)=30
\quad , \quad u(0,y)=0
\quad , \quad u(2,y)=60.
$$
\hfill
\end{minipage}
\hfill
\begin{minipage}{150pt}
\includegraphics*[width=150pt]{images/project-numericalPDE-domain.pdf}
\end{minipage}

To approximate the solution, we divide the domain in $N$ small pieces. In the example $N=4$ and $\Delta = \frac{2-0}{N}=\frac12$.

Then we define the points
$$
\vec{p}_1, \vec{p}_2, \vec{p}_3, \ldots , \vec{p}_M,
$$
as the points in the interior of the domain (usually by moving left$\to$right and bottom$\to$top).

\begin{enumerate}[label=\emph{\arabic*.}] 
\item What are the points $\vec{p}_n$? What is $M$?
\end{enumerate}

Then we define
$$
u_n = u(\vec{p}_n),
$$
where $u(x,y)$ is the solution of the initial-value problem above (PDE with boundary conditions). \\

The next step is to approximate the PDE itself. We do that by approximating the derivatives:
$$
\frac{\partial u}{\partial x}(x_0,y_0) \approx \frac{u(x_0+\Delta,y_0)-u(x_0,y_0)}{\Delta}.
$$

\begin{enumerate}[resume, label=\emph{\arabic*.}] 
\item What is the approximation for $\frac{\partial u}{\partial x} (\vec{p}_5)$ ?
What is the approximation for $\frac{\partial u}{\partial x} (\vec{p}_3)$ ?

\item What is an approximation for $\frac{\partial u}{\partial y}(x_0,y_0)$?
What is the approximation for $\frac{\partial u}{\partial y} (\vec{p}_8)$ ?
\end{enumerate}


From here, we define the second derivative in a similar fashion:

\begin{align*}
\frac{\partial^2u}{\partial x^2}(x_0,y_0) 
	& \approx \frac{\dfrac{u(x_0+\Delta,y_0)-u(x_0,y_0)}{\Delta} - \dfrac{u(x_0,y_0)-u(x_0-\Delta,y_0)}{\Delta}}{\Delta}  \\
	& = \frac{u(x_0+\Delta,y_0) - 2 u(x_0,y_0) + u(x_0-\Delta,y_0)}{\Delta^2}.
\end{align*}

\begin{enumerate}[resume, label=\emph{\arabic*.}] 
\item What is the approximation for $\frac{\partial^2 u}{\partial x^2} (\vec{p}_5)$ ?
What is the approximation for $\frac{\partial^2 u}{\partial x^2} (\vec{p}_3)$ ?

\item What is an approximation for $\frac{\partial^2 u}{\partial y^2}(x_0,y_0)$?
What is the approximation for $\frac{\partial^2 u}{\partial y^2} (\vec{p}_8)$ ?
\end{enumerate}

We are now ready to put it all together. 

The PDE applies to all points in the domain. Instead of applying the PDE to all points $(x,y) \in \Omega$, we apply the approximation of the (second) derivatives to all the points $\vec{p}_n$.


\begin{enumerate}[resume, label=\emph{\arabic*.}] 
\item What is the equation that we obtain for the point $\vec{p}_5$?

\item What is the equation for each point $\vec{p}_n$?
\end{enumerate}


These equations form a linear system of equations.
Define a vector $\vec{u}  = \begin{bmatrix}
u_1 \\
u_2 \\
\vdots \\
u_M
\end{bmatrix}
$

\begin{enumerate}[resume, label=\emph{\arabic*.}] 
\item Write the system of equations in matrix form $\textbf{A} \vec{u} = \vec{b}$.

\item Solve it and plot the solution. (You should use some software to solve this!)

\end{enumerate}

\vfill

\emph{MATLAB. } Here is a quick introduction to some tools in MATLAB that are useful for this problem.

\begin{itemize}
\item Define a matrix $\textbf{A} = \begin{bmatrix}  1 & 2 \\ 3 & 4 \end{bmatrix}$ by
\begin{center}
\tt >> A=[1,2;3,4]
\end{center}

\item Define a vector $\vec{b} = \begin{bmatrix}  5 \\ 6 \end{bmatrix}$ by
\begin{center}
\tt >> b=[5;6]
\end{center}

\item Solve the system $\textbf{A}\vec{u} = \vec{b}$ by defining $\vec{u} = \textbf{A}^{-1} \, \vec{b}$
\begin{center}
\tt >> u=A$\backslash$b
\qquad { \rm or } \qquad 
>> u=inv(A)*b
\end{center}

\hfil\\

\item To plot a 3D plot like this, define a matrix for the solutions and write
\begin{center}
\tt >> surf(p)
\end{center}

To use the typical colouring for the heat equation, type 
\begin{center}
\tt >> colormap(cool)
\end{center}

\end{itemize}


\vfill

\emph{Further Investigation. } 
\begin{enumerate}[label=\emph{\arabic*.}]
\item Approximate the solution for the domain and boundary conditions
\begin{center}
\includegraphics*[width=200pt]{images/project-numericalPDE-domain2.pdf}
\end{center}

\item Formulate the procedure for a general $N$.

\item Formulate the procedure for a different $\Delta x$ and $\Delta y$.

\item This method can be adapted to what kind of domains? And what kind of boundary conditions?

\end{enumerate}

\begin{noexercises}
\end{noexercises}
