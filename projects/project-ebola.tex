\emph{Goal.} We want to model the spread of an epidemic, like the recent Ebola epidemic in Africa.

\vfill


\emph{SIR Model.} To model an infectious disease, we divide the population in 3 groups:
\begin{itemize}
\item Susceptible Individuals $S(t) = $ number of people who haven't contracted the disease;
\item Infected individuals $I(t)= $ number of people infected;
\item Removed individuals $R(t)=$ number people that either died or recovered from the disease and are now immune to it.
\end{itemize}

\vfill

\emph{Assumptions.} 

\begin{enumerate}[label=\emph{(\alph*)}] 
\item Population size $N$ is large and constant (no birth, death, or migration)
\item No latent/incubation period
\item Homogeneous population
\item Infection rate is proportional to the proportion of infected people with constant $\beta$
\item ``Recovery'' rate is constant $\gamma$ (includes rate at which people die or recover from the disease)
\item The typical time until recovery is $T = \dfrac{1}{\gamma}$
\end{enumerate}

\vfill

\emph{ODE. } From here we can obtain the SIR model:

\begin{emphbox}[]
\begin{align*}
\frac{dS}{dt} \quad & = \quad \text{proportional to $S(t)$ and the constant is the infection rate \emph{(e)}} 
	 \quad = \quad  - \beta \underbrace{\frac{I(t)}{N}}_{\substack{\text{proportion of}\\\text{infected people}}}S(t) \\
%	\\
\frac{dI}{dt} \quad & = \quad \underbrace{\beta \frac{I(t)}{N} S(t)}_{\substack{\text{people that stop being}\\ \text{susceptible become infected}}} - \underbrace{\gamma I(t)}_{\text{from \emph{(f)}}} \\
%	\\
\frac{dR}{dt} \quad & = \quad \underbrace{\gamma I(t)}_{\substack{\text{people that stop being}\\ \text{infected become ``recovered''}}}
\end{align*}
\end{emphbox}


\begin{itemize}
\item An important constant in this model is 
$$
R_0 = \frac{\beta}{\gamma} = \text{basic reproductive number}
$$

If $R_0 > 1$, then the disease is an epidemic (why?).

\end{itemize}


\newpage

\emph{Data. } Please download the spreadsheet with the data from the Ebola epidemic in Sierra Leone, Liberia, and Guinea.
%\begin{graybox}
\begin{itemize}
	\item \qrvideo{https://tinyurl.com/ebola-data}
\end{itemize}
%\end{graybox}

\emph{Populations. } The populations of these three countries are
\begin{itemize}
\item Sierra Leone: $N=6.3 \times 10^6$;
\item Liberia: $N=4.5 \times 10^6$;
\item Guinea: $N=12.1 \times 10^6$.
\end{itemize}

\vfill

\emph{Task. } 

\begin{enumerate}[label=\emph{\arabic*.}]
\item Use the total population $N$ to write $R(t)$ in terms of the other functions $S(t)$ and $I(t)$.

\item Find a function that approximates the data.

\item Find an estimate for the constant $\beta$ for Ebola.

\item Find an estimate for the constant $\gamma$.

\item Use function from \emph{1.} to simplify the system of 2 ODEs to a first-order ODE.

\item Solve it.

\item What was the number of recovered people?

\item Analyze the results.
%\begin{itemize}
%\item Compare your solution with a solution of the original system with the same constants
%\item \qrvideo{http://www.public.asu.edu/~hnesse/classes/sir.html}
%
%\item Conclusions about the disease
%
%\end{itemize}

\end{enumerate}




\vfill


\emph{Further Investigation. } 
\begin{enumerate}[label=\emph{\arabic*.}]
\item Study the CoViD-19 pandemic using the SEIR model.
\item Find out about linearizing the system of 2 ODEs and solving it.
\end{enumerate}

\begin{noexercises}
\end{noexercises}

