A large national park in South Africa is home to around 11\,000 elephants. 
Wildlife management policies establish that a healthy environment for this herd will keep it at 11\,000 heads. 
Each year the park officials assess the elephant population. 
During the past 20 years, part of the herd has had to be removed to keep the population as close to the 11\,000 number as possible. 
``Withdrawing'' part of the herd means either slaughtering the animals or moving them. 
Until now, 600 to 800 animals were moved each year.

Recently, environmental and animal welfare groups have protested against killing elephants. 
In addition, it is no longer possible to move even a small population of elephants each year. 
Fortunately, a contraceptive dart has been developed, which prevents a mature elephant from conceiving for a period of two years.

Here is some information about the elephants in the park:
\begin{emphbox}[]
\begin{itemize}
	\item There is very little migration of elephants.
	\item The gender ratio is close to 1:1 and control measures should be chosen to maintain this ratio.
	\item The gender ratio of baby elephants is also 1:1. Twins are born in 1.35\% of pregnancies.
	\item The elephant's maturity period begins between 10 and 12 years and continues until the age of 60. They produce an average of one baby elephant every 3.5 years. Gestation lasts approximately 22 months.
	\item Elephants who have received a contraceptive dart are in oestrus once a month, but cannot conceive. An elephant who has not received a dart usually has only one seduction and mating period per 3.5 years; the monthly heat period can therefore cause additional stress.
	\item An elephant can receive a dart each year without negative effects. A mature elephant will not be able to conceive during a period of 2 years from the moment when she received the dart.
	\item About 70\% to 80\% of baby elephants reach the age of one year. Afterwards, the survival rate is uniform up to the age of 60 and very high (above 95\%). Few elephants are over 70 years of age.
	\item There is no hunting or poaching in the park.
\end{itemize}
\end{emphbox}

The park management team has \emph{data} $\bigg($\url{https://uoft.me/elephants} \qrcode[height=0.35in]{https://uoft.me/elephants} $\ \bigg)$ on the approximate age and gender of elephants relocated outside the park during the past two years. 
Unfortunately, no data is available on the elephants that have been slaughtered or that constitute the herd present in the park. 

\emph{Task.} The overall goal of the project is to investigate the feasibility of a herd control program using contraceptive darts. 
More specifically:

\begin{enumerate}[label=\emph{\arabic*.}]
	\item Develop a model and use it to describe the probable survival rate of elephants from 2 to 60 years old. Also speculate on the demographic curve of the herd present in the park.

	\item Estimate how many elephants need a dart each year to keep the herd at around 11\,000 heads. Show how the uncertainties in the data at your disposal can influence your estimate. Comment on possible changes in the demographic curve and how these changes could have an impact on the tourism industry over a 30 to 60 year horizon.

	\item Some opponents of the contraceptive dart technique claim that, if there was a sudden disappearance of a large part of the herd (for example following an illness or poaching out of control), the ability of the population to return to its optimal level would be seriously compromised, even if the dart program was quickly discontinued. Investigate this statement and respond to this concern.

\end{enumerate}



%\vfill
%
%\emph{Further Investigation.}
%\begin{enumerate}[label=\emph{\arabic*.}]
%\item Can you think of other forms for $H(P,t)$? 
%
%\item If, instead of a logistic model, you include an extinction threshold as well, what can you say about the model for constant effort fishing? for constant rate fishing? Is it a useful addition to the model? Have fun with it!
%\end{enumerate}



\vfill

\begin{graybox}
\hfill ``\elephantstitle'' is a collaboration with Yvan Saint-Aubin.	
\end{graybox}


\begin{noexercises}
\end{noexercises}
