\begin{center}
	\includegraphics*[width=250pt]{images/project-dragon.jpg}
	
	One of the dragons from the Games of Thrones series. (\copyright HBO)
\end{center}


Scientists are sometimes asked to advise producers of fictional films: about the feasibility of a pileup, description of a sinking ship, problems to be resolved during the colonization of a planet, etc. Another example is the use of dragons in the Game of Thrones television series and in films from the books on the young wizard Harry Potter\footnote{J.K. Rowling, Harry Potter, seven books published by Bloomsbury (in England) from 1997 to 2007.}: how do you realistically and plausibly describe fictional animals? Sometimes science comes after the film is made. L.M. Krauss's The Physics of Star Trek (Harper (1995)) is one such example. This latest project invites you to advise the producers of the Game of Thrones series after the fact.

In this series from the novels ``A Song of Ice and Fire''\footnote{J.R.R. Martin, A song of Ice and Fire , five books published by Bantam (in Canada and USA) from 1996 to 2011.}, three dragons are bred by Daenerys Targaryen. At hatching, dragons are small, around 10kg, and after a year they weigh between 30 and 40kg. Their growth lasts a lifetime and depends on their environment and the food available to them.

Suppose that these three dragons live today and that the data in the previous paragraph is correct. You will need to make additional assumptions, such as the ability to fly long distances, spit fire, and survive major injuries. You will have to relate these hypotheses (height and weight, diet, etc.) to the questions you are studying. 

\emph{Task.} Your team should analyze the ecology of dragons: their behaviour, their habits, their diet and the interaction with their environment. 

A non-exhaustive list of questions includes:
\begin{enumerate}[label=\emph{\arabic*.}]
	\item What are the dimensions, physical characteristics and longevity of an adult dragon?
	\item What is the ecological impact and the needs of a dragon?
	\item What is the energy expenditure of a dragon (daily, weekly?) and what are their calorie intake needs?
	\item How much land is required to maintain these three dragons alive?
	\item What is the size of the team (of humans) required to care for these dragons?
	\item How much energy is released when a dragon spits? How long does a flame jet last? How many jets can it emit per day?
	\item Can dragons migrate? And if so, how far and for how long can they migrate?
\end{enumerate}

Other questions may be added depending on your modelling process. If necessary, you can comment on the impact of the climate (arid, temperate, northern regions) on the questions you have chosen. Some of the issues considered will be linked together; you should highlight these links and ensure that the models developed are not contradictory.

Finally, even if your analysis is focused on fictional animals, you are invited to describe and discuss (real) situations, other than the ecology of fictional dragons, where your modelling efforts could be useful.

\emph{One last word.} The animals are fictional, but the model must be realistic and plausible!



%\vfill
%
%\emph{Further Investigation.}
%\begin{enumerate}[label=\emph{\arabic*.}]
%\item Can you think of other forms for $H(P,t)$? 
%
%\item If, instead of a logistic model, you include an extinction threshold as well, what can you say about the model for constant effort fishing? for constant rate fishing? Is it a useful addition to the model? Have fun with it!
%\end{enumerate}



\vfill

\begin{graybox}
\hfill ``\dragonstitle'' is a collaboration with Yvan Saint-Aubin.	
\end{graybox}


\begin{noexercises}
\end{noexercises}
