

\begin{center}
	\includegraphics*[width=300pt]{images/project-maximus-screenshot.png}
\end{center}


\emph{Maximus, the Roman general turned slave turned gladiator} is seeking \emph{vengeance!} (In this life or the next).


He has managed to make his way to the Roman Forum, and now all that stands between him and nefarious Emperor Commodus is the \emph{Pr{\ae}torian Guard}.

Maximus must make his way across the Forum, defeating all Pr{\ae}torians in his way, if he is to achieve his goal in this lifetime.

\begin{graybox}
\emph{Conditions. }
\begin{itemize}
	\item The forum can be represented by the horizontal $x$-axis, in the interval $[1,1823]$;
	\item Maximus starts at the left side of the forum;
	\item Maximus moves at a constant rate across the forum - we call this $M_s$ (Maximus' speed);
	\item The arrival of Pr{\ae}torian guards follows a Poisson process with rate $P_r$ (Pr{\ae}torian arrival rate);
	\item Pr{\ae}torians appear at the right side of the forum, and move at a constant rate toward Maximus. This rate is called $P_s$ (Pr{\ae}torian speed);
	\item Both Maximus and the Pr{\ae}torian(s) will move toward each other until blocked (by Maximus in the case of the front Pr{\ae}torian, or by another Pr{\ae}torian if there's a lineup waiting to turn Maximus into pulp);
	\item The maximum number of Pr{\ae}torians is limited by the size of the forum.
\end{itemize}
\end{graybox}

\begin{redbox}
\emph{When Maximus meets a Pr{\ae}torian.}	
Fisticuffs ensue!

\begin{itemize}
\item Maximus and the Pr{\ae}torians both start with a fixed health value, let's call this $H$ (initial health).
\item Maximus strikes on the Pr{\ae}torians are another Poisson process, with rate $M_{hr}$ (Maximus hit-rate) Pr{\ae}torians strike Maximus also according to a Poisson process, with rate $P_{hr}$ (Pr{\ae}torian hit-rate)
\item When Maximus hits a Pr{\ae}torian, he causes damage in the amount $d=M_{hv} \cdot \frac{M_h}{H}$.

	That is, the damage is given by a parameter $M_{hv}$ (Maximus hit-value) multiplied by $M_h$ (Maximus' current health) divided by the initial (maximum) health. It's a tough world and Maximus' strikes get weaker the weaker he gets!
\item The same is true for the Pr{\ae}torians (with their own hit-value and corresponding health), but there's an additional factor here - it's Maximus we're talking about! He's trained in the arena and defeated the most skilled gladiators! Therefore, he is able to dodge many of the Pr{\ae}torians' strikes!

	Pr{\ae}torians land a blow on Maximus only with probability $1/M_{tf}$ where $M_{tf}$ is Maximus' training factor. The better trained Maximus is, the harder it is for the Pr{\ae}torians to actually strike him.
\item If at any point the health of the Pr{\ae}torian goes below zero, the soldier dies and Maximus can continue his progress along the Forum (at least until he meets the next Pr{\ae}torian).
\end{itemize}
\end{redbox}

If at any point Maximus' health goes below zero, it's over and he'll have to get his Vengeance on the next life.

Happily for Maximus, as long as he's just walking (not fighting), his health regenerates at a rate given by $M_{rr}$ (Maximus' health recovery rate). How much he'll recover depends on how far he can get before meeting his next foe! \\




\newpage
\begin{graybox}
To help you visualize the task, here is some code you can run on Octave or Matlab.
\begin{itemize}
	\item \qrvideo{https://uoft.me/maximus}
\end{itemize}
\end{graybox}

\hfill

\emph{Task. } 
\begin{enumerate}[label=\emph{\arabic*.}]
	\item Using the code supplied above, find some values for which sometimes Maximus achieves vengeance in this life, and sometimes only in the next.
	
	\emph{Hint.} To speed the simulation, you might want to disable the graphics.
	\item Model the fight with deterministic processes instead of the Poisson processes.
	\item Fix all values and find the \emph{critical} training value for Maximus with the deterministic processes.
	\item Model the actual fight.
\end{enumerate}











\vfill

Will Maximus make his way across the Forum and get his vengeance on Emperor Commodus? And more importantly$\ldots$

\begin{center}
\emph{ARE YOU NOT ENTERTAINED?????!!!!!!	}
\end{center}

%\vfill
%
%\emph{Further Investigation:}
%\begin{enumerate}[label=\emph{\arabic*.}]
%%	\item Assume that the lungs only absorb a fraction of the air in contact with its surface. Combine the two tasks.
%	\item Investigate how the what is known about human lungs. Compare how the branching of real human lungs differs from this model. Refine the estimate.
%	\item Investigate how the what is known about human lungs. Compare how the gas exchange of real human lungs differs from this model. Refine the model.
%\end{enumerate}
%
%\vspace{1cm}


\vfill

\begin{graybox}
\hfill ``\maximustitle'' is a collaboration with Francisco Estrada.
\end{graybox}

\begin{noexercises}
\end{noexercises}
