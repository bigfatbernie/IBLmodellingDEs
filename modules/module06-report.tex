\begin{siam}
In this module you will learn
\begin{itemize}
	\item how to put all that you have done together into a well structured report
\end{itemize}

\hfill \\



This is the final stage of the modelling project.

By now, you have started with a mathematically defined problem, with some assumptions, and you have created a mind map to help you navigate the problem.
You have also constructed a model and assessed it to make sure it is sound.

All that we have left is to put all this work together into the form of a report.



The report should consist of two parts:

\begin{enumerate}
	\item \textbf{Summary. } Should be at most one page long, and contain a statement of the problem, a brief description of the methods chose to solve it, and some final results and a conclusion. In this part of the report, you should keep mathematical symbols to a minimum, so the reader gets an idea of what to expect in the remainder of the report without getting bogged down in unfamiliar mathematics.

	\item \textbf{In-depth report. } This is where the details go in. It should start with an introduction to the problem assuming that the reader is not aware of it. It should then be structured according to the steps we did before:
	\begin{itemize}
		\item Optionally, you can include a mind map with a description of how it guided the whole process
		\item Assumptions and variables in the model
		\item The model described in detail
		\item The solution process
		\item The assessment of the model
		\item A conclusion, with a description of the results
	\end{itemize}
\end{enumerate}





\begin{example}
As an example of an excellent report, please read the report from the winning team of the 2019 $M_3C$ challenge:
\begin{itemize}
	\item \qrvideo{https://uoft.me/modelling-app-report}
	\item Read the summary and chapters 1, 2, 5.
\end{itemize}
\end{example}

\end{siam}

\begin{siam2019}


\begin{definition}[Report checklist]

\begin{tabular}{|p{75pt}|p{200pt}|p{125pt}|}
\hline
\textbf{Component}
	& \textbf{Questions about your model and how you made it}
	& \textbf{Useful vocabulary} \\ \hline
\multirow{2}{75pt}[-10pt]{\textbf{Defining the problem}}
	& What is/are the big problem/s that you have been asked to solve?
		& open-ended problem \\ \cline{2-3}
	& What is the specific problem your model is going to solve?
		& specific, focus \\ \hline
\multirow{4}{75pt}[-25pt]{\textbf{Making assumptions}}
	& What ideas did you think about that you decided not to try? 
		& eliminate, prioritize \\ \cline{2-3}
	& What have you assumed in order to solve the problem? Why did you make these choices? 
		& assumption, constraints \\ \cline{2-3} %\hline
%\multirow{2}{75pt}{\textbf{Defining variables}}
	& What quantities are important? Which ones change and which ones stay the same? 
		& variable  \\ \cline{2-3}
	& Where did you find the numbers that you used in your model? 
		& resources, citations \\ \hline
\multirow{2}{75pt}[-15pt]{\textbf{Getting a solution}}
	& What pictures, diagrams or graphs might help people understand your information, model, and results? 
		& diagram, graph, labels  \\ \cline{2-3}
	& What mathematical ideas did you use to describe the situation and solve your problem? 
		& situation  \\ \hline
\multirow{4}{75pt}[-30pt]{\textbf{Model assessment}}
	& How do you know that your calculations are correct? Did you remember to use units (like dollars or metres?) 
		& calculation, unit \\ \cline{2-3}
	& When does your model work? When do you need to be careful because it might not? 
		& limitations  \\ \cline{2-3}
	& How do you know you have a good/useful model? Why does your model make sense? 
		& testing, validation \\ \cline{2-3}
	& If you were going to make your model better, what would you do? 
		& improvement, iteration \\ \hline
\multirow{3}{75pt}[-20pt]{\textbf{Reporting results}}
	& Explain your mathematical model in words and math. 
		& clarity, concision \\ \cline{2-3}
	& What are the strengths and weaknesses of your model?
		& strengths, weaknesses \\ \cline{2-3}
	& What are the 5 most important things for your audience/client to understand about your model and/or solution? 
		& client, audience \\ \hline
\end{tabular}

\hfill \\

This checklist is adapted from

\begin{graybox}
\begin{minipage}{.75\textwidth}
\begin{verbatim}
	GAIMME: Guidelines for Assessment and Instruction in Mathematical
	Modeling Education, Second Edition, Sol Garfunkel and Michelle
	Montgomery, editors, COMAP and SIAM, Philadelphia (2019)
\end{verbatim}
\begin{center}
\url{https://uoft.me/gaimme}
\end{center}
\end{minipage}
\hfill
\begin{minipage}{.20\textwidth}
	\hfill\qrcode{https://uoft.me/gaimme}	
\end{minipage}
\end{graybox}

\end{definition}

	
\end{siam2019}



