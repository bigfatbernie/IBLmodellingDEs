\begin{siam}
In this module you will learn
\begin{itemize}
	\item how to put all that you have done together into a well structured report
\end{itemize}

\hfill \\



This is the final stage of the modelling project.

By now, you have started with a mathematically defined problem, with some assumptions, and you have created a mind map to help you navigate the problem.
You have also constructed a model and assessed it to make sure it is sound.

All that we have left is to put all this work together into the form of a report.



The report should consist of two parts:

\begin{enumerate}
	\item \textbf{Summary. } Should be at most one page long, and contain a statement of the problem, a brief description of the methods chose to solve it, and some final results and a conclusion. In this part of the report, you should keep mathematical symbols to a minimum, so the reader gets an idea of what to expect in the remainder of the report without getting bogged down in unfamiliar mathematics.

	\item \textbf{In-depth report. } This is where the details go in. It should start with an introduction to the problem assuming that the reader is not aware of it. It should then be structured according to the steps we did before:
	\begin{itemize}
		\item Optionally, you can include a mind map with a description of how it guided the whole process
		\item Assumptions and variables in the model
		\item The model described in detail
		\item The solution process
		\item The assessment of the model
		\item A conclusion, with a description of the results
	\end{itemize}
\end{enumerate}



\begin{example}
You can find the report from the winning team of the 2019 $M_3C$ challenge in appendix \ref{2019M3C}.
	
\end{example}

\end{siam}


