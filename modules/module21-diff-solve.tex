In this module you will learn
\begin{itemize}
	\item how to solve some types of difference equations
\end{itemize}

\hfill \\


%Let us start with some easier difference equations to gain some intuition on how to solve them.


%\subsection{First-order linear difference equations with constant coefficients}
%
%Let us start with a simple example.
%
%\begin{example}


Let us start with a technique that is very simple and useful, although because it is so simple, it requires some ingenuity to pull off in some cases. \\

\submodule{Expanding to find a pattern} 

We'll start with an example.

\begin{example}
Consider the initial-value problem
$$
\begin{cases}
u_{k+1} = \frac32 u_k & \text{ for } k \geq 0 \\
u_0 = 5	
\end{cases}
$$

Then we can start calculating:
\begin{itemize}
	\item $u_1 = \frac32 u_0 = 7.5$
	\item $u_2 = \frac32 u_1 = 11.25$
	\item $u_3 = \frac32 u_2 = 16.875$
	\item $u_4 = \frac32 u_3 = 25.3125$
	\item $u_5 = \frac32 u_4 = 37.96875$
	\item $\vdots$
\end{itemize}

As you can notice, it's not particularly easy to find a pattern in these numbers. 

The problem is that we \emph{over-simplified}. The trick with this technique is to simplify without over-simplifying.

Let's calculate again:
\begin{itemize}
	\item $u_1 = \frac32 u_0 = \frac32 \cdot 5$
	\item $u_2 = \frac32 u_1 = \frac32 \cdot \frac32 \cdot 5 = \left(\frac32\right)^2 \cdot 5$
	\item $u_3 = \frac32 u_2 = \frac32 \cdot \left(\frac32\right)^2 \cdot 5 = \left(\frac32\right)^3 \cdot 5$
	\item $u_4 = \frac32 u_3 = \frac32 \cdot \left(\frac32\right)^3 \cdot 5 = \left(\frac32\right)^4 \cdot 5$
	\item $u_5 = \frac32 u_4 = \frac32 \cdot \left(\frac32\right)^4 \cdot 5 = \left(\frac32\right)^5 \cdot 5$
	\item $\vdots$
\end{itemize}

Now the pattern should be clear:
$$
u_k = \left(\frac32\right)^k \cdot 5.
$$

To show that this is indeed the solution, we need to use Mathematical Induction (see appendix \ref{induction}) to prove it. 
\end{example}

The main idea of this technique is to calculate the the terms of the fraction one by one in terms of the initial data.

This is a technique that requires practice, as it is often difficult to judge which parts to simplify and which parts to expand to make sure the pattern emerges clearly. \\

\begin{video}
\begin{itemize}
	\item \qrvideo{https://youtu.be/0OcUAjOXmFc}
\end{itemize}	
\end{video}





\submodule{Educated Guessing}

This is the technique we used several times in the book already. We used it with systems of differential equations and with second-order differential equations.

Observe that in the last example, the solution was an exponential, as was the case with differential equations.

\begin{example}
Consider the Fibonacci sequence:
$$
\begin{cases}
f_{k+1} = f_k + f_{k-1} \\
f_0 = 0 \\
f_1 = 1	
\end{cases}
$$

We want to find a formula for $f_k$. To do that, let us assume that the sequence is an exponential. So we can assume that
$$
f_k = r^k,
$$
for some value of $r$.

Let us now use this form of $f_k$ into the difference equation to obtain:
$$
r_{k+1} = r_k + r_{k-1},
$$
which can be simplified by dividing by $r^{k-1}$:
$$
r^2 = r + 1 \quad \Leftrightarrow \quad r^2 - r - 1 = 0.
$$

This is a quadratic equation that we can solve:
$$
r_{\pm} = \frac{1 \pm \sqrt{1 + 4}}{2} = \frac{1 \pm \sqrt{5}}{2}.
$$

So we have two values of $r$ that seem to work. 

That is similar to what we had when solving second-order ODEs (and this is a second-order difference equation).
In that case, the solution turned out to be a linear combination of the two solutions found:
$$
f_k 
	\quad = \quad  c_1 r_-^k + c_2 r_+^k
	\quad = \quad  c_1 \left(\frac{1 - \sqrt{5}}{2}\right)^k + c_2 \left(\frac{1 + \sqrt{5}}{2}\right)^k.
$$

Now we need to find $c_1$ and $c_2$ using the initial data:
\begin{align*}
0 & = c_1 + c_2
	\tag{$k=0$}	 \\
1 & = \quad  c_1 \frac{1 - \sqrt{5}}{2} + c_2 \frac{1 + \sqrt{5}}{2}
	\tag{$k=1$}
\end{align*}
This yields:
\begin{align*}
c_1 & = -\frac{1}{\sqrt{5}}\\
c_2 & = \frac{1}{\sqrt{5}}	
\end{align*}

So the formula we obtain is
$$
f_k =  \frac{1}{\sqrt{5}} \left[\left(\frac{1 + \sqrt{5}}{2}\right)^k - \left(\frac{1 - \sqrt{5}}{2}\right)^k \right].
$$
\end{example}


The idea of this technique is to assume that the solution is an exponential of the form $r^k$ and find the values for $r$ that solve the particular difference equation. The general solution will be a linear combination of these solutions.


\begin{video}
\begin{itemize}
	\item \qrvideo{https://youtu.be/A5tBvxDM9V4}
\end{itemize}	
\end{video}


