\newpage 

\begin{exercises}

	\begin{problist}
	\prob Consider the model for student learning:
	\begin{itemize}
		\item $\vec{x} = \begin{bmatrix} x_1 \\ x_2 \end{bmatrix}$
		\item $x_1=$ student confidence in his/her own abilities ($x_1 \in [0,1]$)
		\item $x_2=$ student knowledge measured in IQ past 100
		\item $\vec{x}'(t) =
		\begin{bmatrix}
			a & b \\
			c & - d 
		\end{bmatrix}
		\vec{x}(t)
		+ \begin{bmatrix}
 			-e \\ 0
 		\end{bmatrix}$
		\item Constants $a,b,c,d,e>0$.
	\end{itemize}
	\begin{enumerate}
		\item What is the equilibrium solution $\vec{x}_e$? 
		\item If tests are harder, then $d$ is larger. How does that affect the equilibrium confidence and knowledge of students?
		\item Is the equilibrium solution stable?
		\item Assume $a=1, b=c=2, d=3,e=0$. As $t \to +\infty$, what are the possible outcomes for $\vec{x}(t)$ Explain the meaning for the students.
		\item Assume $a=1, b=c=2, d=3,e=0$. Some solutions satisfy $\displaystyle \lim_{t \to +\infty} \begin{bmatrix}c(t) \\ k(t) \end{bmatrix} = \begin{bmatrix} + \infty \\ + \infty \end{bmatrix}$.

			Show on a graph which initial conditions $\vec{x}(0) = \begin{bmatrix}c(0) \\ k(0) \end{bmatrix}$ guarantee this limit?

		\item If the tests become harder, i.e., $d$ increases, then is that good or bad for students?.
	\end{enumerate}
	
	\prob Consider the model for a tree:
	\begin{itemize}
		\item $\vec{x}(t) = \begin{bmatrix} \ell(t) \\ h(t) \end{bmatrix}$
		\item $\ell(t)=$ area of leafs on the tree
		\item $h(t)=$ height of the tree
		\item $\vec{x}'(t) =
		\begin{bmatrix}
			a & -b \\
			c & -d 
		\end{bmatrix}
		\vec{x}(t)$
		\item Constants $a,b,c,d>0$.
	\end{itemize}
	\begin{enumerate}
		\item Is it possible to have the tree growing taller and taller forever while the leaf area remains bounded?
		\item What would happen to the tree if the area of leafs is proportional to the height squared (not square root)?
		\item If $ad=bc$, explain what happens to the tree as $t\to \infty$.
	\end{enumerate}
	
	
	\prob Consider the model of a car:
	\begin{itemize}
		\item $\vec{c}(t) = \begin{bmatrix} v(t) \\ f(t) \end{bmatrix}$
		\item $v(t)=$ speed of the car
		\item $f(t)=$ amount of fuel in the car's tank
		\item $\vec{c}'(t) =
		\begin{bmatrix}
			-2 & 1 \\
			-2 & 0 
		\end{bmatrix}
		\vec{c}(t)
		+ \begin{bmatrix}
 			0 \\ -1	
		\end{bmatrix}$
	\end{itemize}
	\begin{enumerate}
		\item What is the equilibrium solution $\vec{c}_{\rm eq}$? What is the meaning of your result?
		\item If the car runs out of fuel at 300 m/s, then describe what happens to the car. 
		\item Describe what happens to the car when it starts at rest with a full tank of $300$ L.
		\item If the car attains its maximum velocity when there are still $300$ L of fuel left, what was the car's maximum velocity?


	\end{enumerate}


	\prob Consider the model for crying babies:
	\begin{itemize}
		\item $\vec{c}(t) = \begin{bmatrix} a(t) \\ b(t) \end{bmatrix}$
		\item $a(t)=$ volume of baby A's cries in dB
		\item $b(t)=$ volume of baby B's cries in dB
		\item $\vec{c}'(t) =
		\begin{bmatrix}
			-\alpha & \beta \\
			\beta & -\alpha 
		\end{bmatrix}
		\vec{c}(t)$
		\item Constants $\alpha,\beta>0$.
	\end{itemize}

	The constants $\alpha$ and $\beta$ are $1$ and $2$. Does it make a difference which is 1 and which is 2?


	
		
	\end{problist}
\end{exercises}