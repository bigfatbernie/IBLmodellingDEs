\newpage

\begin{exercises}

	\begin{problist}
	
	\prob Consider a mountain with shape $y=f(x)$ a hiker who is climbing down the mountain with horizontal position $x(t)$. She starts at a peak of the mountain at $x_0=0$. As she climbs down the mountain, she notices that from her point-of-view, the rate of change of the slope of the mountain is decreasing linearly with time.
		The hiker also notices that her horizontal speed is constant.
	
		Model the hiker's position and the shape of the mountain.
	
	\prob 
	
	\begin{center}
		\includegraphics*[width=150pt]{images/module20-catenary.pdf}
	\end{center}

	\prob Model a ping pong ball travelling through the air.
	
	\begin{center}
		\includegraphics*[height=100pt]{images/module20-hotairballoon.pdf}
	\end{center}
	
	\prob Model an old TV floating or sinking in the ocean.
	
	\prob Model a container floating in the ocean with a leak that allows water to get inside.
	
	\prob Model a hot air balloon flying through the air.
	
	\prob Model the shape of a rope hanging between two poles.

	\begin{center}
		\includegraphics*[width=150pt]{images/module20-suspensionbridge.pdf}
	\end{center}

	\prob Model the shape of the cables of a suspension bridge.
	
	\prob Imagine a cylinder floating vertically partially submerged in a lake. Model the position of its top.
	
	\prob Model a ball rolling down a hill.
	
	\prob Model an electric circuit with a resistor, and inductor, and a capacitor in series.
	
	\prob Start with the Law of Conservation of Energy and assume a conservative force. Then show that you obtain Newton's Second Law of motion.
	
	\prob Model a self-balancing device, like a segway or a onewheel.
	
	\prob Imagine a device that can turn the handlebar of a bicycle. Model how it should act to keep the bicycle upright.
	
	
	\end{problist}
\end{exercises}
