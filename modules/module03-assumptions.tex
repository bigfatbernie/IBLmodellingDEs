\begin{siam}
	
In this module you will learn
\begin{itemize}
	\item that we need to make assumptions to be able to create a model
	\item how to strike a balance between accuracy and solvability
\end{itemize}

\hfill \\




Real problems are complex, so when modelling a real problem mathematically, we must make some assumptions. 

The assumptions that we make will affect the problem we are solving and its difficulty, so we need to strike a balance between:
\begin{itemize}
\item accuracy -- the fewer assumption the better, and
\item solvability -- the more assumptions the better.
\end{itemize}

\begin{annotation}
	\begin{goals}
		When building a mind map, keep track of the assumptions necessary for each step.
	\end{goals}
\end{annotation}

Many assumptions follow naturally when building a mind map. \\


\begin{annotation}
	\begin{goals}
		Remember to justify all your assumptions.
	\end{goals}
\end{annotation}

When figuring which assumptions to make, keep in mind the key-factors of the problem and find data when available (usually online). 
If not available, measure data when possible, and if it's not possible, make a reasonable assumption on what the data might look like.

Another thing to keep in mind are \emph{time constraints}. Whether in a class, test, or working in a project, there will be deadlines. Your assumptions should take time constraints into consideration. 



\begin{example}

Let us revisit the example of the previous module about the ``best'' recycling centre.

For this example, imagine that the team decided on focusing their attention on the least cost to the city through building drop off centres. \\

For this, we need to find out how many people would make use of the drop-off centres (termed ``likelihood of participation'').  \\

The two extremes would be to assume that the 100\% of the people near a recycling centre would use it or that none would use it. Neither of these seems like a reasonable assumption, so what would be a better assumption?  \\

Maybe the best idea is do some investigation and see if there has been any successful research on participation rates in drop-off centres.  \\

The team found a study that had been done in Ohio that estimated that about 15\% of households participated in drop-off centre recycling, and made an assumption that this rate would hold in every city across the U.S.. \\

One might ask if it is safe to assume that across the U.S. 15\% of households will participate in drop-off centre recycling if it is available. Is it true that residents of Arizona will behave the same way residents of Ohio do? Certainly some cities would garner a participation rate much higher than 15\%, while other cities would have a significantly lower participation rate. In fact, what are the chances that any city would actually have a participation rate of exactly 15\%?

In some sense, one might say that assigning one participation rate to every city across the U.S. is a ridiculous assumption. \\

In response to that line of thinking, remember two things:
\begin{itemize}
	\item First, remember that \emph{one must make assumptions in order to make a model}. It is not practical or feasible to poll every citizen of every city to determine who will bring recyclables to a drop off centre. If we had to rely on data with that level of certainty at every juncture of the modelling process, we would never get any work done.

		It's practical and important to make reasonable assumptions when we cannot find data.

	\item Second, you are developing a model that is intended to help one understand some complex behaviour or assist in making a complex decision. It is not likely to predict the exact outcome of a situation, only to help provide insight and predict likely outcomes. When you \emph{provide a list of your assumptions}, you've done your part to inform anyone who might use your model. They can decide whether they think your assumption is or is not appropriate to model the behaviour they are interested in predicting.

\end{itemize} 


   
	
\end{example}
\end{siam}




%\hfill \\

%	\section*{Step D. Parameters or Variables?}\label{D-parvsvar}
%	\addcontentsline{toc}{subsection}{Step D. Parameters or Variables?}
%	
%	
%	
%	When you have defined the problem you want to solve and you have made your (initial) assumptions, it is then time to define some details of the problem.
%	
%	
%	
%	
%	With the problem statement clearly defined and an initial set of assumptions made (a list that will likely get longer), you are ready to start to define the details of your model. Now is the time to pause to ask what
%	is important that you can measure. Identifying these notions as variables, with units and some sense of their range, is key to building the model.
%	The purpose of a model is to predict or quantify something of interest. We refer to these predictions
%	as the outputs of the model. Another term we use
%	for outputs is dependent variables. We will also have independent variables, or inputs to the model. Some quantities in a model might be held constant, in which case they are referred to as model parameters. Let's look at a few simple examples that will help you distinguish between these concepts. We'll also see how they depend on your viewpoint and the problem statement.
%	
%	
%	
%	%There is a clear difference between \emph{variables} and \emph{parameters}. 
%	
%	\begin{definition}[Variables and Parameters]
%	
%	
%	A \emph{variable} represents a model state, and may change during simulation.
%	
%	A \emph{parameter} is commonly used to describe objects statically. A \emph{parameter} is normally a constant in a single simulation, and is changed only when you need to adjust your model behaviour. 
%	\end{definition}
%	
%	%Use a variable instead of a parameter if you need to model some data unit continuously changing over time. Use a parameter instead of a variable if you just need to model some parameter of an object changed only at particular moments of time.
%	
%	
%	
%	\begin{annotation}
%		\begin{goals}
%		\qrcode{https://en.wikipedia.org/wiki/Parameter\#Mathematical\_models}	
%		\end{goals}
%	\end{annotation}
%	\begin{note}{(from Wikipedia)}
%	The quantities appearing in the equations we classify into variables and parameters. The distinction between these is not always clear cut, and it frequently depends on the context in which the variables appear. 
%	
%	Usually a model is designed to explain the relationships that exist among quantities which can be measured independently in an experiment; these are the \emph{variables} of the model. 
%	
%	To formulate these relationships, however, one frequently introduces ``constants'' which stand for inherent properties of nature (or of the materials and equipment used in a given experiment). These are the \emph{parameters}.	
%	\end{note}






%
%
%
%The choice of question in the previous module should determine the \emph{dependent} variable.

%
%The \emph{parameters} are the independent variables in the problem, e.g. the speed of the elevators. The final answer will depend on the parameters in the problem. 
%
%We can estimate the parameters, and sometimes even change them. \\
%
%The \emph{variables} are dependent. This meant that if we change the parameters, the variables will change automatically. 
