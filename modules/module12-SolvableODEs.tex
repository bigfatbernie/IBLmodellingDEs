In this module you will learn
\begin{itemize}
	\item to identify specific types of differential equations that can be solved rigorously
	\item how to solve these types of differential equations
\end{itemize}

\hfill \\

We just learned how to model a situation and end up with a differential equation.
We will now focus on solving differential equations. 

There are a few different techniques that depend on the differential equation. 

\hfill

\begin{submodule}{Separable Differential Equations}

\begin{definition}[Separable ODE]
A differential equation is called \emph{separable} if it has the form
$$
g(y) y'(t) = h(t),
$$
that is if we can separate all the $y$'s into the left-hand side and all the $t$'s into the right-hand side of the equation. Observe that the $y$'s on the left hand side must all be multiplied by $y'(t)$. 
\end{definition}


\subparagraph{\emph{Method of solution. }} The idea to solve this type of DEs is simple:
\begin{enumerate}[label={\bf \arabic*. } ]
\item Integrate both sides with respect to $t$:
$$
\int g(y) y'(t) \, dt = \int h(t) \, dt
$$

\item Change variables on the left-hand side to $u = y(t)$, so $du = y'(t) dt$ and we get
$$
\int g(u) \, du = \int h(t) \, dt.
$$

\item Solve both integrals and we obtain a solution, usually in implicit form:
$$
G(u) = H(t) + C.
$$

\item To finish, recall that $u=y(t)$, so we obtain
$$
G\big(y(t)\big) = H(t) + C.
$$
\end{enumerate}

\begin{important}
	Observe that the solution is given in implicit form. In general, when using this technique, the solution $y(t)$ will be given in implicit form, so there is still some work ahead to find an explicit formula for $y(t)$.	
\end{important}

\begin{graybox}
A simplified method of solution is the following:
\begin{enumerate}
	\item Start with the differential equation
	$$ g(y) \frac{dy}{dt} = h(t) .$$
	\item ``Move the dt to the other side'':
	$$ g(y) dy = h(t) dt .$$
	\item Integrate both sides
	$$ \int g(y) dy = \int h(t) dt.$$
\end{enumerate}	
\end{graybox}



\def\arcsinh{{\rm arcsinh\ }}
\begin{example}
The shape $y(x)$ of a free falling chain under its own weight, called a catenary, satisfies the differential equation:
$$
y''(x) = \frac1a \sqrt{1+\big(y'(x)\big)^2}.
$$

It doesn't seem to be a \emph{separable equation}, but if we can define $z(x) = y'(x)$, which satisfies
$$
z'(x) = \frac1a \sqrt{1+\big(z(x)\big)^2}
\qquad \Leftrightarrow \qquad \frac{1}{\sqrt{1+\big(z(x)\big)^2}} z'(x) = \frac1a.
$$

This is now in the form of a \emph{separable ODE}. \\

We can solve it using the method described above: we need to solve
$$
\int \frac{1}{\sqrt{1+z^2}}\,dz \quad = \quad \int \frac1a \,dx \quad = \quad \frac{x}{a} + C_1
$$
The integral on the left can be solved using a hyperbolic substitution $z = \sinh u$:
$$
\int \frac{1}{\sqrt{1+z^2}}\,dz = \int 1 \, du = u = \arcsinh z.
$$
This means that the solution satisfies
$$
\arcsinh z = \frac{x}{a} + C_1 
\qquad \Leftrightarrow \qquad z = \sinh \left(\frac{x}{a} + C_1 \right).
$$

Now recall that $z(x) = y'(x)$, so we need to integrate $z(x)$ to obtain the catenary curve $y(x)$:
$$
y(x) = \int z(x) \, dx = a \cosh \left(\frac{x}{a} + C_1\right) + C_2.
$$

To find $C_1$ and $C_2$, we use the fact that $y'(0)=0$:
$$
y(x) = a \cosh \frac{x}{a}  + C_2.
$$
(the constant $C_2$ moves the curve up or down, so it doesn't change the shape).
\end{example}


\begin{video}
\begin{itemize}
	\item \href{https://youtu.be/txtFH89HwOA}{https://youtu.be/txtFH89HwOA} \hfill \qrcode{https://youtu.be/txtFH89HwOA}
	\item \href{https://youtu.be/8xG_Xg6X2MQ}{https://youtu.be/8xG\_Xg6X2MQ} \hfill \qrcode{https://youtu.be/8xG_Xg6X2MQ}
	\item \href{https://youtu.be/ZE1Agfkhr28}{https://youtu.be/ZE1Agfkhr28} \hfill \qrcode{https://youtu.be/ZE1Agfkhr28}
\end{itemize}	
\end{video}

\end{submodule}

%\hfill \\
\newpage

\begin{submodule}{First-Order Linear Differential Equations}

\begin{definition}[First-Order Linear ODE]
A differential equation is called \emph{first-order linear} if it has the form
$$
y'(t) + p(t) y(t) = f(t).
$$
\end{definition}


The idea to solve this type of DEs is to transform it into the result of a product rule. \\


\begin{example}
Consider the following \emph{first-order linear ODE}
$$
t^2 \frac{dy}{dt} + 2t y = \sin t.
$$

Observe that the left-hand side of the DE is the result of the product rule:
$$
\frac{d}{dt} \bigg[ t^2 y \bigg] = \sin t.
$$
So  we can integrate both sides with respect to $t$ to obtain
$$
t^2 y = - \cos t + C 
\quad \Leftrightarrow \quad y = -\frac{\cos t}{t^2} + \frac{C}{t^2}.
$$
\end{example}


Now let us look at another example, where the left-hand side of the ODE is not in the form of the result of a product rule, but can be transformed into one.


\begin{example}
Consider the \emph{first-order linear ODE}
\begin{equation}\tag{$\star$}\label{eq:intfact_1}
\frac{dy}{dt} + \frac12 y = \frac13 e^{\frac t3}.
\end{equation}
%which passes through the point $(0,1)$.


Again, the ``trick'' is to look at this equation and realize that the left-hand side can look like the result of the product rule.
It's not obvious that this can be done (yet!), but if we multiply the whole ODE by the function
$$
e^{\frac{t}{2}},
$$
then we obtain
$$
e^{\frac t2} \frac{dy}{dt} + \frac12e^{\frac t2} y = \frac13 e^{\frac t2} e^{\frac t3}
$$
and now the left-hand side is the result of a product rule
$$
\frac{d}{dt}\left[e^{\frac t2} y \right] = \frac13e^{\frac 56 t}.
$$

We integrate both sides to obtain
$$
e^{\frac t2} y = \frac13 \frac65 e^{\frac56 t} + c
$$
thus
$$
y %= \frac25 e^{\frac56 t-\frac12t} + c e^{-\frac t2} 
	= \frac25 e^{\frac t3} + c e^{-\frac t2}.
$$

\end{example}


This last example required us to come up with a function to multiply the ODE so that it becomes of the right form: with a left-hand side that is the result of the product rule.

This function is called the \emph{integrating factor}.

Let us now see how we can find this function in more detail.


\begin{example}
Consider the same ODE \eqref{eq:intfact_1}:
\begin{equation}\tag{$\star$}
\frac{dy}{dt} + \frac12 y = \frac13 e^{\frac t3}.
\end{equation}

So we multiply both sides of the equation with an unknown function $\mu(t)$, called the \emph{integrating factor}:
$$%\begin{equation}\tag{\#}\label{eq:intfactor2}
\mu(t) \frac{dy}{dt} + \frac12 \mu(t) y = \frac13 \mu(t) e^{\frac t3}.
$$%\end{equation}

And we find which $\mu(t)$ makes the left-hand side equal to the product rule:
$$
\frac{d}{dt} \big[ \mu(t) y \big] 
	= \mu(t) \frac{dy}{dt} + \frac{d \mu(t)}{dt} y,
$$
and this needs to equal the left-hand side:
$$
\mu(t) \frac{dy}{dt} + \frac{d \mu(t)}{dt} y = \mu(t) \frac{dy}{dt} + \frac12 \mu(t) y
\qquad \Leftrightarrow \qquad
	\mu'(t) = \frac12 \mu(t).
$$

We now need to solve this differential equation for $\mu(t)$. Fortunately, this is a \emph{separable ODE}:
\begin{align*}
\mu'(t) = \frac12 \mu(t) \quad
	& \Leftrightarrow \quad \frac{\mu'(t)}{\mu(t)} = \frac12 \\
	& \Leftrightarrow \quad \ln |\mu(t)| = \frac t2  + A\\
	& \Leftrightarrow \quad \mu(t) = a e^{\frac t2},
\end{align*}
where $a = e^A$.

We say that the function $\mu(t) = e^{\frac t2}$ is an \emph{integrating factor} for the equation \eqref{eq:intfact_1}.
Observe that we chose $a=1$ ($A=0$), because we only need \emph{one} function $\mu(t)$ that satisfies our condition $\mu' = \frac12 \mu$, we don't need to find all possible solutions. \\


After finding the integrating factor $\mu(t)$, the rest of the solution is the same as in the previous example.
%
%The equation \eqref{eq:intfactor2} then becomes
%$$
%e^{\frac t2} \frac{dy}{dt} + \frac12e^{\frac t2} y = \frac13 e^{\frac t2} e^{\frac t3}
%$$
%or equivalently
%$$
%\frac{d}{dt}\left[e^{\frac t2} y \right] = \frac13e^{\frac 56 t}.
%$$
%
%We integrate both sides to obtain
%$$
%e^{\frac t2} y = \frac13 \frac65 e^{\frac56 t} + c
%$$
%thus
%$$
%y %= \frac25 e^{\frac56 t-\frac12t} + c e^{-\frac t2} 
%	= \frac25 e^{\frac t3} + c e^{-\frac t2}.
%$$
\end{example}


Now that we have a good idea of the method needed to solve these ODEs, let us tackle the general equation.


\subparagraph{\emph{Method of solution. }} This method is also known as the \emph{Method of the Integrating Factor}. 
\begin{enumerate}[label={\bf \arabic*. } ]
\item Multiply both sides by $\mu(t)$, the integrating factor:
$$
\mu(t) \frac{dy}{dt} + p(t)\mu(t) y = \mu(t) g(t).
$$

Note that we don't know what this function is yet. So it is just a placeholder for a function we will find next.

\item Find function $\mu(t)$ which satisfies
$$
\mu'(t) = p(t) \mu(t).
$$

This is a \emph{separable ODE}, so we can solve it:
$$
\mu(t) = A e^{\int p(t) \, dt}.
$$

We only need one function $\mu(t)$, not the general one, so we take $A=1$ to get
$$
\mu(t) = e^{\int p(t) \, dt}.
$$

\item Observe that $\mu(t)$ satisfies
$$
\mu(t) p(t) = \mu'(t),
$$
so we use this in the equation:
$$
\frac{d}{dt} \big[ \mu(t) y \big] = \mu(t) g(t).
$$

\item Integrate the equation:
$$
\mu(t) y = \int \mu(t) g(t) \, dt + c,
$$
which means that the solution is
$$
y = \frac{1}{\mu(t)} \left[ \int \mu(t) g(t) \, dt + c \right],
$$
where 
$$
\mu(t) = e^{\int p(t) \, dt}.
$$

\end{enumerate}

\begin{important}
	Observe that the solution is given in explicit form. This is always the case with this type of ODEs.
	
	Also, be careful to add the integration constant as soon as you integrate, so that in the end you will have a term $\frac{c}{\mu(t)}$.
\end{important}


\begin{video}
\begin{itemize}
	\item \qrvideo{https://youtu.be/ezhi3E_bdvk}
	\item \qrvideo{https://youtu.be/VdD26Iy4Bkk}
	\item \qrvideo{https://youtu.be/GIpOcHNK7eQ}
\end{itemize}	
\end{video}

\end{submodule}

%\hfill \\

