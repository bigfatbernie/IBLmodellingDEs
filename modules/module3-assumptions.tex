In this module you will learn
\begin{itemize}
	\item that we need to make assumptions to be able to create a model
	\item how to strike a balance between accuracy and solvability
\end{itemize}

\hfill \\




Real problems are complex, so when modelling a real problem mathematically, we must make some assumptions. 

The assumptions that we make will affect the problem we are solving and its difficulty, so we need to strike a balance between:
\begin{itemize}
\item accuracy -- the fewer assumption the better, and
\item solvability -- the more assumptions the better.
\end{itemize}

\begin{annotation}
	\begin{goals}
		When building a mind map, keep track of the assumptions necessary for each step.
	\end{goals}
\end{annotation}

Many assumptions follow naturally when building a mind map. \\


\begin{annotation}
	\begin{goals}
		Remember to justify all your assumptions.
	\end{goals}
\end{annotation}

When figuring which assumption to make, keep in mind the key-factors of the problem and find data when available (usually online). 
If not available, measure data when possible, and if it's not possible, make a reasonable assumption on what the data might look like.

Another thing to keep in mind are \emph{time constraints}. Whether in a class, test, or working in a project, there will be deadlines. Your assumptions should take time constraints into consideration. 



\begin{example}

AN EXAMPLE, PROBABLY BASED ON THE RECYCLING.
	
\end{example}




%\hfill \\

%	\section*{Step D. Parameters or Variables?}\label{D-parvsvar}
%	\addcontentsline{toc}{subsection}{Step D. Parameters or Variables?}
%	
%	
%	
%	When you have defined the problem you want to solve and you have made your (initial) assumptions, it is then time to define some details of the problem.
%	
%	
%	
%	
%	With the problem statement clearly defined and an initial set of assumptions made (a list that will likely get longer), you are ready to start to define the details of your model. Now is the time to pause to ask what
%	is important that you can measure. Identifying these notions as variables, with units and some sense of their range, is key to building the model.
%	The purpose of a model is to predict or quantify something of interest. We refer to these predictions
%	as the outputs of the model. Another term we use
%	for outputs is dependent variables. We will also have independent variables, or inputs to the model. Some quantities in a model might be held constant, in which case they are referred to as model parameters. Let's look at a few simple examples that will help you distinguish between these concepts. We'll also see how they depend on your viewpoint and the problem statement.
%	
%	
%	
%	%There is a clear difference between \emph{variables} and \emph{parameters}. 
%	
%	\begin{definition}[Variables and Parameters]
%	
%	
%	A \emph{variable} represents a model state, and may change during simulation.
%	
%	A \emph{parameter} is commonly used to describe objects statically. A \emph{parameter} is normally a constant in a single simulation, and is changed only when you need to adjust your model behaviour. 
%	\end{definition}
%	
%	%Use a variable instead of a parameter if you need to model some data unit continuously changing over time. Use a parameter instead of a variable if you just need to model some parameter of an object changed only at particular moments of time.
%	
%	
%	
%	\begin{annotation}
%		\begin{goals}
%		\qrcode{https://en.wikipedia.org/wiki/Parameter\#Mathematical\_models}	
%		\end{goals}
%	\end{annotation}
%	\begin{note}{(from Wikipedia)}
%	The quantities appearing in the equations we classify into variables and parameters. The distinction between these is not always clear cut, and it frequently depends on the context in which the variables appear. 
%	
%	Usually a model is designed to explain the relationships that exist among quantities which can be measured independently in an experiment; these are the \emph{variables} of the model. 
%	
%	To formulate these relationships, however, one frequently introduces ``constants'' which stand for inherent properties of nature (or of the materials and equipment used in a given experiment). These are the \emph{parameters}.	
%	\end{note}






%
%
%
%The choice of question in the previous module should determine the \emph{dependent} variable.

%
%The \emph{parameters} are the independent variables in the problem, e.g. the speed of the elevators. The final answer will depend on the parameters in the problem. 
%
%We can estimate the parameters, and sometimes even change them. \\
%
%The \emph{variables} are dependent. This meant that if we change the parameters, the variables will change automatically. 
