In this module you will learn
\begin{itemize}
	\item different ways to analyze models with higher-order differential equations
\end{itemize}

\hfill \\



In this chapter, we have learned how to create models involving systems of ODEs and how to solve some special types of second-order ODEs.



In this module, we'll study one example using a few different methods.

\begin{example}
Consider the model we found earlier for a mass attached to a spring:

\begin{minipage}{0.85\textwidth}
\begin{itemize}
	\item $y(t)=$ vertical position of the mass, where $y=0$ is the position of the mass at rest
	\item $my''(t) = -k y(t) - \gamma y'(t)$
	\item $m,k,\gamma>0$ are constants for the mass, the stiffness of the spring, and the resistance of the dashpot.
\end{itemize}
\end{minipage}
\begin{minipage}{50pt}
\includegraphics*[height=100pt]{images/module16-spring-mass-dashpot.pdf}	
\end{minipage}
\end{example}

\hfill

\begin{center}
\textbf{\color{cyan}
Qualitative evolution of quantities
}
\end{center}


We can try to figure out how these quantities, $y(t)$ is going to increase or decrease as time goes by. \\

Let us imagine that initially  \quad $y(0)=1, y'(0)=0$.

Then, at $t=0$, we have
$$
\begin{cases}
y(0)=1 \\
y'(0)=0 \\
y''(0)= -k < 0 & \text{ (so mass is decelerating, meaning speed will become negative)}
\end{cases}
$$

This means that mass is decelerating, so we have two immediate consequences:
\begin{itemize}
	\item the speed will decrease and become negative
	\item the position will start decreasing
\end{itemize}

Once the speed is negative, we see another effect
$$
y''(t) = -k\underbrace{y(t)}_{\text{decreasing}} \underbrace{- \gamma y'(t)}_{\rm positive},
$$
so the acceleration is negative but approaching zero at time $t_1$:
$$
0=y''(t_1) = -ky(t_1) - \gamma y'(t_1)
\quad \Leftrightarrow \quad k y(t_1) = - \gamma y'(t_1)
$$

Let us summarize our results so far in a table:

\begin{graybox}
\begin{center}
\begin{tabular}{c||c|c|c|c|c|c|c|c}
$\pmb{t}$	& $0$ 		& 			& $t_1$ &  			& 			& 	&	& \hspace{1cm} $+\infty$ \\[5pt] \hline\hline
$\pmb{y}$ & $1$	& $ \searrow$	& $+$ &	$\searrow$ 	&	&  \hspace{0.5cm}	&  	& 	\\[5pt] \hline
$\pmb{y'}$ & $0$ &	$\searrow$	& $-$ & $\nearrow$ & 	&  	& &	\\[5pt] \hline
$\pmb{y''}$ & $-$		& $\nearrow$ & 0  & $\nearrow$  & 	& 	&  &	\\[5pt] \hline
\end{tabular}
\end{center}
\end{graybox}

The next milestone is:
$$
y(t_2)=0 \quad \text{ or } \quad y'(t_2)=0.
$$

If $y'(t_2)=0$ while $y(t_2)>0$, then 
$$
y''(t_2) = -ky(t_2) <0,
$$
which means that $y''$ would have had to decrease again, become zero and then negative, so must have been another milestone before.
We deduce that the next milestone is
$$
y(t_2)=0
\quad \Leftrightarrow \quad	
	y''(t_2) = -\gamma y'(t_2) > 0.
$$

\begin{graybox}
\begin{center}
\begin{tabular}{c||c|c|c|c|c|c|c|c}
$\pmb{t}$	& $0$ 		& 			& $t_1$ &  			&	$t_2$& 	&$t_3$	& \hspace{1cm} $+\infty$ \\[5pt] \hline\hline
$\pmb{y}$ & $1$	& $ \searrow$	& $+$ &	$\searrow$ 	& $0$	&  $\searrow$ 	& $-$ 	& $\cdots$	\\[5pt] \hline
$\pmb{y'}$ & $0$ &	$\searrow$	& $-$ & $\nearrow$ & $-$	&  $\nearrow$	& $0$ &	$\cdots$ \\[5pt] \hline
$\pmb{y''}$ & $-$		& $\nearrow$ & 0  & $\nearrow$  & $+$ & $\nearrow$	& $+$ &	$ \cdots$ \\[5pt] \hline
\end{tabular}
\end{center}
\end{graybox}

We can continue this analysis to conclude that the position seems to cycle back and forth between positive and negative, like you would expect from a spring. \\

In fact, this study is ``slightly'' flawed, since there is a possibility that the time $t_1$ never happens and the spring only approaches the state described without ever reaching it. This will happen for some configuration of the constants $m,k,\gamma$.




\hfill

\begin{center}
\textbf{\color{cyan}
Properties of the solutions
}
\end{center}



We learned earlier in the chapter how to solve this type of differential equations.

To solve them, we assume that solutions are of the form $y=e^{rt}$ and then find a characteristic equation for $r$:
$$
mr^2 = -k - \gamma r
\quad \Leftrightarrow \quad 
r = \frac{-\gamma \pm \sqrt{\gamma^2 - 4mk}}{2m}.
$$

Depending on the constants $m, k, \gamma$, we can have:
\begin{itemize}
	\item Two real distinct solutions 
	\item Two complex distinct solutions
	\item One repeated real solution
\end{itemize}

How do solutions behave in each case?
What kind of springs or dashpots behave in each case?





\hfill

\begin{center}
\textbf{\color{cyan}
Limiting behaviour of the solutions
}
\end{center}


From the analysis done above, we see that the possible values for $r$ are either negative or when $r$ is complex, its real part is negative (why?).

This means that the solution will have the form
$$
y(t) = e^{(\text{negative constant}) t} \left[  a\cos(\alpha t) + b\sin(\beta t) + c\right],
$$
so 
$$
\lim_{t \to \infty} y(t) = 0.
$$

This means that the mass will slow down and eventually stop.


%\newpage
\hfill

\begin{center}
\textbf{\color{cyan}
Numerical approximations
}
\end{center}



In module \ref{Approximation} we learned how to approximate solutions of first-order ODEs using Euler's Method.

We can extend that method to second-order ODEs, which we will leave as an exercise, and approximate the solution:

\begin{center}
\begin{tabular}{ccc}
\includegraphics*[width=125pt]{images/module21-approx-k2g5.png}
	& \includegraphics*[width=125pt]{images/module21-approx-k2g1.png}
	& \includegraphics*[width=125pt]{images/module21-approx-k2g0.png} \\
$k=2,\gamma=5$ 
	& $k=2,\gamma=1$ 
	& $k=2,\gamma=0$
\end{tabular}
\end{center}

These graphs also give us some intuition on how the solutions behave.

\begin{graybox}
You can access this simulation here:
\begin{itemize}
	\item \qrvideo{https://www.desmos.com/calculator/mufgdgku9w}
\end{itemize}	
\end{graybox}














