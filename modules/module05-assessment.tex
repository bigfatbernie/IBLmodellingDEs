\begin{siam}

In this module you will learn
\begin{itemize}
	\item how to analyze a model to check whether it makes sense
\end{itemize}

\hfill \\



At this point, you have defined a problem statement, and a mind map to help you decide how to approach the problem. You have made assumptions and made note of them and justified them.
You finally created a model to solve the problem.

The next step is to analyze the model.

There are two types of analysis:


\paragraph{\textcolor{cyan}{\textbf{Superficial assessment.}}} Are the units correct? Are the variables and parameters of a reasonable magnitude? Does it behave as expected? Does it make sense?



\paragraph{\textcolor{cyan}{\textbf{In-depth assessment.}}} Once the superficial assessment is verified, we need to understand the model at a deeper level. 

What are the model's strengths? What are its weaknesses?

When you change the inputs of the model, how do the outputs change? This is called {\emph sensitivity analysis}. 


Next is a simple example adapted from \cite{bliss}.


\begin{annotation}
	\begin{goals}
	\Goal{Desmos Graph}
	\hfill \qrcode{https://www.desmos.com/calculator/z9cftzus0z}	
	\end{goals}
\end{annotation}

\begin{example}\textbf{Modelling the flu}

History of the project:
\begin{itemize}
	\item Split population into two classes: \emph{infected} and \emph{not infected}
	\item Assume that each infected person infects $R$ number of non infected people every $b$ days
	\item Define $I(n) = $ number of infected people after $n$ days
	\item The two previous points imply \quad $I(n \cdot b) = R \cdot I(n)$
	\item We can then conclude that \quad $I(n b) = (1+R)^n \, I(0)$ \hfill (why?) \\
\end{itemize}

After plotting the resulting function $I(n)$ (click or follow the QR code on the right), we can assess our model: \\

\emph{Strengths:}
\begin{itemize}
	\item After two days $(b=2)$, there are 6 infected people, so it is following our assumption
	\item The number of infected people increases faster and faster as expected 
	\item The disease spreads at a constant rate. Also on Desmos, check the infection rate $\dfrac{I(n+b)}{I(n)}$
	\item We could find an explicit formula for the number of infected individuals $I(n)$ \\
\end{itemize}


\emph{Weaknesses:}
\begin{itemize}
	\item The model is too simple, so it doesn't model the spread of the flu accurately
	\item The model an exponential rate of infection, which is not possible for very long
	\item The model predicts that eventually the disease will spread to everyone
	\item The model assumes that there are only two types of people: infected and susceptible. Do people recover from the disease?
\end{itemize}

\end{example}




After assessing the model, if time allows, it is important to re-think the model and the assumptions made.

\end{siam}

