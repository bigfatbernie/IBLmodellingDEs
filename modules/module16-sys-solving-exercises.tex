\begin{exercises}

	\begin{problist}
	\prob \label{mod16-gensol}Find the general solution of the problem $\vec{r}'(t) = A \vec{r}(t)$ for the following matrices:
	\begin{enumerate}
	\begin{minipage}{.2\textwidth}
		\item $A = \begin{bmatrix} -7 & 6 \\ -9 & 8 \end{bmatrix}$;
		\item $A = \begin{bmatrix} 22 & 24 \\ -15 & -16\end{bmatrix}$;
		\item $A = \begin{bmatrix} 0 & 1 \\ -5 & 0 \end{bmatrix}$;
		\item $A = \begin{bmatrix} 0 & 1 \\ 5 & 0 \end{bmatrix}$;
		\item $A = \begin{bmatrix} 1 & \sqrt{3} \\ \sqrt{3} & -1\end{bmatrix}$;
		\item $A = \begin{bmatrix} 1 & \sqrt{3} \\ -\sqrt{3} & 1\end{bmatrix}$;
	\end{minipage}
	\qquad
	\begin{minipage}{.2\textwidth}
		\item $A = \begin{bmatrix} 0 & 1 \\ -4 & -4 \end{bmatrix}$;
		\item $A = \begin{bmatrix} -4 & -6 \\ 2 & 3 \end{bmatrix}$;
		\item $A = \begin{bmatrix} 2 & -3 \\ 0 & 2 \end{bmatrix}$;
		\item $A = \begin{bmatrix} 2 & 0 \\ 0 & 2 \end{bmatrix}$;
		\item $A = \begin{bmatrix} 0 & 0 \\ 1 & 0\end{bmatrix}$; 
		\item $A = \begin{bmatrix} 0 & 0 \\ 0 & 1\end{bmatrix}$; 
	\end{minipage}
	\end{enumerate}

	\prob For each of the problems in the previous exercise, find the solution that satisfies the initial conditions:
	\begin{enumerate}[label=(\roman*)]
		\item $\vec{r}(0)=\begin{bmatrix} 0 \\ 0 \end{bmatrix}$;
		\item $\vec{r}(0)=\begin{bmatrix} 1 \\ 3 \end{bmatrix}$;
		\item $\vec{r}(1)=\begin{bmatrix} -2 \\ 2 \end{bmatrix}$.
	\end{enumerate}


	\prob Consider the problem $\vec{r}'(t) = \begin{bmatrix} 2 & 0 \\ 1 & 3 \end{bmatrix} \vec{r}(t) + \begin{bmatrix} -2 \\ 11 \end{bmatrix}$.
	\begin{enumerate}
		\item Show that $\vec{e}(t) = \begin{bmatrix} 1 \\ -4 	\end{bmatrix}$ is a solution of this problem.
		\item Find the general solution of 
		$$\vec{u}'(t) = \begin{bmatrix} 2 & 0 \\ 1 & 3 \end{bmatrix} \vec{u}(t).$$
		\item Let $\vec{r}(t) = \vec{u}(t) + \vec{e}(t)$. Show that this is a solution of the original problem.
		\item Let $\vec{r}_1(t)$ and $\vec{r}_2(t)$ be two solutions of the original problem. 
		\begin{enumerate}
			\item Is $\vec{r}_1(t) + \vec{r_2}(t)$ a solution? 
			\item Is $3\vec{r}_1(t) $ a solution? 
			\item Write a result on how one can safely combine solutions of non-homogeneous problems.
		\end{enumerate}
	\end{enumerate}

	
	\prob \label{prob:sys-nonhomogeneous}Consider the problem  \quad $\vec{r}'(t) = \begin{bmatrix} 1 & 2 \\ 3 & 0 \end{bmatrix}
 \vec{r}(t)+\begin{bmatrix} -5 \\ 3 \end{bmatrix}
$.
	\begin{enumerate}
		\item Observe that this system is an autonomous system of ODEs. What is the equilibrium solution? 
		\item Let the equilibrium solution solution you just found be called $\vec{e}$. Consider $\vec{u}(t) = \vec{r}(t) - \vec{e}$, where $\vec{r}(t)$ is the solution of the original problem. Show that 
			$$ \vec{u}'(t) = A \, \vec{u}(t).$$
		\item Find $\vec{u}(t)$.
		\item Find $\vec{r}(t)$.
		\item Write a procedure to solve any problem of the form
			$$ \vec{r}(t) = A \, \vec{r}(t) + \vec{b}. $$
	\end{enumerate}
	
	
	\prob \label{prob:sys-superposition}Consider the problem \quad $\vec{r}'(t) = A \, \vec{r}(t)$.
	Assume that we have two solutions $\vec{r_1}(t)$ and $\vec{r_2}(t)$.
	\begin{enumerate}
		\item Show that $\vec{r}(t) = \vec{r_1}(t) + \vec{r_2}(t)$ is a solution also.
		\item Show that $\vec{r}(t) = 2\vec{r_1}(t) - 3\vec{r_2}(t)$ is a solution also.
		\item Find all possible solutions of the problem.
	\end{enumerate}
	
	
		\prob Consider the problem $\vec{r}'(t) = \begin{bmatrix} 1 & -2 \\ -2 & 1 \end{bmatrix} \vec{r}(t)$.
		\begin{enumerate}
			\item Find the solution that satisfies the initial condition $\vec{r}(0)=\begin{bmatrix}1 \\ 0\end{bmatrix}$. Call it $\vec{u}(t)$.
			\item Find the solution that satisfies the initial condition $\vec{r}(0)=\begin{bmatrix}0 \\ 1\end{bmatrix}$. Call it $\vec{v}(t)$.
			\item Define the matrix function
			$$ \Phi(t) = \begin{bmatrix} \vec{u}(t) \; | \; \vec{v}(t) \end{bmatrix} = \begin{bmatrix} u_1(t) & v_1(t) \\ u_2(t) & v_2(t) \end{bmatrix}.$$
			
			Show that $\vec{r}(t) = \Phi(t) \vec{r}_0$ is a solution of the original system of ODEs. Which initial condition does it satisfy?
			
			\item Write a result relating $\Phi(t)$ to the solution of initial-value problems.
			\end{enumerate}

	
	
		\prob \label{mod16:prob-W1}Consider a system of ODEs $\vec{r}'(t) = A \vec{r}(t)$ with two solutions $\vec{r}_1(t)$ and $\vec{r}_2(t)$. 
		
		We want to study the conditions that are necessary on the solutions $\vec{r}_1$ and $\vec{r}_2$ to guarantee that we can solve any initial-value problem.
		
		\begin{enumerate}
			\item What is the general solution for this problem?
			\item If the initial condition is $\vec{r}(0)= \begin{bmatrix} 1 \\ 2 \end{bmatrix}$, then what are the conditions on $\vec{r}_1,\vec{r}_2$ ?
			\item If the initial condition is $\vec{r}(0)= \vec{r_0}$, then what are the conditions on $\vec{r}_1,\vec{r}_2$ ?
		\end{enumerate}

		\prob Consider a system of ODEs $\vec{r}'(t) = A \vec{r}(t)$ with two solutions $\vec{r}_1(t), \vec{r}_2(t)$.
		
			Let $R(t)$ be the matrix 
				$R(t) = \begin{bmatrix} \vec{r}_1(t) & | & \vec{r}_2(t)	\end{bmatrix} $ 
				and let $W(t) = \det R(t)$.
		
		\begin{enumerate}
			\item Show that $W(t)$ is a solution of $W' = (a_{11} + a_{22}) W$.
			\item Solve the ODE above to obtain an expression for $W(t)$.
			\item Show that $W(t)$ is either identically zero, or it's never zero. 
			\item Use this result to simplify your answer to problem \ref{mod16:prob-W1}(c).
		\end{enumerate}
	
	\end{problist}
\end{exercises}
