	In this module you will learn
	\begin{itemize}
		\item how to define a problem mathematically.
	\end{itemize}

\hfill \\


The first step is to define the problem we want to solve.

\textbf{To do this, we should start from the end! }

We need to decide on what kind of mathematical object we will use in the end to show that we solved the problem we were tasked with. \\


Once this is done, we can define the problem mathematically. 

\begin{example}
	Your team was tasked with optimizing the layout of an airport. 

	The team decided to define:
	\begin{itemize}
		\item $T = $ the total time (in minutes) necessary by the average person to walk from their airport transportation (taxi, train, bus) to their gate, disregarding the time spent in security or immigration.
	\end{itemize}

	At the end of the project, to show that the team did find a good layout for the airport, the team will show that the new layout reduces the value of $T$. \\

	Once this decision is made, the problem to solve (or improve) becomes clear:
	
	\begin{itemize}
		\item Minimize $T$
	\end{itemize}

	There will probably be some constraints, which will be studied in Module \ref{assumption}.

\end{example}
