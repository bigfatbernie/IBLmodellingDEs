In this module you will learn
\begin{itemize}
	\item how to build a model based on the previous steps
\end{itemize}

\hfill \\



This is the part of the modelling where we connect all that we have done so far: the problem we defined, the mind map, the assumptions, and all the variables and parameters in a mathematical model to answer the ``mathematical'' problem defined in \hyperref[define]{Step A}.

This usually means writing down mathematical equations, constructing a graph, analyzing a geometric figure, or do some statistical analysis. \\


\begin{example}
Your team is tasked with finding the best recycling centre (we looked at this example in \hyperref[mindmap]{Step B}) and your  team has chosen to minimize the cost to the city by using drop off centres.

As part of modelling process, your team has made the following assumptions/measurements:
\begin{itemize}
	\item People would be willing to pay \$2.29 to recycle per month or \$0.53 per week
	\item People would make bi-weekly trips to the centre
	\item Gasoline costs around \$1.26 per litre
	\item On average a passenger car needs 10 litres per hundred kilometres
\end{itemize}

This means that the (one-way) distance people are willing to travel every week to the drop-off centre is
$$
d \;=\; \frac{1}{4.3 \text{ trips/month}} \cdot \frac{\$2.29 / {\text{month}} }{(\$1.26 \text{/L}) \ \cdot\  (0.1 \text{ L / km})} \;=\; 4.2 \text{  km/trip}.
$$

This should help us figure out the best way to place the drop-off centres:

The Mathematical model might look like this

\begin{itemize}
	\item Maximize (number of people within a 4.2 km radius of a drop-off centre)
	\item subject to a certain number of drop-off centres (given by the city budget)
\end{itemize}

	
\end{example}

\hfill

Sometimes, the mathematical tools necessary to tackle the problem are clear, but often they are not. In those cases it may be helpful to analyze some simple cases.

