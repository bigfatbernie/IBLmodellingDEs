In this module you will learn
\begin{itemize}
	\item how to solve systems of two linear first-order ODEs with constant coefficients
\end{itemize}

\hfill \\


First, a system of two first-order ODEs has the form:
$$
\begin{cases}
x'(t) = f\big(t, x(t),y(t)\big) \\	
y'(t) = g\big(t, x(t),y(t)\big)
\end{cases}
$$
where the functions $f$ and $g$ are continuous and have continuous partial derivatives. \\

This system could be nonlinear. We are only considering \emph{linear systems with constant coefficients}, which means that they have a very specific form:
$$
\begin{cases}
	x'(t) = a x(t) + b y(t) + e\\
	y'(t) = c x(t) + d y(t) + f
\end{cases}
%
\quad \Leftrightarrow \quad 
	\begin{bmatrix} x'(t) \\ y'(t) \end{bmatrix}
	=
	\begin{bmatrix} a & b \\ c & d \end{bmatrix}
	\begin{bmatrix} x(t) \\ y(t) \end{bmatrix}
	+	\begin{bmatrix} e \\ f \end{bmatrix}
%
\quad \Leftrightarrow \quad 
	\vec{r}'(t)
	=
	A \, \vec{r}(t) + \vec{b}
$$
where
$$
\vec{r}(t) = \begin{bmatrix} x(t) \\ y(t) \end{bmatrix}
\quad , \quad 
A = \begin{bmatrix} a & b \\ c & d \end{bmatrix}
\quad and \quad 
\vec{b}=\begin{bmatrix} e \\ f \end{bmatrix}.
$$

The unknown functions we are trying to find are $x(t)$ and $y(t)$, or equivalently, $\vec{r}(t)$. \\


\paragraph{\emph{Homogeneous Systems.}} These are systems of the form above with $\vec{b} = \vec{0}$.

This means that we want to find all functions $\vec{r}(t)$ that satisfy
$$
\vec{r}'(t) = A \vec{r}(t).
$$


\begin{example}
Let us start with an example of the same problem where $\vec{r}(t)$ is a ``one-dimensional'' vector, a scalar function $u(t)$, and the matrix $A$ is a ``one-dimensional matrix'', a constant $a$. \\

We want to solve the problem	
$$
u'(t) = a \cdot u(t).
$$

We have seen how to solve these kind of problems before. The solutions are
$$
u(t) = c e^{at},
$$
where $c$ can be any constant.
\end{example}

In our two-dimensional case, it is a little more complicated. We can't just write $e^{At}$ where $A$ is a matrix (this expression can make sense, but we would have to find out what is the exponential of a matrix). \\

So we can use the example above to make an \emph{educated guess}: the solution should look like an exponential:
$$
\vec{r}(t) = 
\vec{c} \, e^{\lambda t},
$$
where $\vec{c}$ is a constant vector. \\

If our guess is correct, to find $\vec{r}(t)$, we only need to find $\lambda$ and $\vec{c}$. \\

Let us see what happens when we use this guess and plug it into the system of ODEs:
\begin{align*}
\vec{r}'(t) = A \vec{r}(t) \quad
	& \Leftrightarrow \quad \vec{c} \lambda e^{\lambda t} = A \vec{c} e^{\lambda t} \\
	& \Leftrightarrow \quad \vec{c} \lambda = A \vec{c}.
\end{align*}

This is a problem you have seen before -- and eigenvalue-eigenvector problem: 
\begin{itemize}
	\item $\lambda$ can be any eigenvalue of the matrix $A$
	\item $\vec{c}$ can be any eigenvector of $A$ associated with $\lambda$ 
\end{itemize}

This means that we might have multiple choices for eigenvalues and eigenvectors, or even that eigenvalues and eigenvectors involve complex numbers.
Let us split our study of possible solutions in three cases.



%%%%%%%%%       %%%%%%%%%       %%%%%%%%%       %%%%%%%%%       %%%%%%%%%

\subsection{Two real and distinct eigenvalues}

\begin{example}
Consider the problem
$$
\vec{r}'(t) = \begin{bmatrix}	
 10 & 18 \\ -6 & -11	
 \end{bmatrix} \, \vec{r}(t).
$$

Then, the eigenvalues and eigenvectors of the matrix are
\begin{itemize}
	\item $\lambda_1=-2$ with eigenvector $\vec{v}_1 = \begin{bmatrix} 3 \\ -2 \end{bmatrix}$
	\item $\lambda_2=1$ with eigenvector $\vec{v}_2 = \begin{bmatrix} 2 \\ -1 \end{bmatrix}$
\end{itemize}

This means that we found two solutions:
$$
\vec{r}_1(t) = \begin{bmatrix} 3 \\ -2 \end{bmatrix} e^{-2t}
\quad \text{ and } \quad \vec{r_2}(t) = \begin{bmatrix} 2 \\ -1 \end{bmatrix} e^{t}.
$$

Then, we can show that 
$$
\vec{r}(t) = c_1 \begin{bmatrix} 3 \\ -2 \end{bmatrix} e^{-2t}
+c_2\begin{bmatrix} 2 \\ -1 \end{bmatrix} e^{t}
$$
is also a solution of the problem for any constants $c_1$ and $c_2$.

In fact, we can show that this formula captures all possible solutions for this problem.
\end{example}


\begin{video}
	\begin{itemize}
		\item \qrvideo{https://youtu.be/YUjdyKhWt6E}
	\end{itemize}
\end{video}




%%%%%%%%%       %%%%%%%%%       %%%%%%%%%       %%%%%%%%%       %%%%%%%%%

\subsection{Two complex eigenvalues}

We actually don't need to know a lot about complex numbers to be able to understand how to solve this case.
The results about complex values that are necessary to know will be included in the box below.

\begin{definition}[Complex numbers] \label{EulersFormula}
\begin{itemize}
	\item A complex number is a number of the form $z=a+ib$ where $i$ is called the imaginary constant and satisfies $i^2=-1$.
	\item Given a complex number $z=a+ib$, we call $\overline{z}=a-ib$ its complex conjugate. It satisfies:
	$$ z \cdot \overline{z} = a^2+b^2 = |z|^2.$$

	\item If a matrix has real components and two complex eigenvalues, then the eigenvalues are complex conjugates of each other. Moreover, the eigenvectors are also complex conjugates of each other.
	\item Euler's Formula: $e^{i \theta} = \cos(\theta) + i \sin (\theta)$.
\end{itemize}
\end{definition}




\begin{example}
Consider the problem
$$
\vec{r}'(t) = \begin{bmatrix}	
 1 & 1 \\ -1 & 1
 \end{bmatrix} \, \vec{r}(t).
$$

Then, the eigenvalues and eigenvectors of the matrix are
\begin{itemize}
	\item $\lambda_1=1+i$ with eigenvector $\vec{v}_1 = \begin{bmatrix} -i \\ 1 \end{bmatrix}$
	\item $\lambda_2=1-i$ with eigenvector $\vec{v}_2 = \begin{bmatrix} i \\ 1 \end{bmatrix}$
\end{itemize}

This means that we found two solutions:
$$
\vec{r}_1(t) = \begin{bmatrix} -i \\ 1 \end{bmatrix} e^{(1+i)t}
\quad \text{ and } \quad \vec{r}_2(t) = \begin{bmatrix} i \\ 1 \end{bmatrix} e^{(1-i)t}.
$$

Then, all solutions of this system of ODEs can be expressed as
$$
\vec{r}(t) = c_1 \begin{bmatrix} -i \\ 1 \end{bmatrix} e^{(1+i)t}
+c_2\begin{bmatrix} i \\ 1 \end{bmatrix} e^{(1-i)t}.
$$

There is a problem with the form of these solutions: they involve complex numbers!

Imagine that we start with a problem where we have two (real) quantities that interact with each other through this system of differential equations.
Then we expect these quantities to measure in real numbers, not complex.

\begin{itemize}
	\item This means that we expect \emph{the imaginary part of these solutions to cancel out}.
\end{itemize}

So let us manipulate this formula using Euler's formula and see if we can re-write in such a way that doesn't involve complex numbers. \\

We have:
\begin{align*}
	e^{(1+i)t} &= e^t \cdot e^{it} = e^t \big( \cos(t) + i \sin(t)\big) \\
	e^{(1-i)t} &= e^t \cdot e^{-it} = e^t \big( \cos(t) - i \sin(t)\big)
\end{align*}

So our solution expands to:
$$
\vec{r}(t) = c_1 \begin{bmatrix} -i \\ 1 \end{bmatrix} e^t \big( \cos(t) + i \sin(t)\big)
+c_2\begin{bmatrix} i \\ 1 \end{bmatrix} e^t \big( \cos(t) - i \sin(t)\big).
$$

We can now manipulate this expression:
\begin{align*}
\vec{r}(t) 
	& = e^t
		\begin{bmatrix}
			-i c_1 	\big( \cos(t) + i \sin(t)\big) + i c_2 \big( \cos(t) - i \sin(t)\big) \\
			c_1 	\big( \cos(t) + i \sin(t)\big) + c_2 \big( \cos(t) - i \sin(t)\big) \\
		\end{bmatrix} \\
	& = e^t 
		\begin{bmatrix}
			(c_1+c_2) \sin(t) - i (c_1-c_2) \cos(t) \\
			(c_1+c_2) \cos(t) + i(c_1-c_2) \sin(t)
		\end{bmatrix} \\
	& = e^t \left( (c_1+c_2) \begin{bmatrix} \sin(t) \\ \cos(t) \end{bmatrix}
		+ i (c_1 -c_2) \begin{bmatrix} - \cos(t) \\ \sin(t) \end{bmatrix}
		\right)
\end{align*}

So now we do something that might look like cheating. We define:
$$
a_1 = c_1+c_2 \quad \text{ and } \quad a_2 = i (c_1-c_2).
$$


Then the solution is
$$
\vec{r}(t) = a_1\begin{bmatrix} \sin(t) \\ \cos(t) \end{bmatrix} e^t
		+ a_2 \begin{bmatrix} - \cos(t) \\ \sin(t) \end{bmatrix} e^t.
$$

This last form doesn't include any complex numbers and is equivalent to the previous form.

\end{example}

\begin{graybox}
	It may look like the final solution above still includes complex numbers in the constants $a_1$ and $a_2$. 
	
	To convince yourself that this is not the case, solve the following exercise.\\
	
	Find the unique solution of
	$$
	\vec{r}'(t) = \begin{bmatrix}	
 		1 & 1 \\ -1 & 1
		\end{bmatrix} 
		\, \vec{r}(t)
	\qquad \text{ with} \qquad
	\vec{r}(0) = \begin{bmatrix}
 			-3 \\ 2
	 \end{bmatrix}
	 $$
	 
	 Find the constants $c_1, c_2$ and then the constants $a_1,a_2$. Which ones are complex and which ones are real?
\end{graybox}

\begin{video}
	\begin{itemize}
		\item \qrvideo{https://youtu.be/TRVS5Wo9LoM}
	\end{itemize}
\end{video}



%%%%%%%%%       %%%%%%%%%       %%%%%%%%%       %%%%%%%%%       %%%%%%%%%

\subsection{One real repeated eigenvalue}


\begin{example}
Consider the problem
$$
\vec{r}'(t) = \begin{bmatrix}	
 5 & 0 \\ 1 & 5	
 \end{bmatrix} \, \vec{r}(t).
$$

Then, there is only one eigenvalue with one eigenvector
\begin{itemize}
	\item $\lambda_1=5$ with eigenvector $\vec{v}_1 = \begin{bmatrix} 0 \\ 1 \end{bmatrix}$, which yield a solution $\vec{r}_1(t) = c_1 \begin{bmatrix} 0 \\ 1 \end{bmatrix} e^{5t}$.
\end{itemize}

This is a \emph{problem} because we need two solutions to put together and obtain two constants, as in the two previous cases.
%\end{example}

\begin{graybox}
To convince yourself that it is a problem, try solving the problem above with initial conditions
$$
\vec{r}(0) = \begin{bmatrix}	0 \\ 4 \end{bmatrix}.
$$

What about with initial conditions
$$
\vec{r}(0) = \begin{bmatrix}	 1 \\ 4 \end{bmatrix} \quad ?
$$
\end{graybox}

%\begin{example}

This means that we ware missing one solution -- that will enable us to solve the problem for any initial conditions. \\


Let us re-write the original problem in a different form by letting 
$$
\vec{r}(t) = \begin{bmatrix} x(t) \\ y(t)\end{bmatrix}.
$$
Then we have
$$
\begin{cases}
x'(t) = 5 x(t) \\
y'(t) = x(t) +5 y(t)	
\end{cases}
$$
These are two ODEs, but the first one is independent of the second on. We can solve the first and then tackle the second one.
$$
\begin{cases}
	x(t) = c_2 e^{5t} \\
	y(t) = 	c_1 e^{5t} + c_2 t e^{5t} 
\end{cases}
\quad \Leftrightarrow \quad
	\vec{r}(t) = \begin{bmatrix}
		c_2 \\ c_1 + c_2 t
	\end{bmatrix} e^{5t}
\quad \Leftrightarrow \quad
	\vec{r}(t) = c_1 \begin{bmatrix} 0 \\ 1 \end{bmatrix} e^{5t} + c_2
	\begin{bmatrix}	1 \\ t
	\end{bmatrix} e^{5t}
$$
\end{example}

Observe that the solution we found has the form:
$$
	\vec{r}(t) = \underbrace{c_1 \begin{bmatrix} 0 \\ 1 \end{bmatrix} e^{5t}}_{\text{solution } \vec{r}_1(t)} + c_2 \left( 
	\underbrace{\begin{bmatrix}	1 \\ 0 \end{bmatrix}}_{\text{new vector } \vec{w}} e^{5t} + \underbrace{\begin{bmatrix} 0  \\ 1 
	\end{bmatrix}}_{\vec{v}_1} t e^{5t}\right).
$$

So we can make another \emph{educated guess} that the solution we were missing has the form:
$$
\vec{r}_2(t) = \left(\vec{w} + \vec{v}_1 t\right) e^{\lambda t}.
$$

With this form in mind, we can plug it into the system of ODEs \quad $\vec{r}'(t) = A \vec{r}(t)$ \quad, which has exactly one eigenvalue $\lambda$, to get:
$$
\lambda \vec{w}e^{\lambda t} + \lambda \vec{v}_1 t e^{\lambda t} + \vec{v}_1 e^{\lambda t}
= A \vec{w} e^{\lambda t} + A \vec{v}_1 t  e^{\lambda t}
$$
which is equivalent to:
$$
\lambda \vec{w} + \underbrace{\lambda \vec{v}_1}_{=A \vec{v}_1} t  + \vec{v}_1
= A \vec{w}  + A \vec{v}_1 t 
	\quad \Leftrightarrow \quad
	\left( \lambda I - A \right ) \vec{w} = \vec{v}_1
$$
%which becomes
%$$
%\lambda \vec{w} + \vec{v}_1
%= A \vec{w}  
%	\quad \Leftrightarrow \quad
%	\left( \lambda I - A \right ) \vec{w} = \vec{v}_1
%$$

Since at this point we already know $\lambda$ and $\vec{v}_1$, we can now find $\vec{w}$ in a similar way used to find the eigenvector $\vec{v}_1$. The vector $\vec{w}$ is called a \emph{generalized eigenvector} of $A$ associated with the eigenvalue $\lambda$.


\begin{video}
	\begin{itemize}
		\item \qrvideo{https://youtu.be/hCShTLmeZN4}
	\end{itemize}
\end{video}





