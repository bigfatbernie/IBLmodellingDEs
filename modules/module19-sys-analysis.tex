In this module you will learn
\begin{itemize}
	\item different ways to analyze models with several differential equations
\end{itemize}

\hfill \\



In this chapter, we have learned how to create models involving systems of ODEs and how to solve some special types of systems of ODEs.

Once we create a model that involves a system of ODEs, the ultimate goal is not to solve the system of ODEs, but to be able to understand how the situation will proceed.
Solving the system of ODEs is often a large step in that direction, but it is more important to be able to take the solution and knowing how to interpret in light of the original situation.

Sometimes, when we cannot find an explicit formula for the solution, it is still possible to study the system to find some properties and behaviours of the solutions. \\

In this module, we'll study one example using a few different methods.


\begin{example}
The goal here is not the modelling but the analysis of the model, so we will quickly explain the model. \\

We are going to model population versus cost of living in Toronto.


Consider the following functions
\begin{itemize}
\item $p(t) = $ Population of Toronto (GTA) in millions at time $t$ in years since the beginning of 2015.
\item $c(t) = $ Cost of living in Toronto (in thousands of dollars) at time $t$.
\item Define a vector $\vec{x}(t) = \begin{bmatrix} p(t) \\ c(t) \end{bmatrix}$.
\end{itemize}

These two factors are related according to the following properties:
\begin{itemize}
\item In the absence of any migration, the population will decrease proportionally to the cost of living (with constant $a$);
\item There are always people moving into Toronto independently of its current population or cost of living (with constant $b$)
\item In the absence of any other factors, the cost of living; increases proportionally to the population (with constant $d$)
\item In the absence of any other factors, the cost of living; increases proportionally to the cost of living due to inflation (with constant $e$);
\item The city is always expanding, so the cost of living is always decreasing independently of its current population or cost of living (with constant $f$).
\end{itemize}
The constants $a,b,d,e,f$ are all positive. \\

Our model is:
$$
\vec{x}'(t) = 
	\begin{bmatrix}
 		0 & -a \\
 		d & e
	\end{bmatrix}
					\vec{x}(t) + 
	\begin{bmatrix}
		b \\ -f
	\end{bmatrix}
$$
\end{example}

\hfill

\begin{center}
\textbf{\color{cyan}
Qualitative evolution of quantities
}
\end{center}


%\paragraph{Qualitative evolution of quantities.}
We can try to figure out how these quantities, $p(t)$ and $c(t)$, are going to increase or decrease. \\

As an academic example, let us imagine that initially  \quad $p(0)=c(0)=0$.

Then, at $t=0$, we have
$$
p'(0)= b > 0 \quad \text{ and } \quad
c'(0)=-f < 0.
$$

This means that $p(t)$ is increasing while $c(t)$ wants to decrease.

Here we need to make sure that everything still makes sense: since it doesn't make sense to have a negative cost of living (government incentives to move into the city?!), we need to disregard our system and assume that $c(t)$ will continue constant while $c'(t)<0$.

We then have:
\begin{graybox}
\begin{center}
\begin{tabular}{c||c|c|c|c|c|c|c|c}
$\pmb{t}$	& $0$ 		& 			& &  			& 			& 	&	& \hspace{1cm} $+\infty$ \\[5pt] \hline\hline
$\pmb{p}$ & $0$	& $\nearrow$	& \hspace{0.5cm} &	\hspace{1cm}	&	&  \hspace{0.5cm}	&  	& 	\\[5pt] \hline
$\pmb{c}$ & $0$		& $\rightarrow$	&  0 & \hspace{1cm} & 	\hspace{0.5cm}	& \hspace{1cm} 	& \hspace{0.5cm}	&\hspace{1cm}	\\[5pt]
\end{tabular}
\end{center}
\end{graybox}

While $c(t)=0$, we have
$$
p'(t)= b > 0  \quad \text{ and } \quad 
c'(t)=d\, p(t) -f.
$$

This means that $p(t)$ is increasing with constant slope (linearly) until $c'(t_1)=0$.
We can figure out when this will happen:
$$
0=c'(t_1)=d\, p(t_1) -f \quad \Leftrightarrow \quad p(t_1) = \frac{f}{d}.
$$

So we continue our table:
\begin{graybox}
\begin{center}
\begin{tabular}{c||c|c|c|c|c|c|c|c}
$\pmb{t}$	& $0$ 		& 			& $t_1$ &  			& 			& 	&	& \hspace{1cm} $+\infty$ \\[5pt] \hline\hline
$\pmb{p}$ & $0$	& $\nearrow$	& $\displaystyle\frac{f}{d}$ &	\hspace{1cm}	&	&  \hspace{0.5cm}	&  	& 	\\[5pt] \hline
$\pmb{c}$ & $0$		& $\rightarrow$	&  0 & \hspace{1cm} & 	\hspace{0.5cm}	& \hspace{1cm} 	& \hspace{0.5cm}	&\hspace{1cm}	\\[5pt]
\end{tabular}
\end{center}
\end{graybox}

What happens after $t_1$?

Consider $t>t_1$ slightly after $t_1$. Then
$$
\begin{cases}
p'(t) = -a c(t) + b >0 & \text{ still positive because $c(t)$ is very small, but the slope is decreasing} \\
c'(t) = d p(t) + e c(t) - f	>0 & \text{ increasing quickly as both $p$ and $c$ increase}
\end{cases}
$$

\begin{graybox}
\begin{center}
\begin{tabular}{c||c|c|c|c|c|c|c|c}
$\pmb{t}$	& $0$ 		& 			& $t_1$ &  			& 			& 	&	& \hspace{1cm} $+\infty$ \\[5pt] \hline\hline
$\pmb{p}$ & $0$	& $\nearrow$	& $\displaystyle\frac{f}{d}$ &	\IncDown	&	&  \hspace{1cm}	&  	& 	\\[5pt] \hline
$\pmb{c}$ & $0$		& $\rightarrow$	&  0 & \IncUp & 	\hspace{0.5cm}	& \hspace{1cm} 	& \hspace{0.5cm}	&\hspace{1cm}	\\[5pt]
\end{tabular}
\end{center}
\end{graybox}

At a certain time $t_2$, the population will stop increasing. Let us find out when this happens:
$$
0=p'(t_2) = -a c(t_2) + b >0 
	\quad \Leftrightarrow \quad c(t_2) = \frac{b}{a}.
$$

\begin{graybox}
\begin{center}
\begin{tabular}{c||c|c|c|c|c|c|c|c}
$\pmb{t}$	& $0$ 		& 			& $t_1$ &  			& 	$t_2$		& 	&	& \hspace{1cm} $+\infty$ \\[5pt] \hline\hline
$\pmb{p}$ & $0$	& $\nearrow$	& $\displaystyle\frac{f}{d}$ &	\IncDown	& $\rightarrow$	&  \hspace{1cm}	&  	& 	\\[5pt] \hline
$\pmb{c}$ & $0$		& $\rightarrow$	&  0 & \IncUp & 	 $\displaystyle \frac{b}{a}$	& \hspace{1cm} 	& \hspace{0.5cm}	&\hspace{1cm}	\\[5pt]
\end{tabular}
\end{center}
\end{graybox}

After this point we have $t>t_2$ slightly after $t_2$:
$$
\begin{cases}
p'(t) = -a c(t) + b <0 & \text{ decreasing rapidly while $c'(t)>0$}\\
c'(t) = d p(t) + e c(t) - f	>0 & \text{ still increasing quickly until $p(t)=0$}
\end{cases}
$$

We expect that at some point $p(t_3)=0$. From that point on we have
$$
\begin{cases}
p'(t_3) = -a c(t_3) + b <0 & \text{ we need to ignore the model at this point and keep $p$ constant}\\
c'(t_3) = e c(t_3) - f	>0 & \text{ still increasing exponentially}
\end{cases}
$$

So this is our final table:
\begin{graybox}
\begin{center}
\begin{tabular}{c||c|c|c|c|c|c|c|c}
$\pmb{t}$	& $0$ 		& 			& $t_1$ &  			& 	$t_2$		& 	& $t_3$	& \hspace{1cm} $+\infty$ \\[5pt] \hline\hline
$\pmb{p}$ & $0$	& $\nearrow$	& $\displaystyle\frac{f}{d}$ &	\IncDown	& $\rightarrow$	&  \DecDown	& 0 	& $\rightarrow$	\\[5pt] \hline
$\pmb{c}$ & $0$		& $\rightarrow$	&  0 & \IncUp & 	 $\displaystyle \frac{b}{a}$	& \IncUp 	& 	& \IncUp	\\[5pt]
\end{tabular}
\end{center}
\end{graybox}



Observe that to do this analysis, we didn't need to know how to solve the system of ODEs. \\






\begin{center}
\textbf{\color{cyan}
Finding the equilibrium point(s)
}
\end{center}

%\paragraph{Finding the equilibrium point.} 
This is often easy to find, and by using the intuition we gained while learning to sketch phase portraits, this can give us a lot of insight about the solutions.

Let us find the equilibrium point:
$$
\begin{cases}
0= p'(t) = -a c(t) + b \\
0= c'(t) = d p(t) + e c(t) - f	
\end{cases}
\quad \Leftrightarrow \quad 
	\begin{cases}
 	\displaystyle c(t) = \frac{b}{a} \\[5pt]
	\displaystyle p(t) = \frac{af - be}{ad}
	\end{cases}
$$

Observe that if the population and cost of living are at these levels, then they will remain constant. \\

This also informs us that the disastrous scenario on the first analysis, where the population all left the city, might have been caused by the stating position. \\






\begin{center}
\textbf{\color{cyan}
Interpreting the phase portrait
}
\end{center}

We have seen in the last module how to sketch a phase portrait for a system of ODEs such as this one.

Let us assume that the constants $a=b=d=e=1$, $f=2$. Then the phase portrait is
\begin{center}
	\includegraphics*[width=250pt]{images/module18-pc.pdf}
\end{center}

Observe that both quantities should be positive, so let us focus on the quadrant where both are positive:
\begin{center}
	\includegraphics*[width=250pt]{images/module18-pc-closeup.pdf}
\end{center}

Observations:
\begin{itemize}
	\item Whenever a graph hits the axes, we must stop the model and re-evaluate what that means: when the graph hits the $c=0$ axis, then the population will start increasing until $c'>0 \Leftrightarrow p = 2$, so the graph will continue horizontally until the point $(2,0)$ where the model restarts.
	\item The population seems to converge to $0$ in all cases.
	\item The purple case seems to be an interesting one where the population and cost of living oscillate for a bit near the equilibrium and then the cost of living goes almost to zero before starting to go up again and forcing all the people to leave the city.
\end{itemize}



\hfill

\begin{center}
\textbf{\color{cyan}
Properties of the system
%Other questions about the system
}
\end{center}

%\paragraph{Other questions about the system.} 
We can look for other properties of the system of ODEs.

Based on the two analyses above, we can ask the following question:
\begin{itemize}
	\item Is there a value for the cost of living such that if it is above that, then eventually all the population will leave the city?
\end{itemize}

We know that 
$$
p'(t) = -a c(t) + b < 0 \quad \Leftrightarrow \quad c(t) > \frac{b}{a}.
$$

So as long as the cost of living is above $\frac{b}{a}$, then the population will continue to decrease. \\

Observe that depending on the constants $d, e, f$, we could still have
$$
c'(t) = d p(t) + e \frac{b}{a} - f < 0,
$$
so that we could end up with a cycle around the equilibrium we found before.








