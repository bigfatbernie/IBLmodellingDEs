%%%%%%%%%%%%%%%%%%%%%%%%%%%%%%%%%%%%%%%%%%%%%%%%%%%%%%%%%%%%%%%%%%%%%%%%
%
%		Chapter 3 - System Models
%
%%%%%%%%%%%%%%%%%%%%%%%%%%%%%%%%%%%%%%%%%%%%%%%%%%%%%%%%%%%%%%%%%%%%%%%%


\begin{topic}[Models of Systems]



\vfil

\begin{center}
\begin{minipage}{300pt}
	\includegraphics*[width=300pt]{images/chap3-xkcd.png}

	\hfill {\footnotesize (image from \href{https://www.xkcd.com/2063/}{xkcd - comic \#2063})}
\end{minipage}
\end{center}


\end{topic}






%%%%%%%%%%%%%%%%%%%%%%%%%%%%%%
%
%  MODULE - Modelling Two Interconnected Quantities
%
%%%%%%%%%%%%%%%%%%%%%%%%%%%%%%



\begin{module}{Modelling Two Quantities}
	\label{sys:model}

	\input{modules/module16-sys-model.tex}
	\begin{exercises}

	\begin{problist}
	
	\prob Create a model for two cooperating populations, like sharks and remoras. 
	
	
%	\begin{center}
%		\includegraphics*[height=150pt]{images/module15-spring-mass-dashpot.pdf}
%	\end{center}
	
	
	\begin{minipage}{.35\textwidth}
		\prob We have a spring attached to a mass and with a dashpot. 
		\begin{enumerate}
			\item Model the position of the mass as time changes.
			\item Obtain a system of two first-order ODEs. Remember to explain how the new functions relate to the spring-mass-dashpot system.
		\end{enumerate}
	\end{minipage}
	\hfill
	\begin{minipage}{100pt}
		\includegraphics*[height=100pt]{images/module16-spring-mass-dashpot.pdf}
	\end{minipage}

	\prob Model a vehicle with a special engine that provides an acceleration to the car proportional to the fuel left.
	
	\prob Imagine two twin babies and model their crying volume. Assume that they naturally become tired and stop crying if alone, but they cry more if the other twin is crying.
	
	
	
	\begin{minipage}{.35\textwidth}
		\prob Create a simplified model for a tree, considering the height of the tree and its leaf area and how they affect each other.	
	\end{minipage}
	\hfill
	\begin{minipage}{100pt}
		\includegraphics*[height=100pt]{images/module16-tree.pdf}
	\end{minipage}

	\prob Create a model on how a student's confidence in her own ability affects her learning/knowledge of a subject. Remember the Ebbinghaus' ``forgetting curve''.

	\begin{center}
		\includegraphics*[width=125pt]{images/module16-UofT2YYZ.pdf}	
	\end{center}

	\prob Imagine that there are two ways to travel from UofT to Toronto's Pearson airport (YYZ). Both paths take the same time if there is no traffic. You want to direct people on the fastest path. Create a model for choosing the fastest path.
	
	
	\prob Create a model for the sales of a specific brand of sneakers. The goal is to capture the influence of famous people and non-famous people on each other's purchases.
	
	\prob Create a model on how the population and the cost of living in Toronto affect each other.
	
	
	\end{problist}
\end{exercises}

\end{module}



\begin{lesson}
	\Title{Modelling Two Quantities}

	\Heading{Objectives}
	\begin{itemize}
		\item Bla
	\end{itemize}
	
	\Heading{Motivation} 

\end{lesson}




\question We want to model two competing populations, like cheetahs and lions: they don't hunt each other, but they hunt the same prey.
\begin{parts}
	\item Create a model for these two populations.
	\item Using Desmos or WolframAlpha, create a slope field in the plane where the horizontal axis is one population and the vertical one is the other.
	\item Using the slope field, deduce some properties of your model and discuss how closely it matches what you expect from these populations.

	\item Extend the model to include a population of antelopes.	
\end{parts}
\begin{annotation}
	\begin{goals}
		Stress that students should follow the step-by-step approach from chapter 1.
		
		\emph{.4} only if there is time. Tell the students to ``go nuts'' and include everything that relates.
	\end{goals}
\end{annotation}





\bookonlynewpage


\question
	A cheetah is chasing an antelope. We want a model of their positions as they run.
	
	
\begin{annotation}
	\begin{goals}
		This exercise is not required to do in lecture. \\
		
		Be careful with assumptions! A very general model will be very hard to study. \\
		
		Allow some brainstorming and try to create a structure for this problem:
		\begin{itemize}
			\item Positions seen from above ($xy-$plane).
			\item Only need $x_a(t), y_a(t)$ and $x_c(t), y_c(t)$
			\item Focus on the cheetah: where is she heading to?
			\item For the antelope, students need to come up with an escape strategy
			\item Model will be nonlinear!
		\end{itemize}
	\end{goals}
\end{annotation}




\standardonlynewpage

%%%%%%%%%%%%%%%%%%%%%%%%%%%%%%
%
%  MODULE - Solving Systems
%
%%%%%%%%%%%%%%%%%%%%%%%%%%%%%%



\begin{module}{Systems of two linear ODEs with constant coefficients}
	\label{sys:solve}

	\input{modules/module17-sys-solving.tex}
	\begin{exercises}

	\begin{problist}
	\prob \label{mod17-gensol}Find the general solution of the problem $\vec{r}'(t) = A \vec{r}(t)$ for the following matrices:
	\begin{enumerate}
	\begin{minipage}{.2\textwidth}
		\item $A = \begin{bmatrix} -7 & 6 \\ -9 & 8 \end{bmatrix}$;
		\item $A = \begin{bmatrix} 22 & 24 \\ -15 & -16\end{bmatrix}$;
		\item $A = \begin{bmatrix} 0 & 1 \\ -5 & 0 \end{bmatrix}$;
		\item $A = \begin{bmatrix} 0 & 1 \\ 5 & 0 \end{bmatrix}$;
		\item $A = \begin{bmatrix} 1 & \sqrt{3} \\ \sqrt{3} & -1\end{bmatrix}$;
		\item $A = \begin{bmatrix} 1 & \sqrt{3} \\ -\sqrt{3} & 1\end{bmatrix}$;
	\end{minipage}
	\qquad
	\begin{minipage}{.2\textwidth}
		\item $A = \begin{bmatrix} 0 & 1 \\ -4 & -4 \end{bmatrix}$;
		\item $A = \begin{bmatrix} -4 & -6 \\ 2 & 3 \end{bmatrix}$;
		\item $A = \begin{bmatrix} 2 & -3 \\ 0 & 2 \end{bmatrix}$;
		\item $A = \begin{bmatrix} 2 & 0 \\ 0 & 2 \end{bmatrix}$;
		\item $A = \begin{bmatrix} 0 & 0 \\ 1 & 0\end{bmatrix}$; 
		\item $A = \begin{bmatrix} 0 & 0 \\ 0 & 1\end{bmatrix}$; 
	\end{minipage}
	\end{enumerate}

	\prob For each of the problems in the previous exercise, find the solution that satisfies the initial conditions:
	\begin{enumerate}[label=(\roman*)]
		\item $\vec{r}(0)=\begin{bmatrix} 0 \\ 0 \end{bmatrix}$;
		\item $\vec{r}(0)=\begin{bmatrix} 1 \\ 3 \end{bmatrix}$;
		\item $\vec{r}(1)=\begin{bmatrix} -2 \\ 2 \end{bmatrix}$.
	\end{enumerate}


	\prob Consider the problem $\vec{r}'(t) = \begin{bmatrix} 2 & 0 \\ 1 & 3 \end{bmatrix} \vec{r}(t) + \begin{bmatrix} -2 \\ 11 \end{bmatrix}$.
	\begin{enumerate}
		\item Show that $\vec{e}(t) = \begin{bmatrix} 1 \\ -4 	\end{bmatrix}$ is a solution of this problem.
		\item Find the general solution of 
		$$\vec{u}'(t) = \begin{bmatrix} 2 & 0 \\ 1 & 3 \end{bmatrix} \vec{u}(t).$$
		\item Let $\vec{r}(t) = \vec{u}(t) + \vec{e}(t)$. Show that this is a solution of the original problem.
		\item Let $\vec{r}_1(t)$ and $\vec{r}_2(t)$ be two solutions of the original problem. 
		\begin{enumerate}
			\item Is $\vec{r}_1(t) + \vec{r_2}(t)$ a solution? 
			\item Is $3\vec{r}_1(t) $ a solution? 
			\item Write a result on how one can safely combine solutions of non-homogeneous problems.
		\end{enumerate}
	\end{enumerate}

	
	\prob \label{prob:sys-nonhomogeneous}Consider the problem  \quad $\vec{r}'(t) = \begin{bmatrix} 1 & 2 \\ 3 & 0 \end{bmatrix}
 \vec{r}(t)+\begin{bmatrix} -5 \\ 3 \end{bmatrix}
$.
	\begin{enumerate}
		\item Observe that this system is an autonomous system of ODEs. What is the equilibrium solution? 
		\item Sketch its phase portrait using WolfamAlpha's streamplot function. What can you observe from the phase portrait?
		\item ``Redefine the centre'': Let the equilibrium solution solution you just found be called $\vec{e}$. Consider $\vec{u}(t) = \vec{r}(t) - \vec{e}$, where $\vec{r}(t)$ is the solution of the original problem. Show that 
			$$ \vec{u}'(t) = A \, \vec{u}(t).$$
		\item Find $\vec{u}(t)$.
		\item Find $\vec{r}(t)$.
		\item Write a procedure to solve any problem of the form
			$$ \vec{r}(t) = A \, \vec{r}(t) + \vec{b}. $$
	\end{enumerate}
	
	\prob Consider the problem $
\frac{d \,\vec{r}}{dt} = \begin{bmatrix} 4 & -1 \\ 8 & -2\end{bmatrix} \vec{r} - \begin{bmatrix} 5 \\ 10 \end{bmatrix}$.

	\begin{enumerate}
		\item Find $\vec{r}(t)$.
		\item What is the solution taht satisfies $\vec{r}(0) = \begin{bmatrix} 2\\3 \end{bmatrix}$?
		\item Sketch the phase portrait.
	\end{enumerate}

	
	\prob \label{prob:sys-superposition}Consider the problem \quad $\vec{r}'(t) = A \, \vec{r}(t)$.
	Assume that we have two solutions $\vec{r_1}(t)$ and $\vec{r_2}(t)$.
	\begin{enumerate}
		\item Show that $\vec{r}(t) = \vec{r_1}(t) + \vec{r_2}(t)$ is a solution also.
		\item Show that $\vec{r}(t) = 2\vec{r_1}(t) - 3\vec{r_2}(t)$ is a solution also.
		\item Find all possible solutions of the problem.
	\end{enumerate}
	
	
		\prob Consider the problem $\vec{r}'(t) = \begin{bmatrix} 1 & -2 \\ -2 & 1 \end{bmatrix} \vec{r}(t)$.
		\begin{enumerate}
			\item Find the solution that satisfies the initial condition $\vec{r}(0)=\begin{bmatrix}1 \\ 0\end{bmatrix}$. Call it $\vec{u}(t)$.
			\item Find the solution that satisfies the initial condition $\vec{r}(0)=\begin{bmatrix}0 \\ 1\end{bmatrix}$. Call it $\vec{v}(t)$.
			\item Define the matrix function
			$$ \Phi(t) = \begin{bmatrix} \vec{u}(t) \; | \; \vec{v}(t) \end{bmatrix} = \begin{bmatrix} u_1(t) & v_1(t) \\ u_2(t) & v_2(t) \end{bmatrix}.$$
			
			Show that $\vec{r}(t) = \Phi(t) \vec{r}_0$ is a solution of the original system of ODEs. Which initial condition does it satisfy?
			
			\item Write a result relating $\Phi(t)$ to the solution of initial-value problems.
			\end{enumerate}

	
	
		\prob \label{mod17:prob-W1}Consider a system of ODEs $\vec{r}'(t) = A \vec{r}(t)$ with two solutions $\vec{r}_1(t)$ and $\vec{r}_2(t)$. 
		
		We want to study the conditions that are necessary on the solutions $\vec{r}_1$ and $\vec{r}_2$ to guarantee that we can solve any initial-value problem.
		
		\begin{enumerate}
			\item What is the general solution for this problem?
			\item If the initial condition is $\vec{r}(0)= \begin{bmatrix} 1 \\ 2 \end{bmatrix}$, then what are the conditions on $\vec{r}_1,\vec{r}_2$ ?
			\item If the initial condition is $\vec{r}(0)= \vec{r_0}$, then what are the conditions on $\vec{r}_1,\vec{r}_2$ ?
		\end{enumerate}

		\prob Consider a system of ODEs $\vec{r}'(t) = A \vec{r}(t)$ with two solutions $\vec{r}_1(t), \vec{r}_2(t)$.
		
			Let $R(t)$ be the matrix 
				$R(t) = \begin{bmatrix} \vec{r}_1(t) & | & \vec{r}_2(t)	\end{bmatrix} $ 
				and let $W(t) = \det R(t)$.
		
		\begin{enumerate}
			\item Show that $W(t)$ is a solution of $W' = (a_{11} + a_{22}) W$.
			\item Solve the ODE above to obtain an expression for $W(t)$.
			\item Show that $W(t)$ is either identically zero, or it's never zero. 
			\item Use this result to simplify your answer to problem \ref{mod17:prob-W1}(c).
		\end{enumerate}
	
	\end{problist}
\end{exercises}

\end{module}



\begin{lesson}
	\Title{Systems of two linear ODEs with constant coefficients}

	\Heading{Objectives}
	\begin{itemize}
		\item Bla
	\end{itemize}
	
	\Heading{Motivation} 

\end{lesson}




\question
Consider a cheetah-lion inspired problem:
$$
\frac{d \,\vec{r}}{dt} = \begin{bmatrix} 3 & -2 \\ -1 & 4\end{bmatrix} \vec{r}.
$$
	
\begin{parts}
	\item Find the two solutions $\vec{r}_1, \vec{r}_2$.
	\item Is $\vec{r}_1(t) + \vec{r}_2(t)$ a solution?
	\item Is $\vec{r}_1(t) - \vec{r}_2(t)$ a solution?
	\item Is $2\vec{r}_1(t) + 3\vec{r}_2(t)$ a solution?
	\item What is the general solution?
	\item Find the solution that satisfies $\vec{r}(0) = \begin{bmatrix} 6 \\ 7\end{bmatrix}$?
\end{parts}


\bookonlynewpage


\question
Consider a problem:
$$
\frac{d \,\vec{r}}{dt} = \begin{bmatrix} 2 & -5 \\ 1 & -2\end{bmatrix} \vec{r}.
$$
	
\begin{parts}
	\item Find the general solution.
	\item Find the solution that satisfies $\vec{r}(0) = \begin{bmatrix} 6 \\ 7\end{bmatrix}$?
\end{parts}



\bookonlynewpage

\begin{annotation}
	\begin{goals}
		\Goal{Non-Homogeneous Problem}
%		Introduce this core exercise as a non-homogeneous problem. \\
		
		Students don't know how to solve it yet:
		\begin{enumerate}
			\item Equilibrium solution ($\begin{bmatrix}2\\-1\end{bmatrix}$)
			\item Show phase portrait using WolframAlpha \url{https://uoft.me/modelling-sys-nonhom}
			\item Ask students about properties of the phase portrait (\emph{Goal}: solutions revolve around the equilibrium point)
			\item Redefine centre: $\vec{r} = \vec{\rm eq} + \vec{p}$. What system foes $\vec{p}$ solve?
			\item $\cdots$
		\end{enumerate}
		
		There are practice problems about this.
	\end{goals}
\end{annotation}
\question
Consider a problem:
$$
\frac{d \,\vec{r}}{dt} = \begin{bmatrix} 2 & -5 \\ 1 & -2\end{bmatrix} \vec{r} - \begin{bmatrix} 9 \\ 4 \end{bmatrix}.
$$
%[2,-1]
\begin{parts}
	\item Find the equilibrium solution.
	\item Find the general solution.
	\item Find the solution that satisfies $\vec{r}(0) = \begin{bmatrix} 8 \\ 6 \end{bmatrix}$?
\end{parts}


	





\standardonlynewpage

%%%%%%%%%%%%%%%%%%%%%%%%%%%%%%
%
%  MODULE - Phase Portraits
%
%%%%%%%%%%%%%%%%%%%%%%%%%%%%%%



\begin{module}{Phase Portraits}
	\label{sys:phase}

	\input{modules/module18-sys-phase.tex}
	\begin{exercises}

	\begin{problist}
	\prob For each matrix from practice problem \ref{mod17-gensol} from Module 17, sketch its phase portrait and label them as asymptotically stable or unstable. The system of ODEs is $\vec{r}'(t) = A \vec{r}(t)$ for the following matrices:
	\begin{enumerate}
	\begin{minipage}{.2\textwidth}
		\item $A = \begin{bmatrix} -7 & 6 \\ -9 & 8 \end{bmatrix}$;
		\item $A = \begin{bmatrix} 22 & 24 \\ -15 & -16\end{bmatrix}$;
		\item $A = \begin{bmatrix} 0 & 1 \\ -5 & 0 \end{bmatrix}$
		
		This is called a centre, which is stable, but not asymptotically stable. Can you tell why?
		\item $A = \begin{bmatrix} 0 & 1 \\ 5 & 0 \end{bmatrix}$;
		\item $A = \begin{bmatrix} 1 & \sqrt{3} \\ \sqrt{3} & -1\end{bmatrix}$;
		\item $A = \begin{bmatrix} 1 & \sqrt{3} \\ -\sqrt{3} & 1\end{bmatrix}$;
	\end{minipage}
	\qquad
	\begin{minipage}{.2\textwidth}
		\item $A = \begin{bmatrix} 0 & 1 \\ -4 & -4 \end{bmatrix}$
		
		This is called an improper node.
		\item $A = \begin{bmatrix} -4 & -6 \\ 2 & 3 \end{bmatrix}$;
		\item $A = \begin{bmatrix} 2 & -3 \\ 0 & 2 \end{bmatrix}$;
		\item $A = \begin{bmatrix} 2 & 0 \\ 0 & 2 \end{bmatrix}$;
		
		This is called a proper node.
		\item $A = \begin{bmatrix} 0 & 0 \\ 1 & 0\end{bmatrix}$; 
		\item $A = \begin{bmatrix} 0 & 0 \\ 0 & 1\end{bmatrix}$; 
	\end{minipage}
	\end{enumerate}
	
	\prob Consider a system of ODEs $\vec{r}'(t) = A \vec{r}(t)$.
	For each part, give an example of eigenvalues and eigenvectors of $A$ that would yield the required phase portrait:
	\begin{enumerate}
		\item Spiral sink (asymptotically stable);
		\item Spiral source (unstable);
		\item Centre (stable);
		\item Sink node (asymptotically stable);
		\item Source node (unstable);
		\item Saddle point (unstable);
		\item Improper node (stable);
		\item Improper node (unstable);
		\item Proper node (stable);
		\item Proper node (unstable);
	\end{enumerate}
	
	\prob Consider the system of ODEs
	$$
	\vec{r}'(t) = 
	\begin{bmatrix}
	1 & 1 \\ k & 1
	\end{bmatrix}\vec{r}(t).
	$$

	This system of ODEs changes behaviour depending on the parameter $k$.
	\begin{enumerate}
		\item Label the behaviour of the system for different values of $k$.
		\item We call the $k^\star$ the critical value of $k$ when the behaviour is different for $k<k^\star$ and for $k>k^\star$. For the critical value of $k$, sketch the phase portrait.
	\end{enumerate}
	
	
	\prob Consider the system of ODEs
	$$
	\vec{r}'(t) = 
	\begin{bmatrix}
	0 & 1 \\ -4 & -k
	\end{bmatrix}\vec{r}(t).
	$$

	This system of ODEs changes behaviour depending on the parameter $k$.
	\begin{enumerate}
		\item Label the behaviour of the system for different values of $k$.
		\item We call the $k^\star$ the critical value of $k$ when the behaviour is different for $k<k^\star$ and for $k>k^\star$. For the critical value of $k$, sketch the phase portrait.
		\item Which kinds of behaviours could be critical values?
	\end{enumerate}
	
	
	
	\prob Consider the system of ODEs
	$$
	\vec{r}'(t) = A \vec{r}(t).
	$$
	
	Let $T={\rm trace}(A) = a_{11}+a_{22}$, $D=\det(A) = a_{11}a_{22} - a_{12}a_{21}$, and $\Delta = T^2-4D$.
	
	\begin{enumerate}
		\item Show that the equilibrium solution is a saddle point if $D<0$.
		\item Show that the equilibrium solution is a spiral if $\Delta<0$ and $T\neq 0$.
		\item Show that the equilibrium solution is a centre if $T=0$ and $D>0$.
		\item When is the equilibrium point a node? \\

		\item Show that the equilibrium solution is asymptotically stable if $T<0$ and $D>0$.
		\item Show that the equilibrium solution is unstable if $T>0$ and $D<0$.
	\end{enumerate}






	\end{problist}
\end{exercises}

\end{module}



\begin{lesson}
	\Title{Phase Portraits}

	\Heading{Objectives}
	\begin{itemize}
		\item Bla
	\end{itemize}
	
	\Heading{Motivation} 

\end{lesson}




\begin{annotation}
	\begin{goals}
	\Goal{Unstable Saddle Point}
	At the end, let the students know that the equilibrium is called \emph{saddle point} and it is \emph{unstable}, because solutions go away from it.
	
	For the interpretation question, when one population hits zero, it is extinct, so the graph doesn't make sense. 
	
	We can interpret that if a population becomes extinct, then the other will behave as it would without competitors: grow exponentially fast!
	\end{goals}
\end{annotation}
\question
	Consider the following model for cheetah's and lions, where
	$$ \vec{p}(t) = \begin{bmatrix} \ell(t) = \text{population of lions} \\ c(t) = \text{population of cheetahs} \end{bmatrix} $$
	which satisfies
	$$
	\frac{d\,\vec{p}}{dt} = \begin{bmatrix}
 		1 & -1 \\
 		-3 & 1
 	\end{bmatrix}
	$$
	
	The general solution is:
	$$
	\vec{p}(t) = c_1 \begin{bmatrix} 1 \\ \sqrt{3} \end{bmatrix} e^{(1-\sqrt{3})t} + c_2 \begin{bmatrix} -1 \\ \sqrt{3} \end{bmatrix} e^{(1+\sqrt{3})t}.
	$$
\begin{parts}
	\item Without computing them, what are the eigenvalues and eigenvectors of the matrix?
	\item Sketch the graph of the solution with $c_1=\pm 1$ and $c_2=0$.
	\item Sketch the graph of the solution with $c_1=0$ and $c_2=\pm 1$.
	\item When one constant is set to 0, what is the shape of the graph? Is it always like that? Can you prove it?
	\item Sketch the graph of the solution with $c_1=\pm 1$ and $c_2=\pm 1$.
	\item Provide an interpretation of the different types of solutions.
\end{parts}






\bookonlynewpage

\begin{annotation}
	\begin{goals}
		Get students to compare their results with the previous core exercise.
%
%		Some students might try to sketch everything from scratch.
%		Remind them that the solutions look very similar and they only have to adapt the phase portrait they had before.\\		
	\end{goals}
\end{annotation}
\question
	Let us expand the model from the previous exercise to:
	$$ \vec{p}(t) = \begin{bmatrix} \ell(t) = \text{population of lions} \\ c(t) = \text{population of cheetahs} \end{bmatrix} $$
	which satisfies
	$$
	\frac{d\,\vec{p}}{dt} = \begin{bmatrix}
 		1 & -1 \\
 		-3 & 1
 	\end{bmatrix} \vec{p}
 	+ \begin{bmatrix}
 		- 10 \\ 50
	 \end{bmatrix}.
	$$
	
	The extra term corresponds to the effect of harvesting 10 lions and bringing in 50 cheetahs every year to the reserve. \\
	
	The general solution is:
	$$
	\vec{p}(t) = \begin{bmatrix} 20 \\ 10 \end{bmatrix} +
		c_1 \begin{bmatrix} 1 \\ \sqrt{3} \end{bmatrix} e^{(1-\sqrt{3})t} + c_2 \begin{bmatrix} -1 \\ \sqrt{3} \end{bmatrix} e^{(1+\sqrt{3})t}.
	$$
\begin{parts}
	\item Sketch the phase portrait.
	\item Provide an interpretation of the different types of solutions.
\end{parts}



\bookonlynewpage


\question
	For each of the following general solutions, sketch the phase portrait.
\begin{parts}
	\item $	\vec{r}(t) = c_1 \begin{bmatrix} 2 \\ 1 \end{bmatrix} e^{2t} + c_2 \begin{bmatrix} -1 \\ 1 \end{bmatrix} e^{5t}.$
	\item $	\vec{r}(t) = c_1 \begin{bmatrix} 2 \\ 1 \end{bmatrix} e^{-2t} + c_2 \begin{bmatrix} -1 \\ 1 \end{bmatrix} e^{-5t}.$	
\end{parts}
\begin{annotation}
	\begin{goals}
	At the end, let the students know what these equilibria are called:
	\begin{itemize}
		\item \emph{source} and it is \emph{unstable}, because solutions go away from it.
		\item \emph{sink} and it is \emph{asymptotically stable}, because solutions converge to it.
	\end{itemize}
	
	If there is time, students can think about:
	\begin{itemize}
		\item Given a matrix $A$, which part of $A$ indicates whether the equilibrium is stable / unstable? Which part indicates whether it's a sink/source vs spiral sink/source?
	\end{itemize}
	\end{goals}
\end{annotation}





\standardonlynewpage

%%%%%%%%%%%%%%%%%%%%%%%%%%%%%%
%
%  MODULE - Analysis of Systems
%
%%%%%%%%%%%%%%%%%%%%%%%%%%%%%%



\begin{module}{Analysis of Models with Systems}
	\label{sys:analysis}

	In this module you will learn
\begin{itemize}
	\item different ways to analyze models with several differential equations
\end{itemize}

\hfill \\



In this chapter, we have learned how to create models involving systems of ODEs and how to solve some special types of systems of ODEs.

Once we create a model that involves a system of ODEs, the ultimate goal is not to solve the system of ODEs, but to be able to understand how the situation will proceed.
Solving the system of ODEs is often a large step in that direction, but it is more important to be able to take the solution and knowing how to interpret in light of the original situation.

Sometimes, when we cannot find an explicit formula for the solution, it is still possible to study the system to find some properties and behaviours of the solutions. \\

In this module, we'll study one example using two different methods.


\begin{example}
The goal here is not the modelling but the analysis of the model, so we will quickly explain the model. \\

We are going to model population versus cost of living in Toronto.


Consider the following functions
\begin{itemize}
\item $p(t) = $ Population of Toronto (GTA) in millions at time $t$ in years since the beginning of 2015.
\item $c(t) = $ Cost of living in Toronto (in thousands of dollars) at time $t$.
\item Define a vector $\vec{x}(t) = \begin{bmatrix} p(t) \\ c(t) \end{bmatrix}$.
\end{itemize}

These two factors are related according to the following properties:
\begin{itemize}
\item In the absence of any migration, the population will decrease proportionally to the cost of living (with constant $a$);
\item There are always people moving into Toronto independently of its current population or cost of living (with constant $b$)
\item In the absence of any other factors, the cost of living; increases proportionally to the population (with constant $d$)
\item In the absence of any other factors, the cost of living; increases proportionally to the cost of living due to inflation (with constant $e$);
\item The city is always expanding, so the cost of living is always decreasing independently of its current population or cost of living (with constant $f$).
\end{itemize}
The constants $a,b,d,e,f$ are all positive. \\

Our model is:
$$
\vec{x}'(t) = 
	\begin{bmatrix}
 		0 & -a \\
 		d & e
	\end{bmatrix}
					\vec{x}(t) + 
	\begin{bmatrix}
		b \\ -f
	\end{bmatrix}
$$
\end{example}

\hfill

\begin{center}
\textbf{\color{cyan}
Qualitative evolution of quantities
}
\end{center}


%\paragraph{Qualitative evolution of quantities.}
We can try to figure out how these quantities, $p(t)$ and $c(t)$, are going to increase or decrease. \\

As an academic example, let us imagine that initially  \quad $p(0)=c(0)=0$.

Then, at $t=0$, we have
$$
p'(0)= b > 0 \quad \text{ and } \quad
c'(0)=-f < 0.
$$

This means that $p(t)$ is increasing while $c(t)$ wants to decrease.

Here we need to make sure that everything still makes sense: since it doesn't make sense to have a negative cost of living (government incentives to move into the city?!), we need to disregard our system and assume that $c(t)$ will continue constant while $c'(t)<0$.

We then have:
\begin{graybox}
\begin{center}
\begin{tabular}{c||c|c|c|c|c|c|c|c}
$\pmb{t}$	& $0$ 		& 			& &  			& 			& 	&	& \hspace{1cm} $+\infty$ \\[5pt] \hline\hline
$\pmb{p}$ & $0$	& $\nearrow$	& \hspace{0.5cm} &	\hspace{1cm}	&	&  \hspace{0.5cm}	&  	& 	\\[5pt] \hline
$\pmb{c}$ & $0$		& $\rightarrow$	&  0 & \hspace{1cm} & 	\hspace{0.5cm}	& \hspace{1cm} 	& \hspace{0.5cm}	&\hspace{1cm}	\\[5pt]
\end{tabular}
\end{center}
\end{graybox}

While $c(t)=0$, we have
$$
p'(t)= b > 0  \quad \text{ and } \quad 
c'(t)=d\, p(t) -f.
$$

This means that $p(t)$ is increasing with constant slope (linearly) until $c'(t_1)=0$.
We can figure out when this will happen:
$$
0=c'(t_1)=d\, p(t_1) -f \quad \Leftrightarrow \quad p(t_1) = \frac{f}{d}.
$$

So we continue our table:
\begin{graybox}
\begin{center}
\begin{tabular}{c||c|c|c|c|c|c|c|c}
$\pmb{t}$	& $0$ 		& 			& $t_1$ &  			& 			& 	&	& \hspace{1cm} $+\infty$ \\[5pt] \hline\hline
$\pmb{p}$ & $0$	& $\nearrow$	& $\displaystyle\frac{f}{d}$ &	\hspace{1cm}	&	&  \hspace{0.5cm}	&  	& 	\\[5pt] \hline
$\pmb{c}$ & $0$		& $\rightarrow$	&  0 & \hspace{1cm} & 	\hspace{0.5cm}	& \hspace{1cm} 	& \hspace{0.5cm}	&\hspace{1cm}	\\[5pt]
\end{tabular}
\end{center}
\end{graybox}

What happens after $t_1$?

Consider $t>t_1$ slightly after $t_1$. Then
$$
\begin{cases}
p'(t) = -a c(t) + b >0 & \text{ still positive because $c(t)$ is very small, but the slope is decreasing} \\
c'(t) = d p(t) + e c(t) - f	>0 & \text{ increasing quickly as both $p$ and $c$ increase}
\end{cases}
$$

\begin{graybox}
\begin{center}
\begin{tabular}{c||c|c|c|c|c|c|c|c}
$\pmb{t}$	& $0$ 		& 			& $t_1$ &  			& 			& 	&	& \hspace{1cm} $+\infty$ \\[5pt] \hline\hline
$\pmb{p}$ & $0$	& $\nearrow$	& $\displaystyle\frac{f}{d}$ &	\IncDown	&	&  \hspace{1cm}	&  	& 	\\[5pt] \hline
$\pmb{c}$ & $0$		& $\rightarrow$	&  0 & \IncUp & 	\hspace{0.5cm}	& \hspace{1cm} 	& \hspace{0.5cm}	&\hspace{1cm}	\\[5pt]
\end{tabular}
\end{center}
\end{graybox}

At a certain time $t_2$, the population will stop increasing. Let us find out when this happens:
$$
0=p'(t_2) = -a c(t_2) + b >0 
	\quad \Leftrightarrow \quad c(t_2) = \frac{b}{a}.
$$

\begin{graybox}
\begin{center}
\begin{tabular}{c||c|c|c|c|c|c|c|c}
$\pmb{t}$	& $0$ 		& 			& $t_1$ &  			& 	$t_2$		& 	&	& \hspace{1cm} $+\infty$ \\[5pt] \hline\hline
$\pmb{p}$ & $0$	& $\nearrow$	& $\displaystyle\frac{f}{d}$ &	\IncDown	& $\rightarrow$	&  \hspace{1cm}	&  	& 	\\[5pt] \hline
$\pmb{c}$ & $0$		& $\rightarrow$	&  0 & \IncUp & 	 $\displaystyle \frac{b}{a}$	& \hspace{1cm} 	& \hspace{0.5cm}	&\hspace{1cm}	\\[5pt]
\end{tabular}
\end{center}
\end{graybox}

After this point we have $t>t_2$ slightly after $t_2$:
$$
\begin{cases}
p'(t) = -a c(t) + b <0 & \text{ decreasing rapidly while $c'(t)>0$}\\
c'(t) = d p(t) + e c(t) - f	>0 & \text{ still increasing quickly until $p(t)=0$}
\end{cases}
$$

We expect that at some point $p(t_3)=0$. From that point on we have
$$
\begin{cases}
p'(t_3) = -a c(t_3) + b <0 & \text{ we need to ignore the model at this point and keep $p$ constant}\\
c'(t_3) = e c(t_3) - f	>0 & \text{ still increasing exponentially}
\end{cases}
$$

So this is our final table:
\begin{graybox}
\begin{center}
\begin{tabular}{c||c|c|c|c|c|c|c|c}
$\pmb{t}$	& $0$ 		& 			& $t_1$ &  			& 	$t_2$		& 	& $t_3$	& \hspace{1cm} $+\infty$ \\[5pt] \hline\hline
$\pmb{p}$ & $0$	& $\nearrow$	& $\displaystyle\frac{f}{d}$ &	\IncDown	& $\rightarrow$	&  \DecDown	& 0 	& $\rightarrow$	\\[5pt] \hline
$\pmb{c}$ & $0$		& $\rightarrow$	&  0 & \IncUp & 	 $\displaystyle \frac{b}{a}$	& \IncUp 	& 	& \IncUp	\\[5pt]
\end{tabular}
\end{center}
\end{graybox}



Observe that to do this analysis, we didn't need to know how to solve the system of ODEs. \\






\begin{center}
\textbf{\color{cyan}
Finding the equilibrium point(s)
}
\end{center}

%\paragraph{Finding the equilibrium point.} 
This is often easy to find, and by using the intuition we gained while learning to sketch phase portraits, this can give us a lot of insight about the solutions.

Let us find the equilibrium point:
$$
\begin{cases}
0= p'(t) = -a c(t) + b \\
0= c'(t) = d p(t) + e c(t) - f	
\end{cases}
\quad \Leftrightarrow \quad 
	\begin{cases}
 	\displaystyle c(t) = \frac{b}{a} \\[5pt]
	\displaystyle p(t) = \frac{af - be}{ad}
	\end{cases}
$$

Observe that if the population and cost of living are at these levels, then they will remain constant. \\

This also informs us that the disastrous scenario on the first analysis, where the population all left the city, might have been caused by the stating position. \\


\begin{center}
\textbf{\color{cyan}
Properties of the system
%Other questions about the system
}
\end{center}

%\paragraph{Other questions about the system.} 
We can look for other properties of the system of ODEs.

Based on the two analyses above, we can ask the following question:
\begin{itemize}
	\item Is there a value for the cost of living such that if it is above that, then eventually all the population will leave the city?
\end{itemize}

We know that 
$$
p'(t) = -a c(t) + b < 0 \quad \Leftrightarrow \quad c(t) > \frac{b}{a}.
$$

So as long as the cost of living is above $\frac{b}{a}$, then the population will continue to decrease. \\

Observe that depending on the constants $d, e, f$, we could still have
$$
c'(t) = d p(t) + e \frac{b}{a} - f < 0,
$$
so that we could end up with a cycle around the equilibrium we found before.









	\newpage 

\begin{exercises}

	\begin{problist}
	\prob Consider the model for student learning:
	\begin{itemize}
		\item $\vec{x} = \begin{bmatrix} x_1 \\ x_2 \end{bmatrix}$
		\item $x_1=$ student confidence in his/her own abilities ($x_1 \in [0,1]$)
		\item $x_2=$ student knowledge measured in IQ past 100
		\item $\vec{x}'(t) =
		\begin{bmatrix}
			a & b \\
			c & - d 
		\end{bmatrix}
		\vec{x}(t)
		+ \begin{bmatrix}
 			-e \\ 0
 		\end{bmatrix}$
		\item Constants $a,b,c,d,e>0$.
	\end{itemize}
	\begin{enumerate}
		\item What is the equilibrium solution $\vec{x}_e$? 
		\item If tests are harder, then $d$ is larger. How does that affect the equilibrium confidence and knowledge of students?
		\item Is the equilibrium solution stable?
		\item Assume $a=1, b=c=2, d=3,e=0$. As $t \to +\infty$, what are the possible outcomes for $\vec{x}(t)$ Explain the meaning for the students.
		\item Assume $a=1, b=c=2, d=3,e=0$. Some solutions satisfy $\displaystyle \lim_{t \to +\infty} \begin{bmatrix}c(t) \\ k(t) \end{bmatrix} = \begin{bmatrix} + \infty \\ + \infty \end{bmatrix}$.

			Show on a graph which initial conditions $\vec{x}(0) = \begin{bmatrix}c(0) \\ k(0) \end{bmatrix}$ guarantee this limit?

		\item If the tests become harder, i.e., $d$ increases, then is that good or bad for students?.
	\end{enumerate}
	
	\prob Consider the model for a tree:
	\begin{itemize}
		\item $\vec{x}(t) = \begin{bmatrix} \ell(t) \\ h(t) \end{bmatrix}$
		\item $\ell(t)=$ area of leafs on the tree
		\item $h(t)=$ height of the tree
		\item $\vec{x}'(t) =
		\begin{bmatrix}
			a & -b \\
			c & -d 
		\end{bmatrix}
		\vec{x}(t)$
		\item Constants $a,b,c,d>0$.
	\end{itemize}
	\begin{enumerate}
		\item Is it possible to have the tree growing taller and taller forever while the leaf area remains bounded?
		\item What would happen to the tree if the area of leafs is proportional to the height squared (not square root)?
		\item If $ad=bc$, explain what happens to the tree as $t\to \infty$.
	\end{enumerate}
	
	
	\prob Consider the model of a car:
	\begin{itemize}
		\item $\vec{c}(t) = \begin{bmatrix} v(t) \\ f(t) \end{bmatrix}$
		\item $v(t)=$ speed of the car
		\item $f(t)=$ amount of fuel in the car's tank
		\item $\vec{c}'(t) =
		\begin{bmatrix}
			-2 & 1 \\
			-2 & 0 
		\end{bmatrix}
		\vec{c}(t)
		+ \begin{bmatrix}
 			0 \\ -1	
		\end{bmatrix}$
	\end{itemize}
	\begin{enumerate}
		\item What is the equilibrium solution $\vec{c}_{\rm eq}$? What is the meaning of your result?
		\item If the car runs out of fuel at 300 m/s, then describe what happens to the car. 
		\item Describe what happens to the car when it starts at rest with a full tank of $300$ L.
		\item If the car attains its maximum velocity when there are still $300$ L of fuel left, what was the car's maximum velocity?


	\end{enumerate}


	\prob Consider the model for crying babies:
	\begin{itemize}
		\item $\vec{c}(t) = \begin{bmatrix} a(t) \\ b(t) \end{bmatrix}$
		\item $a(t)=$ volume of baby A's cries in dB
		\item $b(t)=$ volume of baby B's cries in dB
		\item $\vec{c}'(t) =
		\begin{bmatrix}
			-\alpha & \beta \\
			\beta & -\alpha 
		\end{bmatrix}
		\vec{c}(t)$
		\item Constants $\alpha,\beta>0$.
	\end{itemize}

	The constants $\alpha$ and $\beta$ are $1$ and $2$. Does it make a difference which is 1 and which is 2?


	
		
	\end{problist}
\end{exercises}
\end{module}



\begin{lesson}
	\Title{Analysis of Models with Systems}
%
%	\Heading{Objectives}
%	\begin{itemize}
%		\item Bla
%	\end{itemize}
%	
%	\Heading{Motivation} 

\end{lesson}




\question
	Consider the following model for the sales from a designer clothing brand:
	\begin{itemize}
	\item $x_1(t) = $ purchases by ``common mortals'' (CM) at time $t$ in years since the beginning of 2015.
	\item $x_2(t) = $ purchases by ``famous people'' (FP) at time $t$.
	\end{itemize}
	
	Our model is based on the following two principles:
	\begin{enumerate}[label={($P_{\arabic*}$)}]
		\item CM will buy more items from the brand when CM or FP buy more.
		\item FP will buy less when CM buy them, but will buy more when FP buy it.
	\end{enumerate}

	The model we considered is:
	$$
	\vec{x}'(t) = 
	\begin{bmatrix}
 		a & b \\
 		-c & d
	\end{bmatrix}
	\vec{x}(t)
	$$
	
	\begin{parts}
		\item Suppose that at the beginning only CM buy this brand. Describe how $x_1(t)$ and $x_2(t)$ evolve as $t>0$.


		\item Suppose that at the beginning only FP buy this brand. Describe how $x_1(t)$ and $x_2(t)$ evolve as $t>0$.


		\item What conditions on the constants $a,b,c,d$ will guarantee that the solutions will spiral? In that case, is it a spiral source or spiral sink? Is it clockwise or counterclockwise?
		\item Are there constants $a,b,c,d>0$, such that the solution $\vec{x}$ is periodic?
		\item Consider the constants $a=b=c=d=1$. Assume that initially CM were buying $c_0>0$ items and FP were buying $f_0>0$ items.
			What will happen to $x_1(t)$ and $x_2(t)$ as $t \to \infty$? Explain the results in terms of the evolution of purchases from CM and FP.
\begin{annotation}
	\begin{goals}
		Can leave .6 as a practice problem if there isn't enough time.
	\end{goals}
\end{annotation}
		\item Consider the constants $a=b=c=d=1$.  If $c_0=10$, $f_0=100$, then at what time will FP stop buying items? And at what time will FP be buying the maximum number of items?
	
	\end{parts}




\bookonlynewpage


\hfill

\bookonlynewpage
\standardonlynewpage
