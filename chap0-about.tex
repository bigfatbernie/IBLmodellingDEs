\begin{center}
{\huge\bf Inquiry Based Modelling with Differential and Difference Equations}\\

\vspace{.7in}
{
\it \copyright\,Galv\~ao-Sousa-Siefken, 2019--2020 \\
Creative Commons By-Attribution Share-Alike\, \makebox(30,5){\includegraphics[height=1.2em]{by-sa.pdf}}
}
\end{center}


\subsection*{For the student}

This book is your introductory guide to mathematical modelling and to differential and difference equations. It is divided into
\emph{modules}, and each module is further divided into \emph{exposition},
\emph{practice problems}, and \emph{core exercises}.

The \emph{exposition} is easy to find---it's the text that starts each
module and explains the big ideas of modelling and differential or difference equations. The \emph{practice
problems} immediately follow the exposition and are there so you can
practice with concepts you've learned.  Following the practice problems
are the \emph{core exercises}. The core exercises build up, through
examples, the concepts discussed in the exposition.

To optimally learn from this text, you should:
\begin{itemize}
	\item Start each module by reading through the \emph{exposition} to get familiar with the main ideas. In most modules, there are some videos to help you further understand these ideas, you should watch them after reading through the exposition.

	\item Work through the \emph{core exercises} to develop an understanding and intuition behind the main ideas and their subtleties.

	\item Re-read the \emph{exposition} and identify which concepts each core exercise connects with.

	\item Work through the \emph{practice problems}. These will serve as a check on whether you've understood the main ideas well enough to apply them.
\end{itemize}

{\bf The core exercises.} Most (but not all) core exercises will be
worked through during lecture time, and there is space for you to work
provided after each of the core exercises. 
The point of the core exercises is to develop the main ideas of modelling and differential or difference equations by exploring examples. When working on core exercises, think
``it's the journey that matters not the destination''. The
answers are not the point! If you're struggling, keep with it. The
concepts you struggle though you remember well, and if you look up the
answer, you're likely to forget just a few minutes later. 


{\bf Contributing to the book.} Did you find an error? Do you
have a better way to explain a linear algebra concept? Please,
contribute to this book!  This book is open-source, and we welcome
contributions and improvements. To contribute to/fix part of
this book, make a \emph{Pull Request} or open an \emph{Issue} at
\url{https://github.com/bigfatbernie/IBLModellingDEs}. If you contribute,
you'll get your name added to the contributor list.


\subsection*{For the instructor}

This book is designed for a one-semester introductory modelling course focusing on differential and difference equations (MAT231 at the University of Toronto). 


Each module contains exposition about a subject, practice problems (for students to work on by themselves), and core exercises (for students to work on with your guidance). Modules group related concepts, but the modules have been designed to facilitate learning modelling rather than to serve as a reference. 

{\bf Using the book.} This book has been designed for use in large 
active-learning classrooms driven by a \emph{think, pair-share}/small-group-discussion format.
Specifically, the \emph{core exercises} (these are the problems which aren't labeled ``Practice Problems'' and for which space is provided to write answers) are designed for use during class time.

A typical class day looks like:
\begin{enumerate}
	\item {\bf Student pre-reading.} Before class, students will read through the relevant module.

	\item {\bf Introduction by instructor.} This may involve giving 	a broader context for the day's topics, or answering questions.

	\item {\bf Students work on problems.} Students work individually or in pairs/small groups
		on the prescribed core exercise. During this time the instructor moves
		around the room addressing questions that students may have and giving
		one-on-one coaching.

	\item {\bf Instructor intervention.} When most students have successfully solved
		the problem, the instructor refocuses the class by providing an
		explanation or soliciting explanations from students.
		This is also time for the instructor to ensure that everyone has
		understood the main point of the exercise (since it is sometimes
		easy to miss the point!).

		If students are having trouble, the instructor can give hints
		and additional guidance to ensure students' struggle is productive.

	\item {\bf Repeat step 3.}
\end{enumerate}

Using this format, students are thinking (and happily so) most of the class. Further,
after struggling with a question, students are especially primed to hear the insights of the instructor.

{\bf Conceptual lean.}
The \emph{core exercises} are geared towards concepts instead of computation, though some core exercises focus on simple computation. They also have a modelling lean. 
Learning algorithms for solving differential and difference equations is devalued to make room for modelling and analysis of equations and solutions. \\

Specifically lacking are exercises focusing on the mechanical skills of algorithmic solving of differential and difference equations. Students must practice these skills, but they require little instructor intervention and so can be learned outside of lecture (which is why core exercises don't focus on these skills).

{\bf How to prepare.}
Running an active-learning classroom is less scripted than lecturing.
The largest challenges are: (i) understanding where students are at, (ii) figuring out what to do given the current understanding of the students, and (iii) timing.

To prepare for a class day, you should:
\begin{enumerate}
	\item {\bf Strategize about learning objectives.} Figure out what the point of the day's lesson is and brain storm some examples that would illustrate that point.
	\item {\bf Work through the core exercises.} 
	%	By working through the exercises yourself, you
	%	will be ready to build off student reasoning, and better able to direct a class towards
	%	the important ideas\footnote{ The content of linear algebra is fairly non-linear. One of the hardest parts
	%	of teaching linear algebra is coming up with an explanation that only depends on ideas that have already been taught.}.
	\item {\bf Reflect.} Reflect on how each core exercise addresses the day's goals. Compare with the examples you brainstormed and prepare follow-up questions that you can use in class to test for understanding.
	\item {\bf Schedule.} Write timestamps next to each core exercise indicating at what minute you hope to start each exercise. Give more time for the exercises that you judge as foundational, and be prepared to triage. It's appropriate to leave exercises or parts of exercises for homework, but change the order	of exercises at your peril---they really do build on each other.
\end{enumerate}

A typical 50 minute class is enough to get through 1--3 core exercises (depending on the difficulty), and class observations show that class time is split 50/50 between students working and instructor explanations.

\subsection*{License}
Unless otherwise mentioned, pages of this document are licensed under
the Creative Commons By-Attribution Share-Alike License. That means, you are free
to use, copy, and modify this document provided that you provide attribution to the
previous copyright holders and you release your derivative work under the same license.
Full text of the license is at \url{http://creativecommons.org/licenses/by-sa/4.0/}

If you modify this document, you may add your name to the copyright list. Also,
if you think your contributions would be helpful to others, consider making a
pull request, or opening an \emph{issue} at \url{https://github.com/bigfatbernie/IBLModellingDEs}

{\bf Incorporated content.}
Content from other sources is reproduced here with permission and retains the Author's copyright. Please see the footnote of each page to verify the copyright.

Included in this text, in chapter 1, are expositions adapted from the handbook ``Math Modeling: Getting Started and Getting Solutions'' by K. M. Bliss, K. R. Fowler, and B. J. Gallizzo, published by SIAM in 2014 \url{https://m3challenge.siam.org/resources/modeling-handbook}.

{\bf Contributing.} You can report errors in the book or contribute to the book by filing an \emph{Issue} or a \emph{Pull Request} on the book's GitHub page: \url{https://github.com/bigfatbernie/IBLModellingDEs/}







%\newpage
%
%
%\section*{About the Document}
%
%
%This document is a mix of student resources, student projects, problem sets, and labs. 
%A typical class day looks like:
%\begin{enumerate}
%	\item \textbf{Preparation by students.} Students prepare for lecture by watching a short video and solving a short quiz. 
%
%	\item \textbf{Introduction by instructor.} This may involve giving a broader context for the day's topics, or answering questions.
%
%	\item \textbf{Students work on problems.} Students work individually or in small groups
%		on the prescribed problem. During this time the instructor moves
%		around the room addressing questions that students may have and giving
%		one-on-one coaching.
%
%	\item \textbf{Instructor intervention.} If most students have successfully solved
%		the problem, the instructor regroups the class by providing a concise
%		explanation so that everyone is ready to move to the next concept.
%		This is also time for the instructor to ensure that everyone has
%		understood the main point of the exercise (since it is sometimes
%		easy to do some computation while being oblivious to the larger context).
%
%		If students are having trouble, the instructor can give hints to
%		the group, and additional guidance to ensure the students don't get
%		frustrated to the point of giving up.
%
%	\item \textbf{Repeat step 2.}
%\end{enumerate}
%
%Using this format, students are working (and happily so) most of the class. Further,
%they are especially primed to hear the insights of the instructor, having already
%invested substantially into each problem.
%
%This problem-set is geared towards concepts instead of computation, though some problems
%focus on simple computation.
%
%\begin{annotation}
%	\begin{goals}
%	\Goal{http://creativecommons.org/\\licenses/by-sa/4.0/}
%
%	\hfill \qrcode{http://creativecommons.org/licenses/by-sa/4.0/}	
%	\end{goals}
%\end{annotation}
%
%\section*{License}
%
%Unless otherwise mentioned, pages of this document are licensed under
%the Creative Commons By-Attribution Share-Alike License. That means, you are free
%to use, copy, and modify this document provided that you provide attribution to the
%previous copyright holders and you release your derivative work under the same license.
%Full text of the license is at \url{http://creativecommons.org/licenses/by-sa/4.0/}
%
%\begin{annotation}
%	\begin{goals}
%	\Goal{https://github.com/bigfatbernie/\\IBLmodellingDEs}
%	
%	\hfill \qrcode{https://github.com/bigfatbernie/IBLmodellingDEs}	
%	\end{goals}
%\end{annotation}
%
%If you modify this document, you may add your name to the copyright list. Also,
%if you think your contributions would be helpful to others, consider making a
%pull request, or opening an \emph{issue} at \url{https://github.com/bigfatbernie/IBLmodellingDEs}
%
%Content from other sources is reproduced here with permission and retains the
%Author's copyright. Please see the footnote of each page to verify the
%copyright.
%
%




\section*{Contributors}

% sorting code from
% http://tex.stackexchange.com/questions/121489/alphabetically-display-the-items-in-itemize
\newcommand{\sortitem}[2][\relax]{%
  \DTLnewrow{list}% Create a new entry
  \ifx#1\relax
    \DTLnewdbentry{list}{sortlabel}{#2}% Add entry sortlabel (no optional argument)
  \else
    \DTLnewdbentry{list}{sortlabel}{#1}% Add entry sortlabel (optional argument)
  \fi%
  \DTLnewdbentry{list}{description}{#2}% Add entry description
}
\newenvironment{sortedlist}{%
  \DTLifdbexists{list}{\DTLcleardb{list}}{\DTLnewdb{list}}% Create new/discard old list
}{%
  \DTLsort{sortlabel}{list}% Sort list
  \begin{itemize*}[label={\color{mypink}$\circ$}]%
    \DTLforeach*{list}{\theDesc=description}{%
      \item \theDesc}% Print each item
  \end{itemize*}%
}

This book is a collaborative effort.  The following people have contributed to its creation:
\begin{quote}
\begin{sortedlist}
	\sortitem[Shujah]{Sarah Shujah}
	\sortitem[Orfano]{Stephanie Orfano}
	\sortitem[Saint-Aubin]{Yvan Saint-Aubin}
	\sortitem[Slaght]{Graeme Slaght}
\end{sortedlist}
{\color{mypink}$\circ$}
\end{quote}




\newpage

\tableofcontents

