\subsection{Linear Algebra Review}
\label{LinAlg}

\begin{center}\textbf{\color{cyan}Algebra of Solving Systems of 2 Linear Equations} 	
\end{center}

We can write a linear system of equations 
\begin{align*}
a_{11} x_1 + a_{12} x_2 &= b_1 \\
a_{21} x_1 + a_{22} x_2 &= b_2
\end{align*}
into matrix form
$$
{\bf A} \vec{x} = \vec{b},
$$
where
$$
{\bf A} 
	= \begin{pmatrix} 
		a_{11} & a_{12} \\
		a_{21} & a_{22}
	\end{pmatrix}
	\quad \text{,} \quad 
\vec{x}  
	= \begin{pmatrix} 
		x_{1} \\
		x_{2} 
	\end{pmatrix}
	\quad \text{ and } \quad 
\vec{b}  
	= \begin{pmatrix} 
		b_{1} \\
		b_{2} 
	\end{pmatrix} .
$$

We can solve a system like this one in several different ways.

\begin{example}
Solve the system
\begin{align*}
3x_1+2x_2 &= 7 \\
2x_1+3x_2 & = 8.
\end{align*}
\end{example}

\begin{definition}[Solution by substitution.] 
We can write
$$
x_2 = \frac{7-3x_1}{2},
$$
and use this on the second equation
$$
2x_1 + 3\frac{7-3x_1}{2} = 8
	\quad \Leftrightarrow \quad 4x_1 + 21-9x_1=16
	\quad \Leftrightarrow \quad -5x_1 =-5
	\quad \Leftrightarrow \quad x_1 =1
$$
Then re-use the first equation we obtained to get $x_2 = 2$.
	
\end{definition}

\begin{definition}[Solution by Cramer's rule]
	
Using the same method of substitution on the general system, we obtain
$$
a_{12} x_2 = b_1 - a_{11} x_1,
$$
and we use this into the second equation (after multiplying by $a_{12}$)
$$
a_{12} a_{21} x_1 = a_{22} b_1 - a_{11}a_{22} x_1 = a_{12} b_2
$$
This implies 
$$
x_1 
	= \frac{b_1 a_{22} - a_{12} b_2}{a_{11}a_{22} - a_{12}a_{21}}
	= \frac{
		\begin{vmatrix} 
			b_1 & a_{12} \\
			b_2 & a_{22}
		\end{vmatrix}
		}{
		\begin{vmatrix} 
			a_{11} & a_{12} \\
			a_{21} & a_{22}
		\end{vmatrix}
		}
$$
Then we use this to obtain
$$
x_2
	= \frac{a_{11} b_2  - b_1 a_{21} }{a_{11}a_{22} - a_{12}a_{21}}
	= \frac{
		\begin{vmatrix} 
			a_{11} & b_1 \\
			a_{21} & b_2
		\end{vmatrix}
		}{
		\begin{vmatrix} 
			a_{11} & a_{12} \\
			a_{21} & a_{22}
		\end{vmatrix}
		}
$$

\end{definition}

\begin{important}
This implies that there is a unique solution of the system if and only if
$$
\det ({\bf A}) = a_{11}a_{22} - a_{12}a_{21} \neq 0.
$$
\end{important}



\begin{definition}[Solution by inverse matrix]
 A matrix is {\bf invertible} or {\bf nonsingular} \quad iff \quad ${\bf A}^{-1}$ exists \quad iff \quad $\det({\bf A}) \neq 0$.

If the matrix {\bf A} is invertible, then we can write
$$
{\bf A}^{-1} 
	= \frac{1}{\det({\bf A})} \begin{pmatrix}
						a_{22} & -a_{12} \\
						-a_{21} & a_{11}
						\end{pmatrix}
$$

We can now use this to solve the system of equations:
\begin{align*}
{\bf A} \vec{x}	& = \vec{b}  \\
{\bf A}^{-1} {\bf A} \vec{x}	& = {\bf A}^{-1} \vec{b}  \\
{\bf I} \vec{x}	& = {\bf A}^{-1} \vec{b}  \\
\vec{x}	& = {\bf A}^{-1} \vec{b}
\end{align*}

\end{definition}





\begin{definition}[Homogeneous Systems]
	
A system of equations is called \emph{homogeneous} if $\vec{x}=\vec{0}$ is a solution, which means that $\vec{b} = 0$:
$$
{\bf A} \vec{x} = \vec{0}.
$$

Otherwise, it is called \emph{nonhomogeneous}.
\end{definition}



\hfil







\begin{center}\textbf{\color{cyan}Eigenvalues and Eigenvectors} 	
\end{center}


We can think of the matrix multiplication $\vec{y}={\bf A}\vec{x}$ as a mapping or transformation: given a vector $\vec{x}$ it transforms it into a different vector $\vec{y}$.

In many applications, it is important to know which vectors $\vec{x}$ are transformed into multiples of themselves.

These vectors satisfy the property
$$
{\bf A}\vec{x} = \lambda \vec{x} \qquad \Leftrightarrow \qquad ({\bf A} - \lambda {\bf I} ) \vec{x} = \vec{0}.
$$

One such vector is $\vec{x}=\vec{0}$. But that's not very interesting. We want to look for nonzero vectors that satisfy this property.

These vectors are called {\bf eigenvectors} and the corresponding $\lambda$ is called an {\bf eigenvalue}.

\begin{important}
The second formulation above implies that the matrix $({\bf A} - \lambda {\bf I})$ is singular, otherwise the unique solution would be $\vec{x}=\vec{0}$. 
So that implies that 
\begin{equation}\tag{\bf characteristic equation}
\det({\bf A} - \lambda {\bf I}) = 0
\end{equation}
\end{important}



\begin{example}
Let us find the eigenvalues and eigenvectors for
$$
{\bf A} = \begin{pmatrix} 1 & 8 \\ 4 & 5 \end{pmatrix}.
$$

First we solve the characteristic equation:
$$
\det({\bf A} - \lambda {\bf I}) 
	= \begin{vmatrix}
		1 - \lambda & 8 \\	
		4 & 5 - \lambda
	\end{vmatrix}
	= (1-\lambda)(5-\lambda) - 32 = 0
\qquad \Leftrightarrow \qquad
	\lambda^2 - 6 \lambda -27=0
$$
which implies that
$$
\lambda = 3 \pm \sqrt{9 + 27} = 3 \pm 6
$$


\subparagraph{Eigenvalue $\lambda_1 = -3$. } To find the eigenvector, we write its equation
$$
({\bf A} - \lambda_1 {\bf I} ) \vec{x} = \vec{0}
\qquad \Leftrightarrow \qquad
\begin{pmatrix}
	4 & 8 \\
	4 & 8
\end{pmatrix}
\begin{pmatrix}
x_1 \\ x_2 
\end{pmatrix}
	= \begin{pmatrix}
	0 \\ 0
	\end{pmatrix}
$$
which implies that
$$
4x_1 +8x_2 = 0
\qquad \Leftrightarrow \qquad
x_1 = -2 x_2.
$$
So one eigenvector for this eigenvalue is
$$
\vec{x}_1 = \begin{pmatrix}
	-2 \\ 1
	\end{pmatrix}
$$


\subparagraph{Eigenvalue $\lambda_2 = 9$. } To find the eigenvector, we write its equation
$$
\begin{pmatrix}
	-8  & 8 \\
	4 & -4
\end{pmatrix}
\begin{pmatrix}
x_1 \\ x_2 
\end{pmatrix}
	= \begin{pmatrix}
	0 \\ 0
	\end{pmatrix}
$$
which implies that
$$
-8x_1 +8x_2 = 0
\qquad \Leftrightarrow \qquad
x_1 = x_2.
$$
So one eigenvector for this eigenvalue is
$$
\vec{x}_2 = \begin{pmatrix}
	1 \\ 1
	\end{pmatrix}
$$
\end{example}





\begin{theorem}
Let {\bf A} have real or complex eigenvalues $\lambda_1$ and $\lambda_2$ such that $\lambda_1 \neq \lambda_2$ and let the corresponding eigenvectors be
$$
\vec{x}_1 = \begin{pmatrix} x_{11}  \\ x_{21} \end{pmatrix}
\qquad \text{ and } \qquad
\vec{x}_2 = \begin{pmatrix} x_{12}  \\ x_{22} \end{pmatrix}.
$$
If {\bf X} is the matrix with columns taken from the eigenvectors:
$$
{\bf X} = \begin{pmatrix}
x_{11} & x_{12} \\
x_{21} & x_{22}
\end{pmatrix},
$$
then 
$$\det({\bf X}) \neq 0.$$
\end{theorem}





