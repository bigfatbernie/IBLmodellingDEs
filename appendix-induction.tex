\subsection{Mathematical Induction Review}
\label{app:induction}






Mathematical induction is a very powerful tool for proving results. It allows us to prove generalized results. 
Its shortcoming is that you already have to suspect what the solution is. It then allows you to prove it.


\begin{definition}[The Principle of Mathematical Induction]
	
Assume that ${\bf P}(1)$, ${\bf P}(2)$, ${\bf P}(3), \ldots$ is an infinite sequence of mathematical statements. \\

\begin{tabular}{ll|ll}
{\bf If}  & & {\bf If} \\
	& (a) \quad ${\bf P}(1)$ is true, 		& & first domino falls \\
	& 	{\bf and} & & {\bf and} \\
	& (b) \quad  for any $k$, ${\bf P}(k)$ implies ${\bf P}(k+1)$ 	
		& & if a domino falls, then the next one falls \\
{\bf then} & & & {\bf then} \\
	& all the statements in the sequence are true	& & all dominoes fall!
\end{tabular}

\end{definition}


\begin{example}
For any $n\in \N$, $1+3+5+\cdots+(2n-1) = n^2$. \\

Let us prove this formula using Mathematical Induction.

\paragraph{Proof.}
If $n=1$, we get $1=1^2$, which is true -- ${\bf P}(1)$ holds.

Assume that $1+3+5+\cdots+(2k-1) = k^2$ for some $k$.
Then
\begin{align*}
1+3+5+\cdots+\big(2(k+1)-1\big) 
	& = \underbrace{1+3+5+\cdots+(2k-1)}_{= k^2 \text{ by induction hypothesis}} + (2k+1) \\
	& = k^2+(2k+1)\\
	& = k^2 + 2k + 1\\
	& = (k+1)^2.
\end{align*}
By induction, the equality holds for all $n\in \N$.

\end{example}


\begin{example}
	
For any $x\in \R$ with $x\geq -1$ and $n \in \N$, then $(1+x)^n \geq 1+nx$. \hfill (Bernoulli's Inequality)

\paragraph{Proof.}
For $n=1$, $1+x\geq 1+x$, which is true!

Assume that $(1+x)^k \geq 1+kx$. Then
\begin{align*}
(1+x)^{k+1} 
	& = (1+x)^k (1+x) 
		\geq (1+kx)(1+x) \\
	& = 1 + x + kx^2 + kx 
		= 1+(k+1)x+kx^2 \\
	& \geq 1+(k+1)x.
\end{align*}
By induction, the claim holds for all $n \in \N$.

\begin{graybox}
	
In the proof, we didn't use the fact that $x\geq -1$, but 
for $n=3$ and $x=-4$, $(1+x)^n = -27 \not\geq -11 = 1+nx$.  \\

Where did we use the hypothesis $x\geq -1$?
\end{graybox}

\end{example}






%%%%%%%%%%

\newpage

\begin{definition}[Principle of Strong Mathematical Induction]

Assume that ${\bf P}(1)$, ${\bf P}(2)$, ${\bf P}(3), \ldots$ is an infinite sequence of mathematical statements. \\

\begin{tabular}{lll|ll}
{\bf If} & & & {\bf If} \\
	& (a) & ${\bf P}(1)$ is true, 		 & & first domino falls \\
	&  {\bf and} & & & {\bf and} \\
	& (b) &  For any $k$, ${\bf P}(1), \ldots, {\bf P}(k)$ implies ${\bf P}(k+1)$ 	& & if all previous dominos fell, \\
	& & & & then the next one falls \\
{\bf then} & & & {\bf then} \\
	& \multicolumn{2}{l|}{all the statements in the sequence are true} & & all dominoes fall!
\end{tabular}

\end{definition}


\begin{example}
\begin{theorem}
Every number $n\in\N, n\geq 2$ can be written as a product of primes (or is a prime).
\end{theorem}

\paragraph{Proof.}
Base case: $n=2$ is a prime.

Assume that the Theorem holds for $n=2,3,4,\ldots, k$ and consider $n=k+1$.

If $k+1$ is prime, the claim holds.

If $k+1$ is not a prime, then it is divisible by some $2 \leq m \leq k$: $k+1=m\cdot \ell$ for some $2 \leq m,\ell\leq k$.
By hypothesis, both $m$ and $\ell$ are products of primes, hence so is $k+1$.
The Theorem follows by strong induction.
\end{example}




%\begin{ex} How many cuts are needed to cut a chocolate bar with $n$ squares into $1\times 1$ pieces?
%\end{ex}
%
%\paragraph{Claim: } $n-1$ cuts are needed to separate $n$ squares. \\
%
%\begin{minipage}{12cm}
%\begin{proof}
%For $n=1$, no cuts are needed, so the claim holds.
%
%Assume that the claim holds for $n=1,2,3,\ldots,k$ and consider a bar with $k+1$ squares.
%Perform one cut to obtain smaller bars with $m$ and $\ell$ squares (so $k+1=m+\ell$).
%By assumption, $m-1$ and $\ell-1$ are needed to cut the smaller bars, so to cut the original bar, we need
%$$
%1 + m-2+\ell-1 = m + \ell -1 = k
%$$
%The claim holds for $n=k+1$, so by strong induction it holds for all $n\in\N$.
%\end{proof}
%\end{minipage}
%\qquad
%\begin{minipage}{125pt}
%\includegraphics*[width=100pt]{figures/chocobar.pdf}
%\end{minipage}
%
%
%
%\begin{thm}
%Every $n\in \N$ can be written as a sum of distinct nonnegative integer powers of $2$.
%\end{thm}
%
%\begin{ex}
%\begin{align*}
%17 & = 2^0 + 2^4 \\
%42 & = 2^1 + 2^3+ 2^5 \\
%65 & = 2^0 + 2^6
%\end{align*}
%\end{ex}
%
%\begin{proof}
%The Theorem holds for $n=1$, since $1 = 2^0$.
%
%Assume that it works for $n=1,2,\ldots, k$ and consider $n=k+1$.
%
%If $k$ is even then by assumption $k=2^{a_1} + 2^{a_2} + \cdots + 2^{a_\ell}$ for $0<a_1<a_2<\cdots<a_\ell$ and 
%$$
%k+1 = 2^0 + 2^{a_1} + 2^{a_2} + \cdots + 2^{a_\ell}
%$$
%is the required representation.
%
%If $k$ is odd, then $k+1$ is even, hence $k+1=2 m$ for some $m\in \N$. By assumption, $m=2^{a_1} + 2^{a_2} + \cdots + 2^{a_\ell}$ for some $0\leq a_1<a_2<\cdots<a_\ell$ and hence
%$$
%k+1 = 2^{a_1+1} + 2^{a_2+1} + \cdots + 2^{a_\ell+1}.
%$$
%By strong induction, the Theorem follows.
%\end{proof}
%
%




%%%%%%%%%%%%%%%%%%%%%%%%%%%%%%%%%%%%%%%%%%%%%%%%%%%%%%%%%%%%%
%
%	practice problems
%
%%%%%%%%%%%%%%%%%%%%%%%%%%%%%%%%%%%%%%%%%%%%%%%%%%%%%%%%%%%%%

\begin{exercises}
		% Topics:
		% 
	\begin{problist}
		% 
		\prob Is the triangle inequality true for more than two numbers?
			$$
			|x_1 + x_2 + \cdots + x_n| \stackrel{?}{\leq} |x_1| + |x_2| + \cdots + |x_n| 
			$$
			If it is, prove it.
			
		\prob Is the AGM inequality true for any $x_1, x_2, \ldots, x_n \geq 0$?
			$$
			\sqrt{x_1 x_2 \cdots x_n} \stackrel{?}{\leq}\frac{x_1 + x_2 + \cdots + x_n}{n}
			$$
			If it is, prove it.
			
		\prob Prove that for any $n \in \N$, $2^{6n}+3^{2n-2}$ is divisible by $5$.


		\prob How many subsets does a set $S$ with $n$ elements have (including $S$ and $\emptyset$)?
		
		\prob Show that If $x_1, \ldots, x_n \in [0,1]$ then $\displaystyle \prod_{i=1}^n (1-x_i) \geq 1 - \sum_{i=1}^n x_i$

		\prob Can you use Mathematical induction to prove that $P(m), P(m+1), P(m+2),\ldots$ are true? If so, how?

		\prob Can you use Mathematical induction to prove that $P(2), P(4), P(6), P(8), \ldots$ are true? If so, how?

		\prob Can you use Mathematical induction to prove that $P(1), P(3), P(5), P(7), \ldots$ are true? If so, how?

		\prob For which $n \in \N$, $2^n \geq (n+1)^2$? Prove your answer.

		\prob Show that for even $n$'s, $n(n^2+3n+2)$ is divisible by $24$.

		\begin{center}
			\includegraphics*[width=15pt]{images/app-ind-L-shape.pdf}
		\end{center}
		\prob Prove that for any $n \in \N$, a $2^n \times 2^n$ checkerboard with one single square removed has an L--tiling (i.e., can be covered with L--shapes).
%
%{\it Proof. } 
%For $k=1$, a $2\times 2$ board with one square removed has an L--shape, so it can be covered with a single L.
%
%Assume for $k$ and prove for $k+1$:
%
%\begin{minipage}{10cm}
%Divide the board in 4 $2^k \times 2^k$ boards. One of the smaller boards will have a square missing.
%
%Remove three squares from the centre from each of the other 3 boards - forming an L--shape as in the figure.
%We then have 4 boards with dimensions $2^k \times 2^k$ with one square removed.
%
%By induction hypothesis, we can cover each of these boards with L--shapes and we can cover the middle gap with one L--shape.
%Therefore the $2^{k+1} \times 2^{k+1}$ board has an L--tiling, and the claim follows by induction. \qed
%\end{minipage}
%\quad
%\begin{minipage}{150pt}
%\includegraphics*[width=150pt]{images/app-ind-2k1_board.pdf}
%\end{minipage}

		\prob Read the following proof:
		
		\begin{theorem} All horses have the same colour.
		\end{theorem}
		
		\begin{graybox}
			
		\paragraph{Proof.}
		Assume that the claim holds for groups of $k$ horses, and consider a group with $k+1$ horses $S = \{h_1, \ldots, h_{k+1}\}$.\\
		
		By hypothesis, the horses in $A = \{h_1,\ldots, h_k\}$ and $B = \{h_2, \ldots, h_{k+1}\}$ have the same colour. Since the horse $h_2$ is in both groups, we deduce that the colour of the horses in $A$ must be the same as of these in $B$.  \\
		
		In conclusion, the horses in $S = A \cup B$ must have the same colour, and the claim holds by induction.	
		\end{graybox}
		
		This proof is flawed. Explain how.
		


%
%\paragraph{The proof above is FALSE: } 
%For $k=1$, $k+1=2$ and $A = \{h_1\}$ and $B = \{h_2\}$, so there are no common horses to $A$ and $B$ an the argument that the colour in $A$ is the same as in $B$ fails!
%
%(it's like we proved that the first domino falls and if a domino falls the next one also falls, except the second one never falls!)

		
	\end{problist}
\end{exercises}
