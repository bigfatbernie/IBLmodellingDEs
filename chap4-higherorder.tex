%%%%%%%%%%%%%%%%%%%%%%%%%%%%%%%%%%%%%%%%%%%%%%%%%%%%%%%%%%%%%%%%%%%%%%%%
%
%		Chapter 4 - Higher-Order Models
%
%%%%%%%%%%%%%%%%%%%%%%%%%%%%%%%%%%%%%%%%%%%%%%%%%%%%%%%%%%%%%%%%%%%%%%%%


\begin{topic}[Higher-Order Models]



\vfil

\begin{center}
\begin{minipage}{500pt}
	\includegraphics*[width=500pt]{images/chap4-xkcd.png}

	\hfill {\footnotesize (image from \href{https://www.xkcd.com/226/}{xkcd - comic \#226})}
\end{minipage}
\end{center}


\end{topic}












%%%%%%%%%%%%%%%%%%%%%%%%%%%%%%
%
%  MODULE - Modelling with Second-Order ODEs
%
%%%%%%%%%%%%%%%%%%%%%%%%%%%%%%



\begin{module}{Modelling with Second-Order ODEs}
	\label{2nd:model}

	In this module you will learn
\begin{itemize}
	\item what is an autonomous differential equation
	\item how to obtain some properties of solutions of autonomous differential equations without solving them
\end{itemize}

\hfill \\


	\begin{exercises}

	\begin{problist}
	
	\prob Consider a mountain with shape $y=f(x)$ a hiker who is climbing down the mountain with horizontal position $x(t)$. She starts at a peak of the mountain at $x_0=0$. As she climbs down the mountain, she notices that from her point-of-view, the rate of change of the slope of the mountain is decreasing linearly with time.
	
		Model the hiker's position.
	
	\prob 
	
		\begin{center}
		\includegraphics*[width=150pt]{images/module18-catenary.pdf}
	\end{center}

	\prob Model the shape of a rope hanging between two poles.

	\begin{center}
		\includegraphics*[width=150pt]{images/module18-suspensionbridge.pdf}
	\end{center}

	\prob Model the shape of the cables of a suspension bridge.
	
	\prob Imagine a cylinder floating vertically partially submerged in a lake. Model the position of its top.
	
	
	\prob Start with the Law of Conservation of Energy and assume a conservative force. Then show that you obtain Newton's Second Law of motion.
	
	\end{problist}
\end{exercises}

\end{module}



\begin{lesson}
	\Title{Modelling with Second-Order ODEs}

	\Heading{Objectives}
	\begin{itemize}
		\item Bla
	\end{itemize}
	
	\Heading{Motivation} 

\end{lesson}




\newpage

\question
	Here are some facts about laptop keys:

\begin{itemize}
\begin{minipage}{.4\textwidth}
\item[\color{Gray}(da)] Each key must also include some damping, so that it doesn't keep oscillating back and forth once pressed.

\item[\color{Gray}(di)] A typical letter key is 15mm$\times$15mm and when pressed has a maximum displacement of 0.5mm.

\item[\color{Gray}(fo)] On average, a person exerts the force of $42\,$N with one finger on a key.
\end{minipage}
\hfill
\begin{minipage}{.4\textwidth}
\item[\color{Gray}(gr)] Gravity is much weaker than the spring that keeps the key in place.

\item[\color{Gray}(hl)] Each key has a spring to make the key return to its original position after being pressed (Hooke's Law: ``the force is proportional to the extension'').

\item[\color{Gray}(lo)] Keys last 50 million presses on average.

\item[\color{Gray}(ve)] Keys can only move vertically.
\end{minipage}
\end{itemize}
	
	
\begin{parts}
	\item Model a laptop keypress.
	\item What happens if the damping system of the key is broken? What happens if the damping system is too strong? How strong should the damping system be?
	\item What happens to the key when the spring breaks?
\end{parts}









%%%%%%%%%%%%%%%%%%%%%%%%%%%%%%
%
%  MODULE - Second-Order Linear ODEs with Constant Coefficients
%
%%%%%%%%%%%%%%%%%%%%%%%%%%%%%%



\begin{module}{Second-Order Linear ODEs with Constant Coefficients}
	\label{2nd:solving}

	In this module you will learn
\begin{itemize}
	\item how to solve this type of ODEs
\end{itemize}

\hfill \\

In this module we will learn how to solve a specific type of Second-Order ODEs: linear second-order ODEs with constant coefficients. These equations have the form
$$
a y''(t)  + b y'(t) + c y(t) = f(t).
$$

\subsection{Homogeneous ODEs}

These are ODEs above with $f(t) \equiv 0$.
So we are trying to solve
$$
a y''(t)  + b y'(t) + c y(t) = 0.
$$

The main idea to solve these problems is the same as for systems: making an \emph{educated guess} that the solution should look like an exponential:
$$
y(t) = e^{rt},
$$
and we need to find which values of $r$ yield solutions.

We do that by plugging this formula for $y(t)$ into the ODE:
\begin{itemize}
	\item $y'(t) = r e^{rt}$
	\item $y''(t) = r^2 e^{rt}$
\end{itemize}

We get
$$
a r^2 e^{rt} + br e^{rt} + c e^{rt} = 0
\quad \Leftrightarrow \quad 
	a r^2 + br + c = 0
$$
and we know how to solve this:
$$
r = \frac{-b \pm \sqrt{b^2-4ac}}{2a}.
$$

That means that we have three possible cases.





\paragraph{\color{cyan}Two real distinct roots.} When $b^2-4ac > 0$, we have two possible values for $r$ that are real numbers: $r_1$ and $r_2$.

Then, similarly to what we did with systems of ODEs, we obtain two solutions
$$
y_1(t) = e^{r_1 t} \quad \text{ and } \quad y_2(t) = e^{r_2 t},
$$
and the general solution is
$$
y(t) = c_1 e^{r_1 t} + c_2 e^{r_2 t}.
$$




\paragraph{\color{cyan}Two complex roots.} When $b^2-4ac<0$, we have two possible values for $r$, but they are complex values:
$$
r_{\pm} = a \pm ib.
$$
\begin{graybox}
What are the value of $a$ and $b$?	
\end{graybox}

Then we have two solutions
$$
y_{+}(t) = e^{(a+ib) t} \quad \text{ and } \quad y_{-}(t) = e^{(a-ib) t},
$$
and the general solution is
$$
y(t) = a_1 e^{(a+ib) t}  + a_2 e^{(a-ib) t}.
$$

Just like we did with systems with complex eigenvalues, we prefer to write the solutions without complex numbers, so we expand it using Euler's formula to get
\begin{align*}
y(t) 	& = a_1 e^{(a+ib) t}  + a_2 e^{(a-ib) t} \\
		& = a_1 e^{at}e^{ibt}  + a_2 e^{at}e^{-ibt} \\
		& = a_1 e^{at} \big( \cos(bt) + i \sin(bt) \big)  + a_2 e^{at} \big( \cos(bt) - i \sin(bt) \big) \\
		& = (a_1+a_2)  \cos(bt)e^{at} + i (a_1-a_2)\sin(bt) e^{at} \\
		& = c_1 \cos(bt)e^{at} + c_2\sin(bt) e^{at}
\end{align*}

\begin{graybox}
How do $c_1$ and $c_2$ depend on $a_1,a_2$?	
\end{graybox}

So another way to write the general solution is
$$
y(t) = c_1 \cos(bt)e^{at} + c_2\sin(bt) e^{at}.
$$






\paragraph{\color{cyan}One real repeated root.} When $b^2-4ac=0$, then we are left with only one value for $r=-\frac{b}{2a}$.

We then have one solution
$$
y_1(t) = e^{-\frac{b}{2a}t}.
$$

\begin{example}
Consider the ODE 
$$
y''(t) + 2y'(t) + y(t) = 0.
$$	

To find the general solution, we assume that the solutions have the form $y(t) = e^{rt}$, which means that $r$ must satisfy
$$
r^2 +2r+1 = 0 
	\quad \Leftrightarrow \quad r=-1,
$$
so $y_1(t) = c_1 e^{-t}$.

Now can we solve this ODE with the following initial conditions?
\begin{itemize}
	\item $y(0)=2$ and $y'(0)=-2$.
	\item $y(0)=2$ and $y'(0)=1$.
\end{itemize}
\end{example}

This previous example, should give a good idea on why having one value for $r$ means that we are missing something. 
We need to find a second solution $y_2(t)$. \\


\begin{graybox}
If we want to find all the divisors of $42$., and we already know that $d_1=2$ is a divisor, then we can use the divisor we know $d_1$ and write 
$$
d_1 \cdot x = 42 
	\quad \Leftrightarrow\quad 2x = 42
	\quad \Leftrightarrow\quad x = 21,
$$
where $x$ is the product of all the other divisors.

We used the divisor we knew $d_1$ to obtain a simpler problem for the other divisors.
\end{graybox}


\subparagraph{\color{cyan}Reduction of Order.} The idea here is the same. We use the solution we found to try to obtain a simpler ODE for the other solution:
$$
y(t) = y_1(t) \cdot u(t),
$$
where $y(t)$ is the solution we are still missing, $y_1(t)$ is the solution we already found, and $u(t)$ is a function. If we find $u(t)$, then we find $y(t)$. We hope that the function $u(t)$ satisfies a simpler problem.

To do that, we need to plug the formula above for $y(t)$ into the original ODE. 

\begin{graybox}
You should do these calculations yourself.
Remember to use the product rule and to be careful not to make any mistakes.	

Also remember that we know the value of $r$
\end{graybox}

We obtain
$$
u''(t) = 0
\quad \Leftrightarrow \quad u(t) = c_1 + c_2 t.
$$

This means that we found 
$$
y(t) = (c_1 + c_2 t) e^{rt}
\quad \Leftrightarrow \quad y(t) = \underbrace{c_1 e^{rt}}_{\substack{\rm previous \\ \text{solution } y_1(t)}} + c_2 t e^{rt}.
$$

The general solution is this
$$
y(t) = c_1 e^{rt} + c_2 t e^{rt},
$$
where $r = -\frac{b}{2a}$.








\subsection{Non-Homogeneous ODEs}

We are trying to solve
$$
a y''(t)  + b y'(t) + c y(t) = f(t),
$$
where $f(t)$ is a known function. \\


\begin{graybox}
If $u(t)$ is the general solution of
$$
ay''(t)+by'(t)+cy(t) = 0,
$$
and $v(t)$ satisfies
$$
ay''(t)+by'(t)+cy(t) = f(t),
$$
then $y(t) = u(t) + v(t)$ gives the general solution of
$$
ay''(t)+by'(t)+cy(t) = f(t).
$$

This is a practice problem at the end of this module.
\end{graybox}


This means that to solve this ODE, we split the general solution into two parts
$$
y(t) = y_c(t) + y_p(t),
$$
where
\begin{itemize}
	\item $y_c(t)$ is called the \emph{complementary solution} and it is the general solution of the corresponding homogenous ODE. It is solved using the technique we studied above.
	\item $y_p(t)$ is called the \emph{particular solution} and it is one function that satisfies the original ODE.
\end{itemize}

\begin{important}
It may seem strange that to solve the original ODE, we need its solution, but what we are trying to do is find \emph{all possible solutions} of the original ODE.

To find all possible solutions of the original ODE, we require two things:
\begin{itemize}
	\item One solution of the original ODE:  $y_p(t)$,
	\item and all possible solution of the homogeneous ODE: $y_c(t)$.
\end{itemize}
\end{important}


We already know how to find the complementary solution, so we will focus our attention on finding one particular solution.

\paragraph{\color{cyan}Method of Undetermined Coefficients.} As you probably have gotten used to by now, this is a method of educated guess-and-check. \\

Let us look at the equation from a different point-of-view
\begin{align*}
a y''(t) + by'(t) + cy(t) & = f(t) \\
\substack{\text{linear combination}\\\text{of function and derivatives}} & = f(t).
\end{align*}

Remember that some functions don't change much when differentiated:
\begin{itemize}
	\item Exponentials $y=ce^{rt}$ don't change their form after differentiation $y'=cre^{rt} = de^{rt}$. They even keep the same exponential term.
	\item Polynomials don't change their form either: their derivative is also a polynomial, with lower degree.
	\item Cosines and Sines alternate between one and the other, so functions of the form $y=c_1 \sin(rt) + c_2\cos(rt)$ don't change after differentiation.
\end{itemize}

This means that, if $f(t)$ is one of these types of function, then $y(t)$ must be of the same form.	


\begin{example}
Find a particular solution for the ODE
$$
y''  - 4y = 10 e^{3t} = (\text{constant}) \cdot (\text{exponential of } 3t).
$$	

Our candidate is
$$
y_p(t) = A e^{3t}.
$$

Now we need to find the constant $A$ by plugging it into the ODE:
$$
9 A e^{3t} - 4 \cdot A e^{3t} = 10 e^{3t}
\quad \Leftrightarrow \quad 
	A = 2,
$$
so $y_p(t) = 2 e^{3t}$ is a particular solution.
\end{example}


\begin{example}
Find a particular solution for the ODE
$$
y''  - 4y = 3t^2+2t = (\text{polynomial of degree 2}).
$$	

Our candidate is
$$
y_p(t) = At^2 + Bt + C.
$$

Now we need to find the constants $A, B, C$ by plugging the formula for $y_p$ into the ODE:
$$
2A - 4At^2 - 4B t - 4C = 3t^2+2t
\quad \Leftrightarrow \quad 
\begin{cases}
A = -\frac34 \\
B = -\frac12 \\
C = \frac{A}{2} = -\frac38.	
\end{cases}
$$
so $y_p(t) = -\frac34 t^2 - \frac{t}{2} - \frac38$ is a particular solution.
\end{example}

There are some more details to deal with when using this method that will be addressed in the core exercises.



	\begin{exercises}

	\begin{problist}

	\prob Find the complementary and particular solutions for the following ODEs
	\begin{enumerate}
		\item $y''-2y'-3y=3e^{2t}$
		\item $y''-2y'-3y=-3te^{-t}$
		\item $y''-9y=t^2e^{-3t}-6$
		\item $y''+2y'-8y=e^{-t}-2e^t$
		\item $y''-y'-6y=\sin(t)$
		\item $y''-y'-6y=\sin(t)+3e^{3t}$
		\item $y''+4y=(2t+1)\sin(t) + 4\cos(2t)$
		\item $y''+y=\cos(2t)+t^3$
		\item $y''-y'-2y=t\cos(t) - t\sin(t)$
		\item $y''+5y'+6y = 2e^{-2t}$
	\end{enumerate}



	\prob What is the form of the particular solution for the ODE
	\begin{multline*}
		y^{(6)} + y^{(5)} -5 y^{(4)} + 31 y'''-176y''+220y' \\
			= (3t-1)e^{2t} + t^3e^{-5t}\sin(3t) + (4t^2-2t) e^{-2t} \sin(3t),
	\end{multline*}
	knowing that 
	\begin{multline*}
		x^6 + x^5 - 5 x^4 + 31 x^3 - 176 x^2 + 220 x \\
			= \big((x^2+2)+9\big)*(x-2)^2*x*(x+5)\quad ?
	\end{multline*}

	\prob What is the form of the particular solution for the ODE
	\begin{multline*}
		y'''' - 4 y''' + 10y'' - 12 y' + 5y \\
			= t e^t + t^2 \cos(2t) - (2t+1) e^t \sin(t),
	\end{multline*}
	knowing that 
	\begin{multline*}
		x^4 - 4x^3 + 10 x^2 - 12 x + 5 \\
			= (x-1)^2 \big( (x-1)^2+4\big)\quad  ?
	\end{multline*}

	\prob Consider the ODE
	$$
	t^2 y''+ty'-9y = 0,
	$$
	and a solution $y_1(t) = t^3$.
	
	\begin{enumerate}
		\item Use the reduction of order technique to deduce the general solution to this problem.
		
		{\bf Hint.} You should find a second-order ODE for $u(t)$ without the term $u(t)$. So define $v(t) = u'(t)$ and solve the first-order ODE for $v(t)$.
		
		
			%	Solution: $tu'' + 7 u' = 0$ \\
			%	Define: $v = u'$ \\
			%	Solve: $tv' + 7 v = 0 \Rightarrow v = c_2 t^{-7}$ \\
			%	Then: $u = \int v = c_2 \int e^{t^2} dt + c_1$


		
		
		\item Find the solution with initial conditions $y(1)=1$ and $y'(1)=-3$.
		\item Find the solution with initial conditions $y(1)=1$ and $y'(1)=3$.
		\item Find the solution with initial conditions $y(1)=1$ and $y'(1)=0$.
	\end{enumerate}


	\prob Consider the ODE
	$$
	t^2 y'' - 3 t y' + 4 y = 0
	$$
	and a solution $y_1(t) = t^2$.
	
	\begin{enumerate}
		\item Use the reduction of order technique to deduce the general solution to this problem.
		
	
			%	Solution: $tu'' + 7 u' = 0$ \\
			%	Define: $v = u'$ \\
			%	Solve: $tv' + 7 v = 0 \Rightarrow v = c_2 t^{-7}$ \\
			%	Then: $u = \int v = c_2 \int e^{t^2} dt + c_1$
		
		\item Find the solution with initial conditions $y(1)=1$ and $y'(1)=2$.
		\item Find the solution with initial conditions $y(1)=0$ and $y'(1)=1$.
	\end{enumerate}



	
	\prob Consider the ODE $a y'' +by'+cy = f(t)$, with complementary solution $y_c(t) =c_1 y_1(t) + c_2 y_2(t)$ and particular solution $y_p(t)$.
	
	Consider also the initial conditions $y(0)=y_0$ and $y'(0)=v_0$.
	
	Show that there exist constants $c_1, c_2$ such that $y(t) = y_c(t) + y_p(t)$ solves the ODE with these initial conditions.
	
	\end{problist}
\end{exercises}

\end{module}



\begin{lesson}
	\Title{Second-Order Linear ODEs with Constant Coefficients}

	\Heading{Objectives}
	\begin{itemize}
		\item Bla
	\end{itemize}
	
	\Heading{Motivation} 

\end{lesson}




\newpage

\question
	Consider the ODE \quad $y''(t) -9y(t) = f(t)$.
\begin{parts}
	\item Find a complementary solution.
	\item Find a particular solution for $f(t) = 14 e^{-4t}$.
	\item Find a particular solution for $f(t) = 9 e^{-3t}$.
	\item Find a particular solution for $f(t) = 10\cos(t)$.
\end{parts}

\bookonlynewpage


\question
	Consider the ODE \quad $y''(t) -2y'(t)+5y(t) = f(t)$. %(roots $r = 1 \pm 2i$)
\begin{parts}
	\item Find a complementary solution.
	\item Find a particular solution for $f(t) = \sin(2t)e^t$.
	\item Find a particular solution for $f(t) = (4t+2)\sin(2t)e^t$.
\end{parts}




\bookonlynewpage


\question
	Consider the ODE \quad $y'' + 3y' = 3t$.
\begin{parts}
	\item Find the complementary solution.
	\item Find a particular solution.
	\item Find the solution that also satisfies
	$$ \begin{cases}
		y(0)=0 \\
		y'(0)=0
	\end{cases}$$
\end{parts}



