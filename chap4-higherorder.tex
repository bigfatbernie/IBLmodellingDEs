%%%%%%%%%%%%%%%%%%%%%%%%%%%%%%%%%%%%%%%%%%%%%%%%%%%%%%%%%%%%%%%%%%%%%%%%
%
%		Chapter 4 - Higher-Order Models
%
%%%%%%%%%%%%%%%%%%%%%%%%%%%%%%%%%%%%%%%%%%%%%%%%%%%%%%%%%%%%%%%%%%%%%%%%


\begin{topic}[Higher-Order Models]



\vfil

\begin{center}
\begin{minipage}{500pt}
	\includegraphics*[width=500pt]{images/chap4-xkcd.png}

	\hfill {\footnotesize (image from \href{https://www.xkcd.com/226/}{xkcd - comic \#226})}
\end{minipage}
\end{center}


\end{topic}












%%%%%%%%%%%%%%%%%%%%%%%%%%%%%%
%
%  MODULE - Modelling with Second-Order ODEs
%
%%%%%%%%%%%%%%%%%%%%%%%%%%%%%%



\begin{module}{Modelling with Second-Order ODEs}
	\label{2nd:model}

	\input{modules/module18-2nd-model.tex}
	\input{modules/module18-2nd-model-exercises.tex}
\end{module}



\begin{lesson}
	\Title{Modelling with Second-Order ODEs}

	\Heading{Objectives}
	\begin{itemize}
		\item Bla
	\end{itemize}
	
	\Heading{Motivation} 

\end{lesson}




\newpage

\question
	Here are some facts about laptop keys:

\begin{itemize}
\begin{minipage}{.4\textwidth}
\item[(da)] Each key must also include some damping, so that it doesn't keep oscillating back and forth once pressed.

\item[(di)] A typical letter key is 15mm$\times$15mm and when pressed has a maximum displacement of 0.5mm.

\item[(fo)] On average, a person exerts the force of $42\,$N with one finger on a key.
\end{minipage}
\hfill
\begin{minipage}{.4\textwidth}
\item[(gr)] Gravity is much weaker than the spring that keeps the key in place.

\item[(hl)] Each key has a spring to make the key return to its original position after being pressed (Hooke's Law: ``the force is proportional to the extension'').

\item[(lo)] Keys last 50 million presses on average.

\item[(ve)] Keys can only move vertically.
\end{minipage}
\end{itemize}
	
	
\begin{parts}
	\item Model a laptop keypress.
	\item What happens if the damping system of the key is broken? What happens if the damping system is too strong? How strong should the damping system be?
	\item What happens to the key when the spring breaks?
\end{parts}









%%%%%%%%%%%%%%%%%%%%%%%%%%%%%%
%
%  MODULE - Second-Order Linear ODEs with Constant Coefficients
%
%%%%%%%%%%%%%%%%%%%%%%%%%%%%%%



\begin{module}{Second-Order Linear ODEs with Constant Coefficients}
	\label{2nd:solving}

	\input{modules/module19-2nd-solve.tex}
	\input{modules/module19-2nd-solve-exercises.tex}
\end{module}



\begin{lesson}
	\Title{Second-Order Linear ODEs with Constant Coefficients}

	\Heading{Objectives}
	\begin{itemize}
		\item Bla
	\end{itemize}
	
	\Heading{Motivation} 

\end{lesson}




\newpage

\question
	Core Exercise with several parts
\begin{parts}
	\item Part 1
	\item Part 2
\end{parts}

\bookonlynewpage


\question
	One more core exercise


